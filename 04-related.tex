\section{Related Work}

\CEU follows the lineage of imperative synchronous languages initiated by
Esterel~\cite{esterel.ieee91}.
These languages typically define time as a discrete sequence of logical
``ticks'' in which multiple simultaneous input events can be
active~\cite{ceu.tecs17}.
%
The presence of multiple inputs requires careful static analysis to detect and
reject programs with \emph{causality cycles} and \emph{schizophrenia problems}%
~\cite{esterel.constructive}.
%
In contrast, \CEU defines time as a discrete sequence of reactions to
unique input events, which is a prerequisite for the concurrency checks that
enable safe shared-memory concurrency, as discussed in
Section~\ref{sec.ceu.shared}.

In most synchronous languages, the behavior of external and internal events is
equivalent.
However, in \CEU, internal events introduce stack-based micro reactions within
external reactions, providing more fine-grained control for intra-reaction
execution.
%
This allows for memory-bounded subroutines that can execute multiple times
during the same external reaction.
%
The synchronous languages Statecharts~\cite{statecharts.variants} and
Statemate~\cite{statecharts.statemate} also distinguish internal from external
events.
In the former, \emph{``reactions to external and internal events (...) can be
sensed only after completion of the step''}.
In the latter, \emph{``the receiving state (of the internal event) acts here as
a function''}.
Although the descriptions suggest a stack-based semantics, we are not aware of
formalizations for these ideas for a deeper comparison with \CEU.

Like \CEU, many other synchronous languages%
~\cite{rp.rc,wsn.protothreads,wsn.sol,rp.synchc,rp.pretc}
also rely on lexical scheduling to preserve determinism.
%
In contrast, in Esterel, the execution order for operations within a reaction
is non-deterministic: \emph{``if there is no control dependency, as in
\code{(call~f1()~||~call~f2())},
the order is unspecified and it would be an error to rely on
it''}~\cite{esterel.primer}.
%
For this reason, Esterel, does not support shared-memory concurrency:
\emph{``if a variable is written by some thread, then it can neither be read
nor be written by concurrent threads''}~\cite{esterel.primer}.
%
Considering the constant and low-level interactions with the underlying
architecture in embedded systems (e.g., direct port manipulation), we believe
that it is advantageous to embrace lexical scheduling as part of the language
specification as a pragmatic design decision to enforce determinism.
%
However, since \CEU statically detects trails not sharing memory, an optimized
scheduler could exploit real parallelism in such cases.

Regarding the integration with $C$ language-based environments, \CEU
supports a finalization mechanism for external resources.
%
In addition, \CEU also tracks pointers representing resources that cross $C$
boundaries and forces the programmer to provide associated finalizers.
%
As far as we know, this extra safety level is unique to \CEU among synchronous
languages.
