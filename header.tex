\usepackage[utf8]{inputenc}
\usepackage[T1]{fontenc}
\usepackage{enumerate}
\usepackage{etoolbox}
\usepackage{mathtools}

% Céu.
\newcommand{\CEU}{\textsc{C\'{e}u}\xspace}

% Notes
\usepackage[textsize=footnotesize]{todonotes}
\newcommand{\rc}[1]{\todo[inline,color=green!40]{\textbf{Rodrigo}: #1}}
\newcommand{\fs}[1]{\todo[inline,color=blue!40]{\textbf{Francisco}: #1}}
\newcommand{\gl}[1]{\todo[inline,color=red!40]{\textbf{Guilherme}: #1}}

% Concrete syntax.
\usepackage{listings}
\lstdefinelanguage{ceu}{%
  language=C,
  morekeywords={%
    @const, @pure, @safe, CHAN, FOREVER, PAR, PROC, SIGNAL, abort, and,
    await, bool, call, class, data, define, deterministic, do, each, else,
    emit, end, escape, event, every, finalize, hor, implementation, in,
    input, interface, loop, min, native, new, nohold, not, or, output, par,
    pool, pure, return, s, signal, spawn, tag, then, this, traverse, until,
    var, watching, when, with},
}
\lstset{
  basicstyle=\ttfamily,
  captionpos=b,
  columns=flexible,
  commentstyle=\rmfamily\itshape,
  escapeinside={||},
  frame=tb,
  keepspaces=true,
  keywordstyle=\bfseries,
  language=ceu,
  mathescape=true,
  numbersep=4pt,
  numberstyle=\scriptsize,
  upquote=true,
}
\newcommand{\code}[1]{{\small{\texttt{#1}}}}
%\let\code\lstinline
\usepackage{xspace}
\newcommand{\ax}{\code{[a]}\xspace}
\newcommand{\bx}{\code{[b]}\xspace}
\def\<#1>{\langle#1\rangle}
\let\nil\varepsilon

% Syntactic sets for the abstract syntax.
\def\P{P}                       % programs
\def\N{N}                       % numbers
\def\E{E}                       % events

% Abstract syntax.
\makeatletter
\def\@int{\mathit{int}}
\def\@ext{\mathit{ext}}
\let\ext\@ext
\def\@ceuop#1{\mathop{\normalfont{\texttt{#1}}}}%
\def\@ceubin#1{\mathbin{\normalfont{\texttt{#1}}}}%
\def\ceu{\protect\@ceu}
\def\@ceu#1{%
  \bgroup
  \def\Id{\mathit{id}}%
  \def\Mem{\@ceuop{mem}}%
  \def\AwaitExt{\@ceuop{await}\nolimits_\@ext}%
  \def\AwaitInt{\@ceuop{await}\nolimits_\@int}%
  \def\EmitInt{\@ceuop{emit}\nolimits_\@int}%
  \def\Break{\@ceuop{break}}%
  \def\IfElse##1##2##3{\@ceuop{if}##1\@ceuop{then}{##2}\@ceuop{else}{##3}}%
  \def\Loop{\@ceuop{loop}}%
  \def\Every{\@ceuop{every}}%
  \def\And{\@ceubin{and}}%
  \def\Or{\@ceubin{or}}%
  \def\Fin{\@ceuop{fin}}%
  \def\AtLoop{\@ceuop{@loop}}%
  \def\AtAnd{\@ceubin{@and}}%
  \def\AtOr{\@ceubin{@or}}%
  \def\CanRun{\@ceuop{@canrun}}%
  \def\Nop{\@ceuop{@nop}}%
  \ensuremath{#1}\ignorespaces
  \egroup
}
\makeatother

% Proof environments: assumptions and proof-by-cases.
\makeatletter
\AtEndPreamble{%
  \theoremstyle{definition}
  \newtheorem{assumption}[theorem]{Assumption}
  %%
  \theoremstyle{remark}
  \newtheorem{case}{Case}
  \AtBeginEnvironment{proof}{\setcounter{case}{0}}
  %%
  \newtheoremstyle{subcase}%
    {\thm@preskip\topsep \divide\thm@preskip\tw@}% space above
    {\thm@postskip\thm@preskip}%                   space below
    {\addtolength{\@totalleftmargin}{\parindent}%
      \addtolength{\linewidth}{-\parindent}%
      \parshape 1 \parindent \linewidth
      \normalfont}% body font
    {\z@}%          indent amount
    {\itshape}%     head font
    {.}%            punctuation after head
    {.5em}%         spacing after head
    {}%             head spec
  \theoremstyle{subcase}
  \newtheorem{subcase}{Case}
  \numberwithin{subcase}{case}
  \AtBeginEnvironment{case}{\setcounter{subcase}{0}}
  %%
  \newtheoremstyle{subsubcase}%
    {\thm@preskip\topsep \divide\thm@preskip\tw@}% space above
    {\thm@postskip\thm@preskip}%                   space below
    {\addtolength{\@totalleftmargin}{\parindent}%
      \addtolength{\linewidth}{-\parindent}%
      \parshape 1 2\parindent \linewidth
      \normalfont}% body font
    {\z@}%          indent amount
    {\itshape}%     head font
    {.}%            punctuation after head
    {.5em}%         spacing after head
    {}%             head spec
  \theoremstyle{subsubcase}
  \newtheorem{subsubcase}{Case}
  \numberwithin{subsubcase}{subcase}
  \AtBeginEnvironment{subcase}{\setcounter{subsubcase}{0}}
}
\makeatother

% Proof trees (used to typeset proofs and rules).
\usepackage{bussproofs}
\def\labelSpacing{0em}
\def\ScoreOverhang{0em}

% Functions bcast, clear, eval, and isblocked.
\makeatletter
\let\op\operatorname
\newcommand{\opi}[1]{\op{\mathit{#1}}}
%%
\def\bcast{\opi{bcast}}
\def\clear{\opi{clear}}
\def\eval{\opi{eval}}
\def\isblockedext{\opi{isblocked}_\@ext}
\def\isblocked{\opi{isblocked}}
\def\pot{\opi{pot}}
\def\rank{\opi{rank}}
\makeatother

% Rule labels.
\makeatletter
\def\@R#1{{\bfseries#1}}
\def\Rtag#1{\tag{\@R{#1}}\label{R:#1}}
\def\R#1{\hyperref[R:#1]{\@R{#1}}}
\makeatother

% Transition arrows.
\makeatletter
\newcommand{\@trans}[2][]{%
  \setbox0=\hbox{$\mathord{\longrightarrow}$}%
  \mathbin{%
    \mathrlap{\longrightarrow}%
    \notblank{#2}{%
      \rlap{\raisebox{-3pt}%
        {\makebox[\wd0][c]{\ensuremath{\scriptstyle#2}}}}%
    }{}%
    \notblank{#1}{%
      \rlap{\raisebox{5pt}%
        {\makebox[\wd0][c]{\ensuremath{\scriptstyle#1}}}}%
    }{}%
    \phantom{\longrightarrow}%
  }%
}
\newcommand{\trans}[1][]{\@trans[#1]{}}
\newcommand{\@out}{\mathit{out}}
\newcommand{\@nst}{\mathit{nst}}
\newcommand{\out}[1][]{\@trans[#1]{\@out\,\;}}
\newcommand{\outpush}{\out[\mathit{push}]}
\newcommand{\outpop}{\out[\mathit{pop}]}
\newcommand{\nst}[1][]{\@trans[#1]{\@nst\,\;}}
\makeatother

% Irreducible predicate.
\makeatletter
\newcommand{\Hout}{{\#_{\@out}}}
\newcommand{\Hnst}{{\#\@nst}}
\makeatother
