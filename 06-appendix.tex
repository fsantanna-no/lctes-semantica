\appendix
\section{Proofs}
\label{sec.proofs}

\theoremstyle{remark}
\newtheorem{case}{Case}
\makeatletter
\newtheoremstyle{subcase}%
  {\thm@preskip\topsep \divide\thm@preskip\tw@}% space above
  {\thm@postskip\thm@preskip}% space below
  {\addtolength{\@totalleftmargin}{\parindent}%
    \addtolength{\linewidth}{-\parindent}%
    \parshape 1 \parindent \linewidth
    \normalfont}% body font
  {\z@}% indent amount
  {\itshape}% head font
  {.}% punctuation after head
  {.5em}% spacing after head
  {}% head spec
\makeatletter
\theoremstyle{subcase}
\newtheorem{subcase}{Case}
\numberwithin{subcase}{case}
\AtBeginEnvironment{proof}{\setcounter{case}{0}}
\AtBeginEnvironment{case}{\setcounter{subcase}{0}}

% \makeatletter
% \newcommand\case{\@startsection{paragraph}{4}{\z@}%
%    {-.5\baselineskip \@plus -2\p@ \@minus -.2\p@}%
%    {-3.5\p@}%
%    {\@parfont}}
%  \def\@parfont{\itshape}
% \makeatother

\begin{lemma}
  If~$\delta\out\delta_1$ and~$\delta\out\delta_2$ then~$\delta_1=\delta_2$.
\end{lemma}
\begin{proof}
  The lemma is vacuously true if~$\delta$ cannot be advanced by~$\out$
  transitions.  Suppose that is not the case and let~$\delta=\<p,n,e>$,
  $\delta_1=\<p_1,n_1,e_1>$ and~$\delta_2=\<p_2,n_2,e_2>$.  Then, there are
  two possibilities.
  \begin{case}
    $e\ne\nil$.  Both transitions are applications of rule~\R{push}.
    Hence~$p_1=p_2=\bcast(p,e)$, $n_1=n_2=n+1$, and~$e_1=e_2=\nil$.
  \end{case}
  \begin{case}
    $e=\nil$.  Both transitions are applications of rule~\R{pop}.
    Hence~$p_1=p_2=p$, $n_1=n_2=n-1$, and~$e_1=e_2=\nil$.\qedhere
  \end{case}
\end{proof}

\begin{lemma}
  If~$\delta\nst\delta_1$ and~$\delta\nst\delta_2$ then~$\delta_1=\delta_2$.
\end{lemma}
\begin{proof}
  By induction on the structure of~$\nst$ derivations.  The lemma is
  vacuously true if~$\delta$ cannot be advanced by~$\nst$ transitions.
  Suppose that is not the case and let~$\delta=\<p,n,e>$,
  $\delta_1=\<p_1,n_1,e_1>$ and~$\delta_2=\<p_2,n_2,e_2>$.  Then, by the
  hypothesis of the lemma, there are derivations~$\pi_1$ and~$\pi_2$ such
  that
  \begin{align*}
    \pi_1&\Vdash\<p,n,e>\nst\<p_1,n_1,e_1>\\
    \pi_2&\Vdash\<p,n,e>\nst\<p_2,n_2,e_2>
  \end{align*}
  i.e., the conclusion of~$\pi_1$ is~$\<p,n,e>\nst\<p_1,n_1,e_1>$ and the
  conclusion of~$\pi_2$ is~$\<p,n,e>\nst\<p_2,n_2,e_2>$.

  By the definition of transition~$\nst$, we have that~$e=\nil$ and
  $n_1=n_2=n$.  It remains to be shown that~$p_1=p_2$ and~$e_1=e_2$.

  Depending on the structure of program~$p$, the following~11 cases are
  possible.  (Note that~$p$ cannot be an~$\ceu{\AwaitExt}$,
  $\ceu{\AwaitInt}$, $\ceu{\Break}$, $\ceu{\Every}$, $\ceu{\Fin}$,
  or~$\ceu{\Nop}$ expression as there are no~$\nst$ rules to transition such
  programs.)

  \begin{case}
    $p=\ceu{\Mem(\Id)}$.
    %%
    Derivations~$\pi_1$ and~$\pi_2$ are instances of rule~\R{mem}, i.e.,
    their conclusions are obtained by an application of this rule.
    Hence~$p_1=p_2=\ceu{\Nop}$ and~$e_1=e_2=\nil$.
  \end{case}

  \begin{case}
    $p=\ceu{\EmitInt(e')}$.
    %%
    Derivations~$\pi_1$ and~$\pi_2$ are instances of rule~\R{emit-int}.
    Hence~$p_1=p_2=\ceu{\CanRun(n)}$ and~$e_1=e_2=e'$.
  \end{case}

  \begin{case}
    $p=\ceu{\CanRun(n)}$.
    %%
    Derivations~$\pi_1$ and~$\pi_2$ are instances of rule~\R{can-run}.
    Hence~$p_1=p_2=\ceu{\Nop}$ and~$e_1=e_2=\nil$.
  \end{case}

  \begin{case}
    $p=\ceu{\Ifelse{\Mem(\Id)}{p'}{p''}}$.  There are two subcases.
    \begin{subcase}
      $\eval(\ceu{\Mem(\Id)})$ is true.
      %%
      Derivations~$\pi_1$ and~$\pi_2$ are instances of rule~\R{if-true}.
      Hence~$p_1=p_2=p'$ and~$e_1=e_2=\nil$.
    \end{subcase}
    \begin{subcase}
      $\eval(\ceu{\Mem(\Id)})$ is false.
      %%
      Derivations~$\pi_1$ and~$\pi_2$ are instances of rule~\R{if-false}.
      Hence~$p_1=p_2=p''$ and~$e_1=e_2=\nil$.
    \end{subcase}
  \end{case}

  \begin{case}
    $p=\ceu{p';\,p''}$.  There are three subcases.
    \begin{subcase}
      $p'=\ceu{\Nop}$.
      %%
      Derivations~$\pi_1$ and~$\pi_2$ are instances of rule~\R{seq-nop}.
    \end{subcase}
    \begin{subcase}
      $p'=\ceu{\Break}$.
      %%
      Derivations~$\pi_1$ and~$\pi_2$ are instances of rule~\R{seq-brk}.
    \end{subcase}
    \begin{subcase}
      $p'\ne\ceu{\Nop},\ceu{\Break}$.
      %%
      Derivations~$\pi_1$ and~$\pi_2$ are instances of rule~\R{seq-adv}.
    \end{subcase}
  \end{case}

  \begin{case}
    $p=\ceu{\Loop{p'}}$.
    %%
    Derivations~$\pi_1$ and~$\pi_2$ are instances of rule~\R{loop-expd}.
  \end{case}

  \begin{case}
    $p=\ceu{p'\AtLoop{p''}}$.  There are three subcases.
    \begin{subcase}
      $p'=\ceu{\Nop}$.
      %%
      Derivations~$\pi_1$ and~$\pi_2$ are instances of rule~\R{loop-nop}.
    \end{subcase}
    \begin{subcase}
      $p'=\ceu{\Break}$.
      %%
      Derivations~$\pi_1$ and~$\pi_2$ are instances of rule~\R{loop-break}.
    \end{subcase}
    \begin{subcase}
      $p'\ne\ceu{\Nop},\ceu{\Break}$.
      %%
      Derivations~$\pi_1$ and~$\pi_2$ are instances of rule~\R{loop-adv}.
    \end{subcase}
  \end{case}

  \begin{case}
    $p=\ceu{p'\And{p''}}$.
    %%
    Derivations~$\pi_1$ and~$\pi_2$ are instances of rule~\R{and-expd}.
  \end{case}

  \begin{case}
    $p=\ceu{p'\AtAnd{p''}}$.
  \end{case}

  \begin{case}
    $p=\ceu{p'\Or{p''}}$.
    %%
    Derivations~$\pi_1$ and~$\pi_2$ are instances of rule~\R{and-expd}.
  \end{case}

  \begin{case}
    $p=\ceu{p'\AtOr{p''}}$.
  \end{case}
\end{proof}

% LocalWords:  subcase subcases
