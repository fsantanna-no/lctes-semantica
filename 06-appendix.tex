\appendix
\section{Proofs}
\label{sec.proofs}

% \makeatletter
% \newcommand\case{\@startsection{paragraph}{4}{\z@}%
%    {-.5\baselineskip \@plus -2\p@ \@minus -.2\p@}%
%    {-3.5\p@}%
%    {\@parfont}}
%  \def\@parfont{\itshape}
% \makeatother


\begin{lemma}\label{lem.det-out}
  If~$\delta\out\delta_1$ and~$\delta\out\delta_2$ then~$\delta_1=\delta_2$.
\end{lemma}
\begin{proof}
  The lemma is vacuously true if~$\delta$ cannot be advanced by~$\out$
  transitions.  Suppose that is not the case and let~$\delta=\<p,n,e>$,
  $\delta_1=\<p_1,n_1,e_1>$ and~$\delta_2=\<p_2,n_2,e_2>$.  Then, there are
  two possibilities.
  \begin{case}
    $e\ne\nil$.  Both transitions are applications of~\R{push}.
    Hence~$p_1=p_2=\bcast(p,e)$, $n_1=n_2=n+1$, and~$e_1=e_2=\nil$.
  \end{case}
  \begin{case}
    $e=\nil$.  Both transitions are applications of~\R{pop}.
    Hence~$p_1=p_2=p$, $n_1=n_2=n-1$, and~$e_1=e_2=\nil$.\qedhere
  \end{case}
\end{proof}


\begin{theorem}\label{thm.det-out-pop-n}
  If~$\delta\out[n]\delta_1$ and~$\delta\out[n]\delta_2$
  then~$\delta_1=\delta_2$.
\end{theorem}
\begin{proof}
  By induction on~$n$.
  %%
  The theorem is trivially true for~$n=0$ and follows directly from
  Lemma~\ref{lem.det-out} for~$n=1$.  (Note that for~$n>1$ the theorem is
  vacuously true for~$\outpush$ transitions---by the format of
  rules~\R{push} and~\R{pop}, transitions~$\outpush$ cannot be applied more
  than once in a row and cannot occur after a~$\outpop$ transition.)

  Suppose
  \[
    \delta\out[1]\delta_1'\out[n-1]\delta_1
    \quad\text{and}\quad
    \delta\out[1]\delta_2'\out[n-1]\delta_2,
  \]
  for some~$n>1$ and~$\delta_1'$, $\delta_2'\in\Delta$.
  %%
  By Lemma~\ref{lem.det-out}, $\delta_1'=\delta_2'$.  By induction
  hypothesis, $\delta_1=\delta_2$.\qedhere
\end{proof}


\begin{lemma}\label{lem.det-nst}
  If~$\delta\nst\delta_1$ and~$\delta\nst\delta_2$ then~$\delta_1=\delta_2$.
\end{lemma}
\begin{proof}
  By induction on the structure of~$\nst$ derivations.  The lemma is
  vacuously true if~$\delta$ cannot be advanced by~$\nst$ transitions.
  Suppose that is not the case and let~$\delta=\<p,n,e>$,
  $\delta_1=\<p_1,n_1,e_1>$ and~$\delta_2=\<p_2,n_2,e_2>$.  Then, by the
  hypothesis of the lemma, there are derivations~$\pi_1$ and~$\pi_2$ such
  that
  \begin{align*}
    \pi_1&\Vdash\<p,n,e>\nst\<p_1,n_1,e_1>\\
    \pi_2&\Vdash\<p,n,e>\nst\<p_2,n_2,e_2>
  \end{align*}
  i.e., the conclusion of~$\pi_1$ is~$\<p,n,e>\nst\<p_1,n_1,e_1>$ and the
  conclusion of~$\pi_2$ is~$\<p,n,e>\nst\<p_2,n_2,e_2>$.

  By the definition of transition~$\nst$, we have that~$e=\nil$ and
  $n_1=n_2=n$.  It remains to be shown that~$p_1=p_2$ and~$e_1=e_2$.

  Depending on the structure of program~$p$, the following~11 cases are
  possible.  (Note that~$p$ cannot be an~$\ceu{\AwaitExt}$,
  $\ceu{\AwaitInt}$, $\ceu{\Break}$, $\ceu{\Every}$, $\ceu{\Fin}$,
  or~$\ceu{\Nop}$ expression as there are no~$\nst$ rules to transition such
  programs.)

  \begin{case}
    $p=\ceu{\Mem(\Id)}$.
    %%
    Then derivations~$\pi_1$ and~$\pi_2$ are instances of rule~\R{mem},
    i.e., their conclusions are obtained by an application of this rule.
    Hence~$p_1=p_2=\ceu{\Nop}$ and~$e_1=e_2=\nil$.
  \end{case}

  \begin{case}
    $p=\ceu{\EmitInt(e')}$.
    %%
    Then~$\pi_1$ and~$\pi_2$ are instances of~\R{emit-int}.
    Hence~$p_1=p_2=\ceu{\CanRun(n)}$ and~$e_1=e_2=e'$.
  \end{case}

  \begin{case}
    $p=\ceu{\CanRun(n)}$.
    %%
    Then~$\pi_1$ and~$\pi_2$ are instances of~\R{can-run}.
    Hence~$p_1=p_2=\ceu{\Nop}$ and~$e_1=e_2=\nil$.
  \end{case}

  \begin{case}
    $p=\ceu{\Ifelse{\Mem(\Id)}{p'}{p''}}$.  There are two subcases.
    \begin{subcase}
      $\eval(\ceu{\Mem(\Id)})$ is true.
      %%
      Then~$\pi_1$ and~$\pi_2$ are instances of~\R{if-true}.
      Hence~$p_1=p_2=p'$ and~$e_1=e_2=\nil$.
    \end{subcase}
    \begin{subcase}
      $\eval(\ceu{\Mem(\Id)})$ is false.
      %%
      Then~$\pi_1$ and~$\pi_2$ are instances of~\R{if-false}.
      Hence~$p_1=p_2=p''$ and~$e_1=e_2=\nil$.
    \end{subcase}
  \end{case}

  \begin{case}
    $p=\ceu{p';\,p''}$.  There are three subcases.
    \begin{subcase}
      $p'=\ceu{\Nop}$.
      %%
      Then~$\pi_1$ and~$\pi_2$ are instances of~\R{seq-nop}.
      Hence~$p_1=p_2=p''$ and~$e_1=e_2=\nil$.
    \end{subcase}
    \begin{subcase}
      $p'=\ceu{\Break}$.
      %%
      Then~$\pi_1$ and~$\pi_2$ are instances of~\R{seq-brk}.
      Hence~$p_1=p_2=\ceu{\Break}$ and~$e_1=e_2=\nil$.
    \end{subcase}
    \begin{subcase}
      $p'\ne\ceu{\Nop},\ceu{\Break}$.
      %%
      Then~$\pi_1$ and~$\pi_2$ are instances of~\R{seq-adv}.
      Thus there are derivations~$\pi_1'$ and~$\pi_2'$ such that
      \begin{align*}
        \pi_1'&\Vdash\<p',n,\nil>\nst\<p_1',n,e_1'>\\
        \pi_2'&\Vdash\<p',n,\nil>\nst\<p_2',n,e_2'>
      \end{align*}
      for some~$p_1',p_2'\in\P$ and~$e_1',e_2'\in\E$.  By induction
      hypothesis, $p_1'=p_2'$ and~$e_1'=e_2'$.
      Hence~$p_1=\ceu{p_1';p''}=\ceu{p_2';p''}=p_2$ and~$e_1=e_1'=e_2'=e_2$.
    \end{subcase}
  \end{case}

  \begin{case}
    $p=\ceu{\Loop{p'}}$.
    %%
    Then~$\pi_1$ and~$\pi_2$ are instances of~\R{loop-expd}.
    Hence~$p_1=p_2=\ceu{p'\AtLoop{p'}}$ and~$e_1=e_2=\nil$.
  \end{case}

  \begin{case}
    $p=\ceu{p'\AtLoop{p''}}$.  There are three subcases.
    \begin{subcase}
      $p'=\ceu{\Nop}$.
      %%
      Then~$\pi_1$ and~$\pi_2$ are instances of~\R{loop-nop}.
      Hence~$p_1=p_2=\ceu{\Loop{p''}}$ and~$e_1=e_2=\nil$.
    \end{subcase}
    \begin{subcase}
      $p'=\ceu{\Break}$.
      %%
      Then~$\pi_1$ and~$\pi_2$ are instances of~\R{loop-break}.
      Hence~$p_1=p_2=\ceu{\Nop}$ and~$e_1=e_2=\nil$.
    \end{subcase}
    \begin{subcase}
      $p'\ne\ceu{\Nop},\ceu{\Break}$.
      %%
      Then~$\pi_1$ and~$\pi_2$ are instances of~\R{loop-adv}.
      Thus there are derivations~$\pi_1'$ and~$\pi_2'$ such that
      \begin{align*}
        \pi_1'&\Vdash\<p',n,\nil>\nst\<p_1',n,e_1'>\\
        \pi_2'&\Vdash\<p',n,\nil>\nst\<p_2',n,e_2'>
      \end{align*}
      for some~$p_1',p_2'\in\P$ and~$e_1',e_2'\in\E$.  By induction
      hypothesis, $p_1'=p_2'$ and~$e_1'=e_2'$.
      Hence~$p_1=\ceu{p_1'\AtLoop{p''}}=\ceu{p_2'\AtLoop{p''}}=p_2$
      and~$e_1=e_1'=e_2'=e_2$.
    \end{subcase}
  \end{case}

  \begin{case}
    $p=\ceu{p'\And{p''}}$.
    %%
    Then~$\pi_1$ and~$\pi_2$ are instances of~\R{and-expd}.
    Hence~$p_1=p_2=\ceu{{p'}\And{(\CanRun(n);\,p'')}}$ and~$e_1=e_2=\nil$.
  \end{case}

  \begin{case}
    $p=\ceu{p'\AtAnd{p''}}$.  There are two subcases.
    \begin{subcase}
      $\isblocked(p',n)$~is false.  There are three subcases.
      \begin{subsubcase}
        $p'=\ceu{\Nop}$.
        %%
        Then~$\pi_1$ and~$\pi_2$ are instances of~\R{and-nop1}.
        Hence~$p_1=p_2=p''$ and~$e_1=e_2\nil$.
      \end{subsubcase}
      \begin{subsubcase}\label{lem.det-nst.and-brk1}
        $p'=\ceu{\Break}$.
        %%
        Then~$\pi_1$ and~$\pi_2$ are instances of~\R{and-brk1}.
        Hence~$p_1=p_2=\ceu{\clear(p'');\Break}$ and~$e_1=e_2\nil$.
      \end{subsubcase}
      \begin{subsubcase}\label{lem.det-nst.and-adv1}
        $p'\ne\ceu{\Nop},\ceu{\Break}$.
        %%
        Then~$\pi_1$ and~$\pi_2$ are instances of~\R{and-adv1}.
        Thus there are derivations~$\pi_1'$ and~$\pi_2'$ such that
        \begin{align*}
          \pi_1'&\Vdash\<p',n,\nil>\nst\<p_1',n,e_1'>\\
          \pi_2'&\Vdash\<p',n,\nil>\nst\<p_2',n,e_2'>
        \end{align*}
        for some~$p_1',p_2'\in\P$ and~$e_1',e_2'\in\E$.  By induction
        hypothesis, $p_1'=p_2'$ and~$e_1'=e_2'$.
        Hence~$p_1=\ceu{{p_1'}\And{p''}}=\ceu{{p_2'}\And{p''}}=p_2$
        and~$e_1=e_1'=e_2'=e_2$.
      \end{subsubcase}
    \end{subcase}
    \begin{subcase}
      $\isblocked(p',n)$~is true.  There are three subcases.
      \begin{subsubcase}
        $p''=\ceu{\Nop}$.
        %%
        Then~$\pi_1$ and~$\pi_2$ are instances of~\R{and-nop2}.
        Hence~$p_1=p_2=p'$ and~$e_1=e_2\nil$.
      \end{subsubcase}
      \begin{subsubcase}\label{lem.det-nst.and-brk2}
        $p''=\ceu{\Break}$.
        %%
        Then~$\pi_1$ and~$\pi_2$ are instances of~\R{and-brk2}.
        Hence~$p_1=p_2=\ceu{\clear(p');\Break}$ and~$e_1=e_2=\nil$.
      \end{subsubcase}
      \begin{subsubcase}\label{lem.det-nst.and-adv2}
        $p''\ne\ceu{\Nop},\ceu{\Break}$.
        %%
        Then~$\pi_1$ and~$\pi_2$ are instances of~\R{and-adv2}.  Thus there
        are derivations~$\pi_1''$ and~$\pi_2''$ such that
        \begin{align*}
          \pi_1''&\Vdash\<p'',n,\nil>\nst\<p_1'',n,e_1''>\\
          \pi_2''&\Vdash\<p'',n,\nil>\nst\<p_2'',n,e_2''>
        \end{align*}
        for some~$p_1'',p_2''\in\P$ and~$e_1'',e_2''\in\E$.  By induction
        hypothesis, $p_1''=p_2''$ and~$e_1''=e_2''$.
        Hence~$p_1=\ceu{{p'}\And{p_1''}}=\ceu{{p'}\And{p_2''}}=p_2$
        and~$e_1=e_1''=e_2''=e_2$.
      \end{subsubcase}
    \end{subcase}
  \end{case}

  \begin{case}
    $p=\ceu{p'\Or{p''}}$.
    %%
    Then~$\pi_1$ and~$\pi_2$ are instances of~\R{or-expd}.
    Hence~$p_1=p_2=\ceu{{p'}\Or{(\CanRun(n);\,p'')}}$ and~$e_1=e_2=\nil$.
  \end{case}

  \begin{case}
    $p=\ceu{p'\AtOr{p''}}$.  There are two subcases.
    \begin{subcase}
      $\isblocked(p',n)$~is false.  There are three subcases.
      \begin{subsubcase}
        $p'=\ceu{\Nop}$.
        %%
        Then~$\pi_1$ and~$\pi_2$ are instances of~\R{or-nop1}.
        Hence~$p_1=p_2=\clear(p'')$ and~$e_1=e_2\nil$.
      \end{subsubcase}
      \begin{subsubcase}
        $p'=\ceu{\Break}$.
        %%
        Similar to Case~\ref{lem.det-nst.and-brk1}.
      \end{subsubcase}
      \begin{subsubcase}
        $p'\ne\ceu{\Nop},\ceu{\Break}$.
        %%
        Similar to Case~\ref{lem.det-nst.and-adv1}.
      \end{subsubcase}
    \end{subcase}
    \begin{subcase}
      $\isblocked(p',n)$~is true.  There are three subcases.
      \begin{subsubcase}
        $p''=\ceu{\Nop}$.
        %%
        Then~$\pi_1$ and~$\pi_2$ are instances of~\R{or-nop1}.
        Hence~$p_1=p_2=\clear(p')$ and~$e_1=e_2\nil$.
      \end{subsubcase}
      \begin{subsubcase}
        $p''=\ceu{\Break}$.
        %%
        Similar to Case~\ref{lem.det-nst.and-brk2}.
      \end{subsubcase}
      \begin{subsubcase}
        $p''\ne\ceu{\Nop},\ceu{\Break}$.
        %%
        Similar to Case~\ref{lem.det-nst.and-adv2}.
        %%
        \qedhere
      \end{subsubcase}
    \end{subcase}
  \end{case}
\end{proof}


\begin{theorem}\label{thm.det-nst-n}
  If~$\delta\nst[n]\delta_1$ and~$\delta\nst[n]\delta_2$
  then~$\delta_1=\delta_2$.
\end{theorem}
\begin{proof}
  Similar to the proof of Theorem~\ref{thm.det-out-pop-n}.
\end{proof}


\begin{definition}\label{def.Hnst}
  A description~$\delta=\<p,n,e>$ is \emph{nested-irre\-ducible}
  iff~$e\ne\nil$ or~$p=\ceu{\Nop},\ceu{\Break}$ or~$\isblocked(p,n)$~is
  true.  Nested-irreducible descriptions serve as normal forms for~$\nst$
  transitions: they embody the result of an exhaustive number of~$\nst$
  transitions.  We will write~$\delta_\Hnst$ to indicate that
  description~$\delta$ is nested-irreducible.
\end{definition}

The next lemma justifies the use of qualifier ``irreducible'' in
Definition~\ref{def.Hnst}.


\begin{lemma}\label{lem.irr-nst-n}
  If~$\delta\nst[n]\delta_\Hnst'$ then, for all~$i\ne{n}$, there is
  no~$\delta_\Hnst''$ such that~$\delta\nst[i]\delta''_\Hnst$.
\end{lemma}
\begin{proof}
  By contradiction on the hypothesis that there is such~$i$.
  %%
  Let~$\delta\nst[n]\delta'_\Hnst$, for some~$n\ge0$.
  There are two cases.
  \begin{case}\label{lem.irr-nst-n-case1}
    Suppose there are~$i>n$ and~$\delta''_\Hnst$ such
    that~$\delta\nst[i]\delta''$.
    %%
    Then, by definition of~$\nst[i]$,
    \begin{equation}\label{lem.irr-nst-n-eq1}
      \delta\nst[n]\delta'\nst[n+1]\delta_1'\nst[n+2]\cdots\nst[i]\delta''.
    \end{equation}
    Since~$\delta'=\<p',n,e'>$ is nested-irreducible,
    $p=\ceu{\Nop},\ceu{\Break}$ or~$\isblocked(p',n)$.  In either case, by
    the definition of~$\nst$, there is no~$\delta_1'$ such
    that~$\delta'\nst[1]\delta_1'$, which
    contradicts~\eqref{lem.irr-nst-n-eq1}.  Therefore, no such~$i$ can
    exist.
  \end{case}
  \begin{case}
    Suppose there are~$0\le{i}<n$ and~$\delta''_\Hnst$ such
    that~$\delta\nst[i]\delta''$.  Then since~$n>i$, by
    Case~\ref{lem.irr-nst-n-case1}, $\delta'$~could not exist, which is
    absurd.  Therefore, the assumption that there is such~$i$ is
    false.\qedhere
  \end{case}
\end{proof}

The next lemma establishes some basic properties of sequences of~$\nst$
transitions.


\begin{lemma}\label{lem.props-nst-n}
  If~$\<p_1,n,e>\nst[n]\<p_1',n,e'>$ then, for any~$p_2$:
  \begin{enumerate}[(a)]
  \item\label{lem.props-nst-n.a}
    $\<\ceu{p_1;\,p_2},n,e>\nst[n]\<p_1';p_2,n,e'>$;
    %%
  \item\label{lem.props-nst-n.b}
    $\<\ceu{p_1\AtLoop{p_2}},n,e>\nst[n]\<\ceu{p_1'\AtLoop{p_2}},n,e'>$;
    %%
  \item\label{lem.props-net-n.c}
    $\<\ceu{{p_1}\AtAnd{p_2}},n,e>\nst[n]\<\ceu{{p_1'}\AtAnd{p_2}},n,e'>$;
    %%
  \item\label{lem.props-net-n.d}
    $\<\ceu{{p_1}\AtOr{p_2}},n,e>\nst[n]\<\ceu{{p_1}'\AtOr{p_2}},n,e'>$.
  \end{enumerate}
\end{lemma}
\begin{proof}
  By induction on~$n$.
  %%
  \begin{enumerate}[(a)]
  \item The lemma is trivially true for~$n=0$, as~$p_1=p_1'$, and follows
    directly from~\R{seq-adv} for~$n=1$.  Suppose
    \begin{equation}
      \label{lem.props-nst-n.a.eq1}
      \<p_1,n,e>\nst[1]\<p_1'',n,e''>\nst[n-1]\<p_1',n,e'>,
    \end{equation}
    for some~$n>1$.  Then~$\<p_1'',n,e''>$ is not nested-irreducible, i.e.,
    $e=\nil$ and~$p\ne{\ceu{\Nop},\ceu{\Break}}$ and~$\isblocked(p_1,n)$ is
    false.  By~\eqref{lem.props-nst-n.a.eq1} and by~\R{seq-adv},
    \begin{equation}
      \label{lem.props-nst-n.a.eq2}
      \<\ceu{p_1;\,p_2},n,e>\nst[1]\<\ceu{p_1'';\,p_2},n,e''>.
    \end{equation}
    From~\eqref{lem.props-nst-n.a.eq1}, by induction hypothesis,
    \begin{equation}
      \label{lem.props-nst-n.a.eq3}
      \<\ceu{p_1'';\,p_2},n,e''>\nst[n-1]\<\ceu{p_1';\,p_2},n,e'>.
    \end{equation}
    From~\eqref{lem.props-nst-n.a.eq2} and~\eqref{lem.props-nst-n.a.eq3},
    \[
      \<\ceu{p_1;\,p_2},n,e>\nst[n]\<\ceu{p_1';\,p_2},n,e'>.
    \]
  \end{enumerate}
\end{proof}


\begin{theorem}\label{thm.term-nst-*}
  For any~$\delta$ there is a~$\delta'_\Hnst$ such
  that~$\delta\nst[*]\delta'_\Hnst$.
\end{theorem}
\begin{proof}
  By induction on the structure of programs.
  %%
  Let~$\delta=\<p,n,\nil>$.  The theorem is trivially true if~$\delta$ is
  nested-irreducible itself, as by definition~$\delta\nst[0]\delta_\Hnst$.
  Suppose that is not the case.  Then, depending on the structure of~$p$,
  there are~11 possibilities.  In each one of them, we show that
  such~$\delta'_\Hnst$ indeed exists.
  \begin{case}
    $p=\ceu{\Mem(\Id)}$.
    %%
    Then, by~\R{mem},
    \[
      \<\ceu{\Mem(\Id)},n,\nil>\nst[1]\<\ceu{\Nop},n,\nil>_\Hnst\,.
    \]
  \end{case}

  \begin{case}
    $p=\ceu{\EmitInt(e)}$.
    %%
    Then, by~\R{emit-int},
    \[
      \<\ceu{\EmitInt(e)},n,\nil>\nst[1]\<\ceu{\CanRun(n)},n,e>_\Hnst\,.
    \]
  \end{case}

  \begin{case}
    $p=\ceu{\CanRun(n)}$.
    %%
    Then, by~\R{can-run},
    \[
      \<\ceu{\CanRun(n)},n,\nil>\nst[1]\<\ceu{\Nop},n,\nil>_\Hnst\,.
    \]
  \end{case}

  \begin{case}
    $p=\ceu{\Ifelse{\Mem(\Id)}{p'}{p''}}$.
    %%
    There are two subcases.
    \begin{subcase}
      $\eval(\ceu{\Mem(\Id)})$~is true.
      %%
      Then, by~\R{if-true} and by the induction hypothesis, there is
      a~$\delta'$ such that
      \begin{align*}
        \<\ceu{\Ifelse{\Mem(\Id)}{p'}{p''}},n,\nil>
        &\nst[1]\<p',n,e>\\
        &\nst[*]\delta'_\Hnst\,.
      \end{align*}
    \end{subcase}
    \begin{subcase}
      $\eval(\ceu{\Mem(\Id)})$~is false.
      %%
      Then, by~\R{if-false} and by the induction hypothesis, there is
      a~$\delta'$ such that
      \begin{align*}
        \<\ceu{\Ifelse{\Mem(\Id)}{p'}{p''}},n,\nil>
        &\nst[1]\<p'',n,e>\\
        &\nst[*]\delta'_\Hnst\,.
      \end{align*}
    \end{subcase}
  \end{case}

  \begin{case}
    $p=\ceu{p';\,p''}$.
    %%
    There are three subcases.
    \begin{subcase}
      \label{thm.term-nst-*.seq-nop}
      $p'=\ceu{\Nop}$.
      %%
      Then, by~\R{seq-nop} and by the induction hypothesis, there is
      a~$\delta'$ such that
      \[
        \<\ceu{\Nop;\,p''},n,\nil>
        \nst[1]\<p'',n,e>\nst[*]\delta'_\Hnst\,.
      \]
    \end{subcase}
    \begin{subcase}
      \label{thm.term-nst-*.seq-brk}
      $p'=\ceu{\Break}$.
      %%
      Then, by~\R{seq-brk} and by the induction hypothesis, there is
      a~$\delta'$ such that
      \[
        \<\ceu{\Break;\,p''},n,\nil>
        \nst[1]\<p'',n,e>
        \nst[*]\delta'_\Hnst\,.
      \]
    \end{subcase}
    \begin{subcase}
      $p'\ne\ceu{\Nop},\ceu{\Break}$.
      %%
      By induction hypothesis, there are~$p_1'$ and~$e$ such that
      \[
        \<p',n,\nil>\nst[*]\<p_1',n,e>_\Hnst\,.
      \]
      And by item~\eqref{lem.props-nst-n.a} of Lemma~\ref{lem.props-nst-n},
      \begin{equation}
        \label{thm.term-nst-*.seq-adv.eq1}
        \<\ceu{p';\,p''},n,\nil>\nst[*]\<\ceu{p_1';\,p''},n,e>.
      \end{equation}
      It remains to be shown that~$\<\ceu{p_1';\,p''},n,e>$ is indeed
      nested-irreducible.  There are four possibilities following from the
      fact that the simpler~$\<p_1',n,e>$ is nested-irreducible.
      %%
      \begin{subsubcase}
        $e\ne\nil$.  Then, by the definition of~$\Hnst$,
        description~$\<\ceu{p_1';\,p''},n,e>$ is also nested-irreducible.
      \end{subsubcase}
      \begin{subsubcase}
        $p_1'=\ceu{\Nop}$.
        %%
        From~\eqref{thm.term-nst-*.seq-adv.eq1},
        \[
          \<\ceu{p';\,p''},n,\nil>\nst[*]\<\ceu{\Nop;\,p''},n,e>\,.
        \]
        From this point on, this case is similar to
        Case~\ref{thm.term-nst-*.seq-nop}.
      \end{subsubcase}
      \begin{subsubcase}
        $p_1'=\ceu{\Break}$.
        %%
        From~\eqref{thm.term-nst-*.seq-adv.eq1},
        \[
          \<\ceu{p';\,p''},n,\nil>\nst[*]\<\ceu{\Break;\,p''},n,e>\,.
        \]
        From this point on, this case is similar to
        Case~\ref{thm.term-nst-*.seq-brk}.
      \end{subsubcase}
      \begin{subsubcase}
        $\isblocked(p_1',n)$ is true.
        %%
        Then, by definition,
        \[
          \isblocked(\ceu{p_1';p''},n)=\isblocked(p_1',n)=\mathit{true}\,.
        \]
        Thus, from~\eqref{thm.term-nst-*.seq-adv.eq1} and by the
        definition~$\Hnst$, $\<\ceu{p_1';\,p''},n,e>$~is also
        nested-irreducible.
      \end{subsubcase}
    \end{subcase}
  \end{case}

  \begin{case}
    $p=\ceu{\Loop{p'}}$.
    %%
  \end{case}

  \begin{case}
    $p=\ceu{p'\AtLoop{p''}}$.
  \end{case}

  \begin{case}
    $p=\ceu{{p'}\And{p''}}$.
  \end{case}

  \begin{case}
    $p=\ceu{{p'}\AtAnd{p''}}$.
  \end{case}

  \begin{case}
    $p=\ceu{{p'}\Or{p''}}$.
  \end{case}

  \begin{case}
    $p=\ceu{{p'}\AtOr{p''}}$.
  \end{case}
\end{proof}


% LocalWords:  subcase subcases
