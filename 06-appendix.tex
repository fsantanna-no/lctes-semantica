\appendix
\section{Proofs}
\label{sec.proofs}

\makeatletter
\newtheoremstyle{subcase}%
  {\thm@preskip\topsep \divide\thm@preskip\tw@}% space above
  {\thm@postskip\thm@preskip}% space below
  {\addtolength{\@totalleftmargin}{\parindent}%
    \addtolength{\linewidth}{-\parindent}%
    \parshape 1 \parindent \linewidth
    \normalfont}% body font
  {\z@}% indent amount
  {\itshape}% head font
  {.}% punctuation after head
  {.5em}% spacing after head
  {}% head spec
\newtheoremstyle{subsubcase}%
  {\thm@preskip\topsep \divide\thm@preskip\tw@}% space above
  {\thm@postskip\thm@preskip}% space below
  {\addtolength{\@totalleftmargin}{\parindent}%
    \addtolength{\linewidth}{-\parindent}%
    \parshape 1 2\parindent \linewidth
    \normalfont}% body font
  {\z@}% indent amount
  {\itshape}% head font
  {.}% punctuation after head
  {.5em}% spacing after head
  {}% head spec
\makeatletter

\theoremstyle{remark}
\newtheorem{case}{Case}

\theoremstyle{subcase}
\newtheorem{subcase}{Case}
\numberwithin{subcase}{case}

\theoremstyle{subsubcase}
\newtheorem{subsubcase}{Case}
\numberwithin{subsubcase}{subcase}

\AtBeginEnvironment{proof}{\setcounter{case}{0}}
\AtBeginEnvironment{case}{\setcounter{subcase}{0}}
\AtBeginEnvironment{subcase}{\setcounter{subsubcase}{0}}

% \makeatletter
% \newcommand\case{\@startsection{paragraph}{4}{\z@}%
%    {-.5\baselineskip \@plus -2\p@ \@minus -.2\p@}%
%    {-3.5\p@}%
%    {\@parfont}}
%  \def\@parfont{\itshape}
% \makeatother


\begin{lemma}\label{lem.det-out}
  If~$\delta\out\delta_1$ and~$\delta\out\delta_2$ then~$\delta_1=\delta_2$.
\end{lemma}
\begin{proof}
  The lemma is vacuously true if~$\delta$ cannot be advanced by~$\out$
  transitions.  Suppose that is not the case and let~$\delta=\<p,n,e>$,
  $\delta_1=\<p_1,n_1,e_1>$ and~$\delta_2=\<p_2,n_2,e_2>$.  Then, there are
  two possibilities.
  \begin{case}
    $e\ne\nil$.  Both transitions are applications of rule~\R{push}.
    Hence~$p_1=p_2=\bcast(p,e)$, $n_1=n_2=n+1$, and~$e_1=e_2=\nil$.
  \end{case}
  \begin{case}
    $e=\nil$.  Both transitions are applications of rule~\R{pop}.
    Hence~$p_1=p_2=p$, $n_1=n_2=n-1$, and~$e_1=e_2=\nil$.\qedhere
  \end{case}
\end{proof}


\begin{theorem}\label{thm.det-out-pop-n}
  If~$\delta\out[n]\delta_1$ and~$\delta\out[n]\delta_2$
  then~$\delta_1=\delta_2$.
\end{theorem}
\begin{proof}
  By induction on~$n$.
  %%
  The theorem is trivially true for~$n=0$ and follows directly from
  Lemma~\ref{lem.det-out} for~$n=1$.  (Note that, for~$n>1$, the theorem is
  vacuously true for~$\outpush$ transitions.  By the format of
  rules~\R{push} and~\R{pop}, transitions~$\outpush$ cannot be applied more
  than once in a row and cannot occur after a~$\outpop$ transition.)

  Suppose
  \[
    \delta\out[1]\delta_1'\out[n-1]\delta_1
    \quad\text{and}\quad
    \delta\out[1]\delta_2'\out[n-1]\delta_2,
  \]
  for some~$n>1$ and~$\delta_1'$, $\delta_2'\in\Delta$.
  %%
  By Lemma~\ref{lem.det-out}, $\delta_1'=\delta_2'$.  By induction
  hypothesis, $\delta_1=\delta_2$.\qedhere
\end{proof}


\begin{lemma}\label{lem.det-nst}
  If~$\delta\nst\delta_1$ and~$\delta\nst\delta_2$ then~$\delta_1=\delta_2$.
\end{lemma}
\begin{proof}
  By induction on the structure of~$\nst$ derivations.  The lemma is
  vacuously true if~$\delta$ cannot be advanced by~$\nst$ transitions.
  Suppose that is not the case and let~$\delta=\<p,n,e>$,
  $\delta_1=\<p_1,n_1,e_1>$ and~$\delta_2=\<p_2,n_2,e_2>$.  Then, by the
  hypothesis of the lemma, there are derivations~$\pi_1$ and~$\pi_2$ such
  that
  \begin{align*}
    \pi_1&\Vdash\<p,n,e>\nst\<p_1,n_1,e_1>\\
    \pi_2&\Vdash\<p,n,e>\nst\<p_2,n_2,e_2>
  \end{align*}
  i.e., the conclusion of~$\pi_1$ is~$\<p,n,e>\nst\<p_1,n_1,e_1>$ and the
  conclusion of~$\pi_2$ is~$\<p,n,e>\nst\<p_2,n_2,e_2>$.

  By the definition of transition~$\nst$, we have that~$e=\nil$ and
  $n_1=n_2=n$.  It remains to be shown that~$p_1=p_2$ and~$e_1=e_2$.

  Depending on the structure of program~$p$, the following~11 cases are
  possible.  (Note that~$p$ cannot be an~$\ceu{\AwaitExt}$,
  $\ceu{\AwaitInt}$, $\ceu{\Break}$, $\ceu{\Every}$, $\ceu{\Fin}$,
  or~$\ceu{\Nop}$ expression as there are no~$\nst$ rules to transition such
  programs.)

  \begin{case}
    $p=\ceu{\Mem(\Id)}$.
    %%
    Then derivations~$\pi_1$ and~$\pi_2$ are instances of rule~\R{mem},
    i.e., their conclusions are obtained by an application of this rule.
    Hence~$p_1=p_2=\ceu{\Nop}$ and~$e_1=e_2=\nil$.
  \end{case}

  \begin{case}
    $p=\ceu{\EmitInt(e')}$.
    %%
    Then~$\pi_1$ and~$\pi_2$ are instances of~\R{emit-int}.
    Hence~$p_1=p_2=\ceu{\CanRun(n)}$ and~$e_1=e_2=e'$.
  \end{case}

  \begin{case}
    $p=\ceu{\CanRun(n)}$.
    %%
    Then~$\pi_1$ and~$\pi_2$ are instances of~\R{can-run}.
    Hence~$p_1=p_2=\ceu{\Nop}$ and~$e_1=e_2=\nil$.
  \end{case}

  \begin{case}
    $p=\ceu{\Ifelse{\Mem(\Id)}{p'}{p''}}$.  There are two subcases.
    \begin{subcase}
      $\eval(\ceu{\Mem(\Id)})$ is true.
      %%
      Then~$\pi_1$ and~$\pi_2$ are instances of~\R{if-true}.
      Hence~$p_1=p_2=p'$ and~$e_1=e_2=\nil$.
    \end{subcase}
    \begin{subcase}
      $\eval(\ceu{\Mem(\Id)})$ is false.
      %%
      Then~$\pi_1$ and~$\pi_2$ are instances of~\R{if-false}.
      Hence~$p_1=p_2=p''$ and~$e_1=e_2=\nil$.
    \end{subcase}
  \end{case}

  \begin{case}
    $p=\ceu{p';\,p''}$.  There are three subcases.
    \begin{subcase}
      $p'=\ceu{\Nop}$.
      %%
      Then~$\pi_1$ and~$\pi_2$ are instances of~\R{seq-nop}.
      Hence~$p_1=p_2=p''$ and~$e_1=e_2=\nil$.
    \end{subcase}
    \begin{subcase}
      $p'=\ceu{\Break}$.
      %%
      Then~$\pi_1$ and~$\pi_2$ are instances of~\R{seq-brk}.
      Hence~$p_1=p_2=\ceu{\Break}$ and~$e_1=e_2=\nil$.
    \end{subcase}
    \begin{subcase}
      $p'\ne\ceu{\Nop},\ceu{\Break}$.
      %%
      Then~$\pi_1$ and~$\pi_2$ are instances of~\R{seq-adv}.
      Thus there are derivations~$\pi_1'$ and~$\pi_2'$ such that
      \begin{align*}
        \pi_1'&\Vdash\<p',n,\nil>\nst\<p_1',n,e_1'>\\
        \pi_2'&\Vdash\<p',n,\nil>\nst\<p_2',n,e_2'>
      \end{align*}
      for some~$p_1',p_2'\in\P$ and~$e_1',e_2'\in\E$.  By induction
      hypothesis, $p_1'=p_2'$ and~$e_1'=e_2'$.
      Hence~$p_1=\ceu{p_1';p''}=\ceu{p_2';p''}=p_2$ and~$e_1=e_1'=e_2'=e_2$.
    \end{subcase}
  \end{case}

  \begin{case}
    $p=\ceu{\Loop{p'}}$.
    %%
    Then~$\pi_1$ and~$\pi_2$ are instances of~\R{loop-expd}.
    Hence~$p_1=p_2=\ceu{p'\AtLoop{p'}}$ and~$e_1=e_2=\nil$.
  \end{case}

  \begin{case}
    $p=\ceu{p'\AtLoop{p''}}$.  There are three subcases.
    \begin{subcase}
      $p'=\ceu{\Nop}$.
      %%
      Then~$\pi_1$ and~$\pi_2$ are instances of~\R{loop-nop}.
      Hence~$p_1=p_2=\ceu{\Loop{p''}}$ and~$e_1=e_2=\nil$.
    \end{subcase}
    \begin{subcase}
      $p'=\ceu{\Break}$.
      %%
      Then~$\pi_1$ and~$\pi_2$ are instances of~\R{loop-break}.
      Hence~$p_1=p_2=\ceu{\Nop}$ and~$e_1=e_2=\nil$.
    \end{subcase}
    \begin{subcase}
      $p'\ne\ceu{\Nop},\ceu{\Break}$.
      %%
      Then~$\pi_1$ and~$\pi_2$ are instances of~\R{loop-adv}.
      Thus there are derivations~$\pi_1'$ and~$\pi_2'$ such that
      \begin{align*}
        \pi_1'&\Vdash\<p',n,\nil>\nst\<p_1',n,e_1'>\\
        \pi_2'&\Vdash\<p',n,\nil>\nst\<p_2',n,e_2'>
      \end{align*}
      for some~$p_1',p_2'\in\P$ and~$e_1',e_2'\in\E$.  By induction
      hypothesis, $p_1'=p_2'$ and~$e_1'=e_2'$.
      Hence~$p_1=\ceu{p_1'\AtLoop{p''}}=\ceu{p_2'\AtLoop{p''}}=p_2$
      and~$e_1=e_1'=e_2'=e_2$.
    \end{subcase}
  \end{case}

  \begin{case}
    $p=\ceu{p'\And{p''}}$.
    %%
    Then~$\pi_1$ and~$\pi_2$ are instances of~\R{and-expd}.
    Hence~$p_1=p_2=\ceu{{p'}\And{(\CanRun(n);\,p'')}}$ and~$e_1=e_2=\nil$.
  \end{case}

  \begin{case}
    $p=\ceu{p'\AtAnd{p''}}$.  There are two subcases.
    \begin{subcase}
      $\isblocked(p',n)$~is false.  There are three subcases.
      \begin{subsubcase}
        $p'=\ceu{\Nop}$.
        %%
        Then~$\pi_1$ and~$\pi_2$ are instances of~\R{and-nop1}.
        Hence~$p_1=p_2=p''$ and~$e_1=e_2\nil$.
      \end{subsubcase}
      \begin{subsubcase}\label{lem.det-nst.and-brk1}
        $p'=\ceu{\Break}$.
        %%
        Then~$\pi_1$ and~$\pi_2$ are instances of~\R{and-brk1}.
        Hence~$p_1=p_2=\ceu{\clear(p'');\Break}$ and~$e_1=e_2\nil$.
      \end{subsubcase}
      \begin{subsubcase}\label{lem.det-nst.and-adv1}
        $p'\ne\ceu{\Nop},\ceu{\Break}$.
        %%
        Then~$\pi_1$ and~$\pi_2$ are instances of~\R{and-adv1}.
        Thus there are derivations~$\pi_1'$ and~$\pi_2'$ such that
        \begin{align*}
          \pi_1'&\Vdash\<p',n,\nil>\nst\<p_1',n,e_1'>\\
          \pi_2'&\Vdash\<p',n,\nil>\nst\<p_2',n,e_2'>
        \end{align*}
        for some~$p_1',p_2'\in\P$ and~$e_1',e_2'\in\E$.  By induction
        hypothesis, $p_1'=p_2'$ and~$e_1'=e_2'$.
        Hence~$p_1=\ceu{{p_1'}\And{p''}}=\ceu{{p_2'}\And{p''}}=p_2$
        and~$e_1=e_1'=e_2'=e_2$.
      \end{subsubcase}
    \end{subcase}
    \begin{subcase}
      $\isblocked(p',n)$~is true.  There are three subcases.
      \begin{subsubcase}
        $p''=\ceu{\Nop}$.
        %%
        Then~$\pi_1$ and~$\pi_2$ are instances of~\R{and-nop2}.
        Hence~$p_1=p_2=p'$ and~$e_1=e_2\nil$.
      \end{subsubcase}
      \begin{subsubcase}\label{lem.det-nst.and-brk2}
        $p''=\ceu{\Break}$.
        %%
        Then~$\pi_1$ and~$\pi_2$ are instances of~\R{and-brk2}.
        Hence~$p_1=p_2=\ceu{\clear(p');\Break}$ and~$e_1=e_2=\nil$.
      \end{subsubcase}
      \begin{subsubcase}\label{lem.det-nst.and-adv2}
        $p''\ne\ceu{\Nop},\ceu{\Break}$.
        %%
        Then~$\pi_1$ and~$\pi_2$ are instances of~\R{and-adv2}.  Thus there
        are derivations~$\pi_1''$ and~$\pi_2''$ such that
        \begin{align*}
          \pi_1''&\Vdash\<p'',n,\nil>\nst\<p_1'',n,e_1''>\\
          \pi_2''&\Vdash\<p'',n,\nil>\nst\<p_2'',n,e_2''>
        \end{align*}
        for some~$p_1'',p_2''\in\P$ and~$e_1'',e_2''\in\E$.  By induction
        hypothesis, $p_1''=p_2''$ and~$e_1''=e_2''$.
        Hence~$p_1=\ceu{{p'}\And{p_1''}}=\ceu{{p'}\And{p_2''}}=p_2$
        and~$e_1=e_1''=e_2''=e_2$.
      \end{subsubcase}
    \end{subcase}
  \end{case}

  \begin{case}
    $p=\ceu{p'\Or{p''}}$.
    %%
    Then~$\pi_1$ and~$\pi_2$ are instances of~\R{or-expd}.
    Hence~$p_1=p_2=\ceu{{p'}\Or{(\CanRun(n);\,p'')}}$ and~$e_1=e_2=\nil$.
  \end{case}

  \begin{case}
    $p=\ceu{p'\AtOr{p''}}$.  There are two subcases.
    \begin{subcase}
      $\isblocked(p',n)$~is false.  There are three subcases.
      \begin{subsubcase}
        $p'=\ceu{\Nop}$.
        %%
        Then~$\pi_1$ and~$\pi_2$ are instances of~\R{or-nop1}.
        Hence~$p_1=p_2=\clear(p'')$ and~$e_1=e_2\nil$.
      \end{subsubcase}
      \begin{subsubcase}
        $p'=\ceu{\Break}$.
        %%
        Similar to Case~\ref{lem.det-nst.and-brk1}.
      \end{subsubcase}
      \begin{subsubcase}
        $p'\ne\ceu{\Nop},\ceu{\Break}$.
        %%
        Similar to Case~\ref{lem.det-nst.and-adv1}.
      \end{subsubcase}
    \end{subcase}
    \begin{subcase}
      $\isblocked(p',n)$~is true.  There are three subcases.
      \begin{subsubcase}
        $p''=\ceu{\Nop}$.
        %%
        Then~$\pi_1$ and~$\pi_2$ are instances of~\R{or-nop1}.
        Hence~$p_1=p_2=\clear(p')$ and~$e_1=e_2\nil$.
      \end{subsubcase}
      \begin{subsubcase}
        $p''=\ceu{\Break}$.
        %%
        Similar to Case~\ref{lem.det-nst.and-brk2}.
      \end{subsubcase}
      \begin{subsubcase}
        $p''\ne\ceu{\Nop},\ceu{\Break}$.
        %%
        Similar to Case~\ref{lem.det-nst.and-adv2}.
        %%
        \qedhere
      \end{subsubcase}
    \end{subcase}
  \end{case}
\end{proof}


\begin{theorem}\label{thm.det-nst-n}
  If~$\delta\nst[n]\delta_1$ and~$\delta\nst[n]\delta_2$
  then~$\delta_1=\delta_2$.
\end{theorem}
\begin{proof}
  Similar to the proof of Theorem~\ref{thm.det-out-pop-n}.
\end{proof}

% LocalWords:  subcase subcases
