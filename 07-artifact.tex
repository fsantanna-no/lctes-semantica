\appendix
\section{Artifact appendix}

%Submission and reviewing guidelines and methodology:
%{\em http://cTuning.org/ae/submission.html}

%%%%%%%%%%%%%%%%%%%%%%%%%%%%%%%%%%%%%%%%%%%%%%%%%%%%%%%%%%%%%%%%%%%%%
\subsection{Abstract}

Our artifact includes an open-source implementation of the programming language
\CEU.
The implementation is based on and should conform with the formal semantics
presented in this paper.
The artifact also includes an executable script with over 3500 test cases of
valid and invalid programs in \CEU.
The script is customizable and allows to create new tests providing inputs and
expected outputs.
The evaluation platform is a Linux/Intel with Lua-5.3 and GCC-7.2 installed.

\subsection{Artifact check-list (meta-information)}

{\em Obligatory. Use just a few informal keywords in all fields applicable to your artifacts
and remove the rest. This information is needed to find appropriate reviewers and gradually 
unify artifact meta information in Digital Libraries.}

{\small
\begin{itemize}
  %\item {\bf Algorithm: }
  %\item {\bf Program: }
  \item {\bf Compilation: } The output of the \CEU compiler is a C program that requires a C compiler (e.g., GCC-7.2).
  \item {\bf Transformations: } The compiler of \CEU is a Lua program (Lua-5.3) that generates a C program.
  %\item {\bf Binary: }
  \item {\bf Data set: } The data set is a set of program test cases included in the language distribution.
  \item {\bf Run-time environment: } Linux with Lua-5.3 and GCC-7.2. Root access is not required.
  \item {\bf Hardware: } Off-the-shelf Intel machine.
  %\item {\bf Run-time state: }
  \item {\bf Execution: } 5-10 minutes for the full test.
  \item {\bf Output: } Console output: Success (termination) or Fail (abortion).
  \item {\bf Experiments: } Manual steps by user.
  %\item {\bf Workflow frameworks used?: }
  %\item {\bf Publicly available?: } Yes: \url{https://github.com/fsantanna/ceu}
\end{itemize}

\begin{itemize}
  \item {\bf Artifacts publicly available?:} Yes.
  \item {\bf Artifacts functional?:} Yes.
  \item {\bf Artifacts reusable?:} No.
  \item {\bf Results validated?:} No.
\end{itemize}

%%%%%%%%%%%%%%%%%%%%%%%%%%%%%%%%%%%%%%%%%%%%%%%%%%%%%%%%%%%%%%%%%%%%%
\subsection{Description}

\subsubsection{How delivered}

The compiler/distribution of \CEU is available on GitHub:
\url{https://github.com/fsantanna/ceu}

\subsubsection{Hardware dependencies}

Off-the-shelf Intel machine.

\subsubsection{Software dependencies}

Linux, Lua-5.3 (with lpeg-1.0.0), and GCC-7.2.

\subsubsection{Data sets}

A set of more than 3500 programs packed in a single script file which is
included with the compiler distribution.

%%%%%%%%%%%%%%%%%%%%%%%%%%%%%%%%%%%%%%%%%%%%%%%%%%%%%%%%%%%%%%%%%%%%%
\subsection{Installation}

Install all required software (assuming an Ubuntu-based distribution):

\begin{verbatim}
$ sudo apt-get install git lua5.3 lua-lpeg liblua5.3-0 \
                       liblua5.3-dev
\end{verbatim}

Clone the repository of \CEU:

\begin{verbatim}
$ git clone https://github.com/fsantanna/ceu
$ cd ceu/
$ git checkout v0.30
\end{verbatim}

Install \CEU:

\begin{verbatim}
$ make
$ sudo make install     # install as "/usr/local/bin/ceu"
\end{verbatim}

%%%%%%%%%%%%%%%%%%%%%%%%%%%%%%%%%%%%%%%%%%%%%%%%%%%%%%%%%%%%%%%%%%%%%
\subsection{Experiment workflow}

Run the experiment:

\begin{verbatim}
$ cd tst/
$ ./run.lua
\end{verbatim}

%%%%%%%%%%%%%%%%%%%%%%%%%%%%%%%%%%%%%%%%%%%%%%%%%%%%%%%%%%%%%%%%%%%%%
\subsection{Evaluation and expected result}

The experiment will execute all test cases.
In about 5-10 minutes, a summary will be printed on screen:

\begin{verbatim}
$ cd tst/
$ ./run.lua
<...> # output with the test programs
stats = {
    count  = 3483,
    trails = 9094,
    bytes  = 52555128,
    bcasts = 0,
    visits = 4065937,
}
\end{verbatim}

%%%%%%%%%%%%%%%%%%%%%%%%%%%%%%%%%%%%%%%%%%%%%%%%%%%%%%%%%%%%%%%%%%%%%
\subsection{Experiment customization}

The file \code{tst/tests.lua} includes all test cases.
Each test case contains a program in \CEU as well as the expected result, e.g.:

\begin{verbatim}
Test { [[
    var int ret = 0;
    var int i;
    loop i in [0 -> 10[ do
        ret = ret + 1;
    end
    escape ret;
]],
    run = 10,
}
\end{verbatim}

This test case should compile and run successfully yielding \code{10}.

To customize the experiment, include a new test case at line \code{440}, after
the string ``\code{--~OK:~well~tested}''.

%%%%%%%%%%%%%%%%%%%%%%%%%%%%%%%%%%%%%%%%%%%%%%%%%%%%%%%%%%%%%%%%%%%%%
%\subsection{Notes}
