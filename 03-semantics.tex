\newcommand{\NST}{\1\xrightarrow[\mathit{nst}]\1}
\newcommand{\OUT}{\1\xrightarrow[\mathit{out}]\1}
\newcommand{\LL}{\langle}
\newcommand{\RR}{\rangle}
\newcommand{\DS}{\displaystyle}

\newcommand{\1}{\;}
\newcommand{\2}{\;\;}
\newcommand{\3}{\;\;\;}
\newcommand{\5}{\;\;\;\;\;}

\section{Formal Semantics}
\label{sec.sem}

In this section, we introduce a reduced syntax of \CEU and propose an
operational semantics to formally describe the behavior of programs.
We describe a small synchronous kernel highlighting the peculiarities of \CEU,
in particular, the stack-based execution for internal events.
%
For the sake of simplicity, we focus on the control aspects of the language,
leaving out side-effects and system calls (which behave like in conventional
imperative languages).

\subsection{Abstract Syntax}
\label{sec.sem.syntax}

%-
% \begin{lstlisting}[
%   %numbers=left,
%   basicstyle=\ttfamily\footnotesize,
%   float=h,
%   caption={Reduced syntax of \CEU.},
%   label={lst.formal.syntax},
%   mathescape=true
% ]
%                                    // primary expressions
%   p ::= mem(id)                    (any memory access to `id')
%       $|$ awaitExt(id)               (await external event `id')
%       $|$ awaitInt(id)               (await internal event `id')
%       $|$ emitInt(id)                (emit internal event `id')
%       $|$ break                      (loop escape)
%                                    // compound expressions
%       $|$ if mem(id) then p else p   (conditional)
%       $|$ p ; p                      (sequence)
%       $|$ loop p                     (repetition)
%       $|$ every id p                 (event iteration)
%       $|$ p and p                    (par/and)
%       $|$ p or p                     (par/or)
%       $|$ fin p                      (finalization)
%                                    // derived by semantic rules
%       $|$ p @loop p                  (unwinded loop)
%       $|$ p @and q                   (unwinded par/and)
%       $|$ p @or q                    (unwinded par/or)
%       $|$ @canrun(n)                 (can run on stack level `n')
%       $|$ @nop                       (terminated expression)
% \end{lstlisting}
%-

The grammar below defines the syntax of a subset of \CEU that is
sufficient to describe all semantic peculiarities of the language.
\bgroup
\def\lbl#1{\quad\text{\emph{#1}}}%
\newdimen\X
\X=-1\jot
\begin{alignat*}{2}
  p\Coloneqq
      &\enspace\ceu{\Mem(\Id)}
      &&\lbl{any memory access to~``$\ceu\Id$''}\\[\X]
      %%
  \mid&\enspace\ceu{\AwaitExt(\Id)}
      &&\lbl{await external event~``$\ceu\Id$''}\\[\X]
      %%
  \mid&\enspace\ceu{\AwaitInt(\Id)}
      &&\lbl{await internal event~``$\ceu\Id$''}\\[\X]
      %%
  \mid&\enspace\ceu{\EmitInt(\Id)}
      &&\lbl{emit internal event~``$\ceu\Id$''}\\[\X]
      %%
  \mid&\enspace\ceu{\Break}
      &&\lbl{loop escape}\\[\X]
      %%
  \mid&\enspace\ceu{\IfElse{\Mem(\Id)}{p_1}{p_2}}
      &&\lbl{conditional}\\[\X]
      %%
  \mid&\enspace\ceu{p_1\,;\,p_2}
      &&\lbl{sequence}\\[\X]
      %%
  \mid&\enspace\ceu{\Loop p_1}
      &&\lbl{repetition}\\[\X]
      %%
  \mid&\enspace\ceu{\Every{\Id}\ p_1}
      &&\lbl{event iteration}\\[\X]
      %%
  \mid&\enspace\ceu{p_1\And p_2}
      &&\lbl{par/and}\\[\X]
      %%
  \mid&\enspace\ceu{p_2\Or p_2}
      &&\lbl{par/or}\\[\X]
      %%
  \mid&\enspace\ceu{\Fin p}
      &&\lbl{finalization}\\[\X]
      %%
  \mid&\enspace\ceu{p_1\AtLoop p_2}
      &&\lbl{unwinded loop}\\[\X]
      %%
  \mid&\enspace\ceu{p_1\AtAnd\ p_2}
      &&\lbl{unwinded par/and}\\[\X]
      %%
  \mid&\enspace\ceu{p_1\AtOr\ p_2}
      &&\lbl{unwinded par/or}\\[\X]
      %%
  \mid&\enspace\ceu{\CanRun(n)}
      &&\lbl{can run on stack level~$n$}\\[\X]
      %%
  \mid&\enspace\ceu{\Nop}
      &&\lbl{terminated program}
\end{alignat*}
\egroup

The~$\ceu{\Mem(id)}$ primitive represents all accesses, assignments, and system
calls that affect a memory location identified by~$id$.
%
According to the synchronous hypothesis of \CEU, $\ceu{\Mem}$ expressions are
considered to be atomic and instantaneous.
%
As the challenging parts of \CEU reside on its control structures, we are not
concerned here with a precise semantics for side effects, but only with their
occurrences in programs.
%
%The special notation $nop$ is used to represent an innocuous $mem$ expression
%(it can be thought as a synonym for $mem(\epsilon)$, where $\epsilon$ is an
%unused identifier).

We assume that $\ceu{\Mem}$, $\ceu{\AwaitExt}$, $\ceu{\AwaitInt}$
and~$\ceu{\EmitInt}$ expressions do not share identifiers:
any identifier is either a variable, an external event, or an internal event.

Most expressions in the abstract language are mapped to their counterparts in
the concrete language.
The exceptions are the finalization block~$\ceu{\Fin{p}}$ and the
\texttt{@}-expressions which result from expansions in the transition rules to
be presented.

Regarding mismatches between the concrete and abstract languages, the concrete
\code{await} and \code{emit} primitives support communication of values between
them, e.g., an ``\code{emit a(10)}'' awakes a ``\code{v=await a}'' setting
variable~\code{v} to~10.
To reproduce this functionality in the formal semantics, we can use a shared
variable to hold the value of an $\ceu{\EmitInt}$ and access it after the
corresponding $\ceu{\AwaitInt}$ awakes.
%
Also, a ``\code{finalize $A$ with $B$ end; $C$}'' in the concrete language is
equivalent to ``\ceu{A;\;((\Fin{B})\ \Or\ C)}'' in the abstract language.
In the concrete language, $A$ and~$C$ execute in sequence, and
the finalization code~$B$ is implicitly suspended waiting for~$C$
termination.
In the abstract language, ``$\ceu{\Fin B}$'' suspends forever when reached (it is
an awaiting expression that never awakes).
Hence, we need an explicit \code{or} to execute~$C$ in parallel, whose
termination aborts ``$\ceu{\Fin B}$'', which finally causes~$B$ to
execute (by the semantic rules to be discussed).

\subsection{Operational Semantics}

The core of our semantics describes how a program reacts to a single external
input event, i.e., starting from an input event, how the program behaves and
becomes idle again to proceed to a subsequent reaction.
%
We use a set of small-step operational rules, which are designed in such a way
that at most one transition is possible at any time, resulting in deterministic
reactions.
%
The transition rules map a triple with a program~$p$, a stack level~$n$, and an
emitted event~$e$ to a modified triple as follows:
\[
  \<p,n,e>\trans\<p',n',e'>\,,
\]
where~$p,p'\in\P$ are abstract-language programs, $n,n'\in\N$ are non-negative
integers representing the current stack level, and~$e,e'\in\E\cup\{\nil\}$ are
the events emitted before and after the transition (both being possibly the
empty event~$\nil$).

We will refer to the triples on the left-hand and right-hand sides of
symbol~$\to$ as \emph{descriptions} (denoted~$\delta$).  The triple on the
left-hand side of symbol~$\to$ is called the \emph{input description}, and
the triple on its right-hand side is called the \emph{output description}.

%-
% \begin{align*}
% p, p' &\in\P
%     && (program~as~described~in~Listing~\ref{lst.formal.syntax})
% \\
% n, n' &\in\N
%     && (current~stack~level)
% \\
% e, e' &\in\E \cup \{\epsilon\}
%     && (emitted~event,~possibly~none)
% \end{align*}
%-

At the beginning of a reaction to an input event~$id$, the input description is
initialized with stack level~0 ($n=0$) and with the externally emitted event
($e=id$).
%, but \code{emitInt} expressions can increase the stack level.
At the end of a reaction, after an arbitrary but finite number of transitions,
the last output description will block with a (possibly) modified program~$p'$, at stack
level~0, and with no event emitted~($\nil$):
\[
  \<p,0,e>\mathbin{\trans[*]}\<p',0,\nil>\,.
\]

We distinguish between two types of transition rules:
    \emph{outermost transitions} $\out$ and
    \emph{nested transitions} $\nst$\,.

\subsubsection*{Outermost transitions}

The~$\out$ rules \R{push} and \R{pop} are non-recursive definitions which only
apply to the program as a whole, and are the only to manipulate the stack
level:
\begin{align*}
  &\AxiomC{$e\ne\nil$}
  \UnaryInfC{$\<p,n,e>\out\<\bcast(p,e),n+1,\nil>$}
  \DisplayProof
  \Rtag{push}\\[2\jot]
  %%
  &\AxiomC{$n>0$}
  \AxiomC{$\ceu{p=\Nop}\vee\isblocked(p,n)$}
  \BinaryInfC{$\<p,n,\nil>\out\<p,n-1,\nil>$}
  \DisplayProof
  \Rtag{pop}
\end{align*}

%-
% { \setlength{\jot}{20pt}
% \begin{eqnarray*}
% & \frac
%     { \DS e \neq \epsilon }
% %   -----------------------------------------------------------
%     { \DS \LL p,n,e \RR \OUT \LL bcast(p),n+1,\epsilon \RR }
%     & \textbf{(push)}   \\
% %%%
% & \frac
%     { \DS n>0, \2 ((p=@nop) \vee isblocked(n,p)) }
% %   -----------------------------------------------------------
%     { \DS \LL p,n,\epsilon \RR \OUT \LL p,n-1,\epsilon \RR }
%     & \textbf{(pop)}    \\
% %%%
% %& \LL p,0,\epsilon \RR \1\xrightarrow\1 \bot
%     %& \textbf{(end)}    %\\
% \end{eqnarray*}
% }
%-

Rule \R{push} matches whenever there is an emitted event in the input
description,
and instantly broadcasts the event to the program, which means
    (a)~awaking active $\ceu{\AwaitExt}$ or $\ceu{\AwaitInt}$ expressions altogether (see function~$\bcast$ in
        Figure~\ref{fig.bcast}),
    (b)~creating a nested reaction by increasing the stack level, and, at the same time,
    (c)~consuming the event ($e$ becomes~$\nil$).
%
Rule \R{push} is the only rule in the semantics that matches an
emitted event and also immediately consumes it.

Rule \R{pop} only decreases the stack level, not affecting the
program, and only applies if the program is blocked (see function~$\isblocked$ in
Figure~\ref{fig.isblocked}).
This condition ensures that an $\ceu{\EmitInt}$ only resumes after its internal
reaction completes and blocks in the current stack level.

At the beginning of the reaction, an external event is emitted, which
will trigger rule \R{push}, which will immediately raise the stack level
to~1.
At the end of the reaction, the program will block or terminate and
successive applications of
rule~\R{pop} will eventually lead to a description containing this
same program at stack level~0.

\subsubsection*{Nested transitions}

The~$\nst$ rules are recursive definitions with the following general format:
\[
\<p,n,\nil>\nst\<p',n,e>.
\]
%
%-
% \begin{align*}
% \LL p, n,\epsilon \RR &\NST
% \LL p',n,e        \RR
%     & \textbf{(rule-inner)}
% \end{align*}
%-
%
Nested transitions do not affect the stack level and never have an emitted
event as a precondition.  The distinction between~$\out$ and~$\nst$ prevents
rules \R{push} and \R{pop} from matching and, consequently, from
inadvertently modifying the current stack level before the nested reaction
is complete.

A complete reaction consists of a series of transitions:
\begin{align*}
  \<p,0,e_\ext>\outpush\<p_1,1,\nil>
  \Big[\null\nst[*]\null\out\null\Big]\!\!\ast
  \null\nst[*]\null\outpop\<p',0,\nil>\,.
\end{align*}
%
%-
% \begin{align*}
% a) &\5\5
%     \LL p,0,ext \RR
%         \1\xrightarrow[out]{push}\1
%     \LL q,1,\epsilon \RR
% \\
% b) &\5\5 \1[ \1\xrightarrow[in]{*}\1
%     \LL r,i,e \RR
%         \1\xrightarrow[out]\1
%     \LL s,j,\epsilon \RR \1]*
% \\
% c) &\5\5 \1\xrightarrow[in]{*}\1
%     \LL t,k,\epsilon \RR
%         \1\xrightarrow[out]{pop}\1
%     \LL u,0,\epsilon \RR
% \end{align*}
%-
%
First, a~$\outpush$ starts a nested reaction at level~1.
Then, a series of alternations between zero or more~$\nst$ transitions (nested reactions) and a
single~$\out$ transition (stack operation) takes place.
Finally, a last~$\outpop$ transition decrements the
stack level to~0 and terminates the reaction.

The~$\nst$ transition rules for atomic expressions are defined as follows:
\begin{align*}
  \<\ceu{\Mem(\Id)},n,\nil>
  &\nst\<\ceu{\Nop},n,\nil>\Rtag{mem}\\
  %%
  \<\ceu{\EmitInt(\Id)},n,\nil>
  &\nst\<\ceu{\CanRun(n)},n,\ceu{\Id}>\Rtag{emit-int}\\
  %%
  \<\ceu{\CanRun(n)},n,\nil>
  &\nst\<\ceu{\Nop},n,\nil>\Rtag{can-run}
\end{align*}

%-
% { \setlength{\jot}{20pt}
% \begin{align*}
% \LL mem(id), n, \epsilon \RR &\NST
% \LL @nop, n, \epsilon \RR
%     & \textbf{(mem)}        \\
% %%%
% \LL emit(id), n, \epsilon \RR &\NST
% \LL @canrun(n), n, id \RR
%     & \textbf{(emitInt)}    \\
% %%%
% \LL @canrun(n), n, \epsilon \RR &\NST
% \LL @nop, n, \epsilon \RR
%     & \textbf{(canrun)}     \\
% \end{align*}
% }
%-

A $\ceu{\Mem}$ operation becomes a $\ceu{\Nop}$ which indicates the memory
access (rule \R{mem}).
An $\ceu{\EmitInt(id)}$ generates an event $\ceu{\Id}$ and transits to a
$\ceu{\CanRun(n)}$ which can only resume at level~$n$ (rule \R{emit-int}).
Since all~$\nst$ rules can only transit with $e=\nil$, an $\ceu{\EmitInt}$
inevitably causes rule \R{push} to execute at the outer level, creating a new
level~$n+1$ on the stack.
Also, with the new stack level, the resulting $\ceu{\CanRun}(n)$ itself cannot
transit yet (rule~\R{can-run}), providing the desired stack-based semantics for
internal events.

Proceeding to compound expressions, the rules for conditionals and sequences
are straightforward:
%
\begin{gather*}
  \AxiomC{$\eval(\ceu{\Mem(\Id)})$}
  \UnaryInfC{$\<\ceu{\IfElse{\Mem(\Id)}{p}{q}},n,\nil>\nst\<p,n,\nil>$}
  \DisplayProof
  \Rtag{if-true}\\[2\jot]
  %%
  \AxiomC{$\lnot\eval(\ceu{\Mem(\Id)})$}
  \UnaryInfC{$\<\ceu{\IfElse{\Mem(\Id)}{p}{q}},n,\nil>\nst\<q,n,\nil>$}
  \DisplayProof
  \Rtag{if-false}\\[2\jot]
  %%
  \AxiomC{$\<p,n,\nil>\nst\<p',n,e>$}
  \UnaryInfC{$\<\ceu{p\,;\,q},n,\nil>\nst\<\ceu{p';\,q},n,e>$}
  \DisplayProof
  \Rtag{seq-adv}
\end{gather*}
\vskip-\belowdisplayskip
\vskip-\abovedisplayskip
\vskip2\jot
\begin{align*}
  \<\ceu{\Nop;\,q},n,\nil>&\nst\<q,n,\nil>\Rtag{seq-nop}\\[2\jot]
  %%
  \<\ceu{\Break;\,q},n,\nil>&\nst\<\ceu{\Break},n,\nil>\Rtag{seq-brk}
\end{align*}

%-
% { \setlength{\jot}{20pt}
% \begin{eqnarray*}
% & \frac
%     { \DS val(id) \neq 0 }
% %   -----------------------------------------------------------
%     { \DS \LL (if~mem(id)~then~p~else~q),n,\epsilon \RR \NST
%           \LL p, n, \epsilon \RR }
%     & \textbf{(if-true)}       \\
% %%%
% & \frac
%     { \DS val(id,n) = 0 }
% %   -----------------------------------------------------------
%     { \DS \LL (if~mem(id)~then~p~else~q),n,\epsilon \RR \NST
%           \LL q,n,\epsilon \RR }
%     & \textbf{(if-false)}       \\
% %%%
% & \frac
%     { \DS \LL p,n,\epsilon \RR \NST \LL p',n,e \RR }
% %   -----------------------------------------------------------
%     { \DS \LL (p~;~q), n, \epsilon \RR \NST \LL (p'~;~q), n, e \RR }
%     & \textbf{(seq-adv)}      \\
% %%%
% & \LL (@nop~;~q),n,\epsilon \RR \NST  \LL q,n,\epsilon \RR
%     & \textbf{(seq-nop)}      \\
% %%%
% & \LL (break~;~q),n,\epsilon \RR \NST \LL break,n,\epsilon \RR
%     & \textbf{(seq-brk)}
% \end{eqnarray*}
% }
%-

Given that our semantics focuses on control, rules \R{if-true} and
\R{if-false} are the only to query~$\ceu{\Mem}$ expressions.
%
Function~$\eval$ evaluates a given~$\ceu{\Mem}$ expression to a boolean value.
%
%Although the value here is arbitrary, it is unique in a reaction, because a
%given expression can execute only once within it (remember that $loops$ must
%contain $awaits$ which, from rule \textbf{await}, cannot awake in the same
%reaction they are reached).
%For all other rules, we omit these values (e.g., \textbf{seq-nop}).

%As determined for nested rules, compound expressions also can only have
%$\epsilon$ as a precondition and they never modify $n$.
%However, they can still emit an event to nest another reaction.
%For instance, in rule \textbf{seq-adv}, if the sub-expression $p$ emits event
%$e$, the whole composition also emits $e$.
%However, rules \textbf{push} and \textbf{pop} can only match at the outermost
%level.

The rules for loops are analogous to sequences, but use ``\code{@}'' as
separators to properly bind breaks to their enclosing loops:
\begin{align*}
  \<\ceu{\Loop{p}},n,\nil>
  &\nst\<\ceu{p\AtLoop{p}},n,\nil>\Rtag{loop-expd}\\[2\jot]
  %%
  &\hskip-6.35em
  \AxiomC{$\<q,n,\nil>\nst\<q',n,e>$}
  \UnaryInfC{$\<\ceu{q\AtLoop{p}},n,\nil>\nst\<\ceu{q'\AtLoop{p}},n,e>$}
  \DisplayProof
  \Rtag{loop-adv}\\[2\jot]
  %%
  \<\ceu{\Nop\AtLoop{p}},n,\nil>
  &\nst\<\ceu{\Loop{p}},n,\nil>\Rtag{loop-nop}\\[2\jot]
  %%
  \<\ceu{\Break\AtLoop{p}},n,\nil>
  &\nst\<\ceu{\Nop},n,\nil>\Rtag{loop-brk}
\end{align*}

%-
% %
% { \setlength{\jot}{20pt}
% \begin{eqnarray*}
% & \LL (loop~p),n,\epsilon \RR \NST \LL (p~@loop~p), n, \epsilon \RR
%     & \textbf{(loop-expd)}       \\
% %%%
% & \frac
%     { \DS \LL p,n,\epsilon \RR \NST \LL p',n,e \RR }
% % -----------------------------------------------------------
%     { \DS \LL (p~@loop~q),n,\epsilon \RR \NST \LL (p'~@loop~q), n, e \RR }
%     & \textbf{(loop-adv)}    \\
% %%%
% & \LL (@nop~@loop~p), n, \epsilon \RR \NST \LL (loop~p), n, \epsilon \RR
%     & \textbf{(loop-nop)}    \\
% %%%
% & \LL (break~@loop~p), n, \epsilon \RR \NST \LL @nop, n, \epsilon \RR
%     & \textbf{(loop-brk)}
% \end{eqnarray*}
% }
%-

When a program encounters a $\ceu{\Loop}$, it first expands its body in sequence with
itself (rule \R{loop-expd}).
Rules \R{loop-adv} and \R{loop-nop} are similar to rules
\R{seq-adv} and \R{seq-nop}, advancing the loop until a~$\ceu{\Nop}$ is reached.
However, what follows the loop is the loop itself (rule \R{loop-nop}).
Note that if we used ``\code{;}'' as a separator in loops, rules
\R{loop-brk} and \R{seq-brk} would conflict.
%
Rule \R{loop-brk} escapes the enclosing loop, transforming everything into
a~$\ceu{\Nop}$.
%Rule \textbf{loop-brk} escapes the enclosing loop, transforming everything
%into a $clear(p)$.
%We cannot simply transform the loop into a $nop$ because its body may be a
%parallel composition containing finalization blocks.

The semantic rules for~$\ceu{\And}$ and~$\ceu{\Or}$ compositions
force transitions on their left branches~$p$ to occur before their right
branches~$q$:
\begin{gather*}
  \hskip-.9em
  \<\ceu{p\And{q}},n,\nil>
  \nst\<\ceu{p\AtAnd(\CanRun(n);q)},n,\nil>
  \Rtag{and-expd}\\[2\jot]
  %%
  \AxiomC{$\<p,n,\nil>\nst\<p',n,e>$}
  \UnaryInfC{$\<\ceu{p\AtAnd{q}},n,\nil>\nst\<\ceu{p'\AtAnd{q}},n,e>$}
  \DisplayProof
  \Rtag{and-adv1}\\[2\jot]
  %%
  \AxiomC{$\isblocked(p,n)$}
  \AxiomC{$\<q,n,\nil>\nst\<q',n,e>$}
  \BinaryInfC{$\<\ceu{p\AtAnd{q}},n,\nil>\nst\<\ceu{p\AtAnd{q'}},n,e>$}
  \DisplayProof
  \Rtag{and-adv2}\\[2\jot]
  %%
  \<\ceu{p\Or{q}},n,\nil>
  \nst\<\ceu{p\AtOr(\CanRun(n);q)},n,\nil>
  \Rtag{or-expd}\\[2\jot]
  %%
  \AxiomC{$\<p,n,\nil>\nst\<p',n,e>$}
  \UnaryInfC{$\<\ceu{p\AtOr{q}},n,\nil>\nst\<\ceu{p'\AtOr{q}},n,e>$}
  \DisplayProof
  \Rtag{or-adv1}\\[2\jot]
  %%
  \AxiomC{$\isblocked(p,n)$}
  \AxiomC{$\<q,n,\nil>\nst\<q',n,e>$}
  \BinaryInfC{$\<\ceu{p\AtOr{q}},n,\nil>\nst\<\ceu{p\AtOr{q'}},n,e>$}
  \DisplayProof
  \Rtag{or-adv2}
\end{gather*}

%-
% { \setlength{\jot}{20pt}
% \begin{eqnarray*}
% & \LL (p~and~q),n,\epsilon \RR \NST \LL (p~@and~(@canrun(n)~;~q)),n,\epsilon \RR
%     & \textbf{(and-expd)}       \\
% %%%
% & \frac
%     { \DS \LL p,n,\epsilon \RR \NST \LL p',n,e \RR }
% %   -----------------------------------------------------------
%     { \DS \LL (p~@and~q),n,\epsilon \NST \LL (p'~@and~q),n,e \RR }
%     & \textbf{(and-adv1)}      \\
% %%%
% & \frac
%     { \DS isblocked(n,p) \1,\2 \LL q,n,\epsilon \RR \NST \LL q',n,e \RR }
% %   -----------------------------------------------------------
%     { \DS \LL (p~@and~q),n,\epsilon \RR \NST \LL (p~@and~q'), n, e \RR }
%     & \textbf{(and-adv2)}      \\
% %%%
% & \LL (p~or~q), n, \epsilon \RR \NST \LL (p~@or~(@canrun(n)~;~q)), n, \epsilon \RR
%     & \textbf{(or-expd)}       \\
% %%%
% & \frac
%     { \DS \LL p,n,\epsilon \RR \NST \LL p',n,e \RR }
% %   -----------------------------------------------------------
%     { \DS \LL (p~@or~q),n,\epsilon \RR \NST \LL (p'~@or~q), n, e \RR }
%     & \textbf{(or-adv1)}   \\
% %%%
% & \frac
%     { \DS isblocked(n,p) \1,\2 \LL q,n,\epsilon \RR \NST \LL q',n,e \RR }
% %   -----------------------------------------------------------
%     { \DS \LL (p~@or~q),n,\epsilon \RR \NST \LL (p~@or~q'), n, e \RR }
%     & \textbf{(or-adv2)}   %\\
% \end{eqnarray*}
% }
%-

Rules~\R{and-expd} and~\R{or-expd} insert a~$\ceu{\CanRun(n)}$ at the beginning
of the right branch.
This ensures that~any $\ceu{\EmitInt}$ on the left branch, which transits to
a~$\ceu{\CanRun(n)}$, still resumes before the right branch starts.
%
The deterministic behavior of the semantics relies on the \emph{isblocked}
predicate (see Figure~\ref{fig.isblocked}) which appears in rules
\R{and-adv2} and \R{or-adv2}.
These rules require the left branch~$p$ to be blocked for the
right branch to transition from~$q$ to~$q'$.

In a parallel~$\ceu{\AtAnd}$, if one of the sides terminates, the composition is
simply substituted by the other side (rules \R{and-nop1} and
\R{and-nop2} below).
%
In a parallel~$\ceu{\AtOr}$, however, if one of the sides terminates, the whole composition
terminates and function~$\clear$ is used to properly finalize the aborted
side (rules \R{or-nop1} and \R{or-nop2}).
\begin{gather*}
  \<\ceu{{\Nop}\AtAnd{q}},n,\nil>\nst\<q,n,\nil>\Rtag{and-nop1}\\[2\jot]
  %%
  \AxiomC{$\isblocked(p,n)$}
  \UnaryInfC{$\<\ceu{p\AtAnd{\Nop}},n,\nil>\nst\<p,n,\nil>$}
  \DisplayProof
  \Rtag{and-nop2}\\[2\jot]
  %%
  \<\ceu{{\Nop}\AtOr{q}},n,\nil>\nst\<\clear(q),n,\nil>\Rtag{or-nop1}\\[2\jot]
  %%
  \AxiomC{$\isblocked(p,n)$}
  \UnaryInfC{$\<\ceu{p\AtOr{\Nop}},n,\nil>\nst\<\clear(p),n,\nil>$}
  \DisplayProof
  \Rtag{or-nop2}
\end{gather*}

%-
% { \setlength{\jot}{20pt}
% \begin{eqnarray*}
% & \LL (@nop~@and~q), n, \epsilon \RR \NST \LL q,n,\epsilon \RR
%     & \textbf{(and-nop1)}   \\
% %%%
% & \frac
%     { \DS isblocked(n,p) }
% %   -----------------------------------------------------------
%     { \DS \LL (p~@and~@nop), n, \epsilon \RR \NST \LL p,n,\epsilon \RR }
%     & \textbf{(and-nop2)}   \\
% %%%
% & \LL (@nop~@or~q), n, \epsilon \RR \NST \LL clear(q),n,\epsilon \RR
%     & \textbf{(or-nop1)}   \\
% %%%
% & \frac
%     { \DS isblocked(n,p) }
% %   -----------------------------------------------------------
%     { \DS \LL (p~@or~@nop), n, \epsilon \RR \NST \LL clear(p),n,\epsilon \RR }
%     & \textbf{(or-nop2)}   %\\
% \end{eqnarray*}
% }
%-

The~$\clear$ function (see Figure~\ref{fig.clear}) concatenates all
active~$\ceu{\Fin}$ bodies of the side being aborted, so that they execute before the
composition rejoins.
Note that there are no transition rules for~$\ceu{\Fin}$ expressions.
This is because once reached, a $\ceu{\Fin}$ expression halts and will only execute
when it is aborted by a parallel trail and is expanded by the~$\clear$
function.
%In Section~\ref{sec.formal.fins}, we show how to map a finalization block in
%the concrete language to a $fin$ in the formal semantics.
%
Note also that there is a syntactic restriction that postulates that~$\ceu{\Fin}$ bodies cannot
contain awaiting expressions ($\ceu{\AwaitExt}$, $\ceu{\AwaitInt}$,
$\ceu{\Every}$, or $\ceu{\Fin}$),
i.e., the result of a~$\clear$ call is guaranteed to execute entirely within a reaction.

Finally, a~$\ceu{\Break}$ in one of the sides in parallel escapes the closest
enclosing~$\ceu{\Loop}$, properly aborting the other side with the~$\clear$
function:
\begin{gather*}
  \hskip-.5em
  \<\ceu{{\Break}\AtAnd{q}},n,\nil>\nst\<\ceu{\clear(q);\Break},n,\nil>
  \Rtag{and-brk1}\\[2\jot]
  %%
  \hskip-.5em
  \AxiomC{$\isblocked(p,n)$}
  \UnaryInfC{$\<\ceu{p\AtAnd{\Break}},n,\nil>
    \nst\<\ceu{\clear(p);\Break},n,\nil>$}
  \DisplayProof
  \Rtag{and-brk2}\\[2\jot]
  %%
  \<\ceu{{\Break}\AtOr{q}},n,\nil>\nst\<\ceu{\clear(q);\Break},n,\nil>
  \Rtag{or-brk1}\\[2\jot]
  %%
  \AxiomC{$\isblocked(p,n)$}
  \UnaryInfC{$\<\ceu{p\AtOr{\Break}},n,\nil>
    \nst\<\ceu{\clear(p);\Break},n,\nil>$}
  \DisplayProof
  \Rtag{or-brk2}
\end{gather*}

%-
% { \setlength{\jot}{20pt}
% \begin{eqnarray*}
% & \LL (break~@and~q), n, \epsilon \RR \NST \LL (clear(q)~;~break),n,\epsilon \RR
%     & \textbf{(and-brk1)}   \\
% %%%
% & \frac
%     { \DS isblocked(n,p) }
% %   -----------------------------------------------------------
%     { \DS \LL (p~@and~break), n, \epsilon \RR \NST \LL (clear(p)~;~break),n,\epsilon \RR }
%     & \textbf{(and-brk2)}   \\
% %%%
% & \LL (break~@or~q),n,\epsilon \RR \NST \LL (clear(q)~;~break),n,\epsilon \RR
%     & \textbf{(or-brk1)}   \\
% %%%
% & \frac
%     { \DS isblocked(n,p) }
% %   -----------------------------------------------------------
%     { \DS \LL (p~@or~break),n,\epsilon \RR \NST \LL (clear(p)~;~break),n,\epsilon \RR }
%     & \textbf{(or-brk2)}   %\\
% \end{eqnarray*}
% }
%-

A reaction eventually blocks in~$\ceu{\AwaitExt}$, $\ceu{\AwaitInt}$,
$\ceu{\Every}$, $\ceu{\Fin}$, and~$\ceu{\CanRun}$ expressions in parallel
trails.
%
Then, if none of the trails is blocked in~$\ceu{\CanRun}$ expressions, it means
that the program cannot advance in the current reaction.
%
However, $\ceu{\CanRun}$ expressions can still resume at lower stack indexes
and will eventually resume in the current reaction (see rule \R{pop}).

\begin{figure}[h]
\small
\begin{gather*}
  \boxed{
    \begin{align*}
      %%
      %%-
      \intertext{\llap{(i)~}Function~$\bcast$:}
      %%-
      %%
      \bcast(\ceu{\AwaitExt(e)},e)
      &=\ceu{\Nop}\\[-1\jot]
      %%
      \bcast(\ceu{\AwaitInt(e)},e)
      &=\ceu{\Nop}\\[-1\jot]
      %%
      \bcast(\ceu{\Every{e}\ {p}},e)
      &=\ceu{p;\,\Every{e}\ {p}}\\[-1\jot]
      %%
      \bcast(\ceu{\CanRun(n)},e)
      &=\ceu{\CanRun(n)}\\[-1\jot]
      %%
      \bcast(\ceu{\Fin{p}},e)
      &=\ceu{\Fin{p}}\\[-1\jot]
      %%
      \bcast(\ceu{p;\,q})
      &=\ceu{\bcast(p,e);\,q}\\[-1\jot]
      %%
      \bcast(\ceu{p\AtLoop{q}},e)
      &=\ceu{\bcast(p,e)\AtLoop{q}}\\[-1\jot]
      %%
      \bcast(\ceu{p\AtAnd{q}},e)
      &=\ceu{{\bcast(p,e)}\AtAnd{\bcast(q,e)}}\\[-1\jot]
      %%
      \bcast(\ceu{p\AtOr{q}},e)
      &=\ceu{{\bcast(p,e)}\AtOr{\bcast(q,e)}}\\[-1\jot]
      %%
      bcast(\_,e)
      &=\_\enspace
        (\ceu{\Mem},\ceu{\EmitInt},\ceu{\Break},\\[-1\jot]
      &\quad\ceu{\IfElse{}{}},\ceu{\Loop},\ceu{\And},\ceu{\Or},\ceu{\Nop})
      \\[1\jot]
      %%
      %%-
      \intertext{\llap{(ii)~}Predicate~$\isblocked$:}
      %%-
      %%
      \isblocked(\ceu{\AwaitExt(e)},n)
      &=\mathit{true}\\[-1\jot]
      %%
      \isblocked(\ceu{\AwaitInt(e)},n)
      &=\mathit{true}\\[-1\jot]
      %%
      \isblocked(\ceu{\Every{e}\ {p}},n)
      &=\mathit{true}\\[-1\jot]
      %%
      \isblocked(\ceu{\CanRun(m)},n)
      &=(n>m)\\[-1\jot]
      %%
      \isblocked(\ceu{\Fin{p}},n)
      &=\mathit{true}\\[-1\jot]
      %%
      \isblocked(\ceu{p;\,q},n)
      &=\isblocked(p,n)\\[-1\jot]
      %%
      \isblocked(\ceu{p\AtLoop{q}},n)
      &=\isblocked(p,n)\\[-1\jot]
      %%
      \isblocked(\ceu{p\AtAnd{q}},n)
      &=\isblocked(p,n)\land\isblocked(q,n)\\[-1\jot]
      %%
      \isblocked(\ceu{p\AtOr{q}},n)
      &=\isblocked(p,n)\land\isblocked(q,n)\\[-1\jot]
      %%
      \isblocked(\_,n)
      &=\mathit{false}\enspace
        (\ceu{\Mem},\ceu{\EmitInt},\ceu{\Break},\\[-1\jot]
      &\quad\ceu{\IfElse{}{}},\ceu{\Loop},\ceu{\And},\ceu{\Or},\ceu{\Nop})
        \\[1\jot]
      %%
      %%-
      \intertext{\llap{(iii)~}Function~$\clear$:}
      %%-
      %%
      \clear(\ceu{\AwaitExt(e)})
      &=\ceu{\Nop}\\[-1\jot]
      %%
      \clear(\ceu{\AwaitInt(e)})
      &=\ceu{\Nop}\\[-1\jot]
      %%
      \clear(\ceu{\Every{e}\ p})
      %%
      &=\ceu{\Nop}\\[-1\jot]
      %%
      \clear(\ceu{\CanRun(n)})
      &=\ceu{\Nop}\\[-1\jot]
      %%
      \clear(\ceu{\Fin{p}})
      &=p\\[-1\jot]
      %%
      \clear(\ceu{p;\,q})
      &=\clear(p)\\[-1\jot]
      %%
      \clear(\ceu{p\AtLoop{q}})
      &=\clear(p)\\[-1\jot]
      %%
      \clear(\ceu{p\AtAnd{q}})
      &=\ceu{\clear(p);\,\clear(q)}\\[-1\jot]
      %%
      \clear(\ceu{p\AtOr{q}})
      &=\ceu{\clear(p);\,\clear(q)}\\[-1\jot]
      %%
      \clear(\_)
      &=\bot\enspace
        (\ceu{\Mem},\ceu{\EmitInt},\ceu{\Break},\\[-1\jot]
      &\quad\ceu{\IfElse{}{}},\ceu{\Loop},\ceu{\And},\ceu{\Or},\ceu{\Nop})
    \end{align*}}
\end{gather*}
\vskip-2\belowdisplayskip
\caption{%
  (i)~Function~$\bcast$ awakes awaiting trails matching the event by
  converting~$\ceu{\protect\AwaitExt}$ and~$\ceu{\protect\AwaitInt}$
  to~$\ceu{\protect\Nop}$ expressions, and by unwinding $\ceu{\protect\Every}$
  expressions.
  %%
  \enspace(ii)~Predicate~$\isblocked$ is true only if all branches in parallel
  are blocked waiting for events, for finalization clauses, or for certain
  stack levels.
  %%
  \enspace(iii)~Function~$\clear$ extracts~$\ceu{\protect\Fin}$ expressions in
  parallel and put their bodies in sequence.
}
\label{fig.bcast}
\label{fig.isblocked}
\label{fig.clear}
\end{figure}

%-
% {\small
% \begin{align*}
%   bcast(e, awaitExt(e)) &= @nop                         \\
%   bcast(e, awaitInt(e)) &= @nop                         \\
%   bcast(e, every~e~p)   &= p;~every~e~p                 \\
%   bcast(e, @canrun(n))  &= @canrun(n)                   \\
%   bcast(e, fin~p)       &= fin~p                        \\
%   bcast(e, p~;~q)       &= bcast(e,p)~;~q               \\
%   bcast(e, p~@loop~q)   &= bcast(e,p)~@loop~q           \\
%   bcast(e, p~@and~q)    &= bcast(e,p)~@and~bcast(e,q)   \\
%   bcast(e, p~@or~q)     &= bcast(e,p)~@or~bcast(e,q)    \\
%   bcast(e, \_)          &= \bot \2 (mem,emitInt,break,if,  \\
%                                  & \5\5 loop,and,or,@nop) %\\
% \end{align*}
% }
%-

%
%-
% {\small
% \begin{align*}
%   isblocked(n, \1 awaitExt(id)) &= true                                   \\
%   isblocked(n, \1 awaitInt(id)) &= true                                   \\
%   isblocked(n, \1 every~e~p)    &= true                                   \\
%   isblocked(n, \1 @canrun(m))   &= (n > m)                                \\
%   isblocked(n, \1 fin~p)        &= true                                   \\
%   isblocked(n, \1 p~;~q)        &= isblocked(n,p)                         \\
%   isblocked(n, \1 p~@loop~q)    &= isblocked(n,p)                         \\
%   isblocked(n, \1 p~@and~q)     &= isblocked(n,p) \wedge isblocked(n,q)   \\
%   isblocked(n, \1 p~@or~q)      &= isblocked(n,p) \wedge isblocked(n,q)   \\
%   isblocked(n, \1 \_)           &= false \2 (mem,emitInt,break,if,        \\
%                                 & \5\5\5\1 loop,and,or,@nop)   %\\
% \end{align*}
% }
%-

%-
% {\small
% \begin{align*}
%   clear( awaitExt(e) ) &= @nop                  \\
%   clear( awaitInt(e) ) &= @nop                  \\
%   clear( every~e~p )   &= @nop                  \\
%   clear( @canrun(n) )  &= @nop                  \\
%   clear( fin~p )       &= p                     \\
%   clear( p~;~q )       &= clear(p)              \\
%   clear( p~@loop~q )   &= clear(p)              \\
%   clear( p~@and~q )    &= clear(p)~;~clear(q)   \\
%   clear( p~@or~q )     &= clear(p)~;~clear(q)   \\
%   clear( \_ )          &= \bot \2 (mem,emitInt,break,if, \\
%                                   & \5\5 loop,and,or,@nop) %\\
% \end{align*}
% }
%-


\subsection{Properties}


\subsubsection{Determinism}

Transitions~$\out$ and~$\nst$ are defined in such a way that given an input
description either no rule is applicable or exactly one of them can be
applied.  This means that the resulting relation~$\trans$ is in fact a
partial function.

The next two lemmas establish the determinism of a single application
of~$\out$ and~$\nst$.  Lemma~\ref{lem.x.det-out} follows from a simple
inspection of rules~\R{push} and~\R{pop}.  The proof of
Lemma~\ref{lem.x.det-out}, however, requires an induction on the structure
of the derivation trees produced by the rules for~$\nst$.  Both lemmas are
used in the proof of Theorem~\ref{thm.x.det}, the main result of this
section.  Theorem~\ref{thm.x.det} establishes that any given number of
applications of~$\trans$ starting from the same input description will
always lead to the same output description.

\begin{lemma}
  \label{lem.x.det-out}
  %%
  If~$\delta\out\delta_1$ and~$\delta\out\delta_2$ then~$\delta_1=\delta_2$.
\end{lemma}

\begin{lemma}
  \label{lem.x.det-nst}
  %%
  If~$\delta\nst\delta_1$ and~$\delta\nst\delta_2$ then~$\delta_1=\delta_2$.
\end{lemma}

\begin{theorem}[Determinism]
  \label{thm.x.det}\strut\\
  %%
  If~$\delta\trans[i]\delta_1$ and~$\delta\trans[i]\delta_2$
  then~$\delta_1=\delta_2$.
\end{theorem}
\begin{proof}
  By induction on~$i$.  The theorem is trivially true if~$i=0$ and follows
  directly from the lemmas if~$i=1$.  Suppose
  \[
    \delta\trans[1]\delta_1'\trans[i-1]\delta_1
    \quad\text{and}\quad
    \delta\trans[1]\delta_2'\trans[i-1]\delta_2\,,
  \]
  for some~$i>1$, $\delta_1'$ and~$\delta_2'$.
  %%
  Then, by Lemma~\ref{lem.det-out} or~\ref{lem.det-nst}, depending on
  whether the first transition is~$\out$ or~$\nst$ (it cannot be both),
  $\delta_1'=\delta_2'$, and by the induction hypothesis,
  $\delta_1=\delta_2$.
\end{proof}

% The proof for determinism relies on the fact all semantic rules are
% mutually exclusive, i.e., their preconditions are unique in the set of
% rules.  This can be verified by direct inspection of rules.

% Rule \textbf{push} is the only one with $e \neq \epsilon$ as a
% precondition, and is trivially mutually exclusive with all other rules.

% Rule \textbf{pop} either has $p=@nop$ or $isblocked(p,n)$ as
% preconditions.
% %
% Note that rule \textbf{pop} only applies syntactically to top-level
% transitions.  For instance, it can never match $\NST$ rules for
% subprograms as in rule \textbf{seq-adv}.
% %
% Hence, for the first case, rule \textbf{pop} only applies, and is the only
% one to apply, to $nop$ as the whole program (i.e., a $nop$ not surrounded
% by other expressions, such as in rule \textbf{seq-nop}).
% %
% For the second case, we need to show that given $\LL p,n,\epsilon \RR$, no
% $\NST$ transitions apply with $isblocked(p,n)$ and vice versa.  Except for
% $@canrun$, there are no $\NST$ transitions for the other blocking
% expressions ($awaitExt$, $awaitInt$, $every$, and $fin$).  However,
% considering the precondition $\LL p,n,\epsilon \RR$,
% $isblocked(@canrun(n),n)$ is false.  Hence, given the preconditions for
% rule \textbf{pop}, no $\NST$ transitions can occur.  Conversely, if a
% $\NST$ transition is possible, then $isblocked(p,n)$ must be false.
% Again, except for $@canrun$, all other transitions do not involve blocking
% expressions, hence, for these transitions, $isblocked(p,n)$ must be false.
% For rule \R{can-run}, a transition can only occur if the current stack
% level matches $@canrun(n)$.  In this case, $isblocked(@canrun(n),n)$ is
% false.

% Finally, we need to show that $\NST$ transitions are mutually exclusive
% among themselves.
% %
% Note that most rules have unique syntactic prefixes, e.g., $(@nop~@and~q)$
% (rule \textbf{and-nop1}) is trivially mutually exclusive with
% $(@nop~@loop~p)$ (rule \textbf{or-nop1}).
% %
% The only exceptions are rules \textbf{and-adv1} vs. \textbf{and-adv2}, and
% \textbf{or-adv1} vs. \textbf{or-adv2}.  In both cases, we need to show
% that if the left branch can advance, then it cannot be blocked and
% vice-versa, i.e., that $\LL p,n,\epsilon \RR \NST \LL p',n,e \RR$ and
% $isblocked(p,n)$ are mutually exclusive, which is exactly the same
% reasoning for rule \textbf{pop} above.


\subsubsection{Termination}

In this section, we prove that a sufficiently long sequence of applications
of~$\trans$, and consequently~$\out$ and~$\nst$, will eventually lead to an
irreducible description, viz., one that cannot be modified by further
transitions.

\begin{definition}
  \label{def.x.Hnst}
  %%
  A description~$\delta=\<p,n,e>$ is \emph{nested-irre\-ducible}
  iff~$e\ne\nil$ or~$p=\ceu{\Nop},\ceu{\Break}$ or~$\isblocked(p,n)$~is
  true.
\end{definition}

Nested-irreducible descriptions serve as normal forms for~$\nst$
transitions: they embody the result of an exhaustive number of~$\nst$
applications.  We will write~$\delta_\Hnst$ to indicate that
description~$\delta$ is nested-irreducible.

To characterize irreducible descriptions in general, we will need to define
the notions of potency of a program and the rank of a description.

\begin{definition}
  \label{def.x.pot}
  %%
  The potency of a program~$p$ in reaction to event~$e$,
  denoted~$\pot(p,e)$, is the maximum number of~$\ceu{\EmitInt}$ expressions
  that can be executed during a reaction of~$p$ to~$e$.

  More formally,
  \[
    \pot(p,e)=\pot'(\bcast(p,e))\,,
  \]
  where~$\pot'$ is an auxiliary function that counts the number of
  reachable~$\ceu{\EmitInt}$ expressions in the program resulting from the
  broadcast of event~$e$ to~$p$.

  The auxiliary function~$\pot'$ is defined by the following clauses:
  \begin{enumerate}[(a)]
  \item\label{def.pot.first}$\pot'(\ceu{\EmitInt}(e))=1$;
    %%
  \item$\pot'(\ceu{\IfElse{\Mem(\Id)}{p_1}{p_2}})
    =\max\{\pot'(p_1),\pot'(p_2)\}$;
    %%
  \item$\pot'(\ceu{\Loop{p_1}})=\pot'(p_1)$;
    %%
  \item$\pot'(\ceu{{p_1}\And{p_2}})=\pot'(p_1)+\pot'(p_2)$;
    %%
  \item$\pot'(\ceu{{p_1}\Or{p_2}})=\pot'(p_1)+\pot'(p_2)$;
    %%
  \item If~$p_1\ne\ceu{\Break},\ceu{\AwaitExt(e)}$,
    \begin{align*}
      \pot'(\ceu{p_1;\,p_2})&=\pot'(p_1)+\pot'(p_2)\\
      \pot'(\ceu{p_1\AtLoop{p_2}})&=
      \begin{cases}
        \pot'(p_1)              &\text{if\enspace(\dag)}\\
        \pot'(p_1)+\pot'(p_2)   &\text{otherwise}\\
      \end{cases}
    \end{align*}
    where~(\dag) stands for: ``a~$\ceu{\Break}$ or~$\ceu{\AwaitExt}$ occurs
    in all execution paths of~$p_1$'';
    %%
  \item If~$p_1,p_2\ne\ceu{\Break}$,
    \[
      \pot'(\ceu{{p_1}\AtAnd{p_2}})=\pot'(p_1)+\pot'(p_2)\,;
    \]
    %%
  \item\label{def.pot.last} If~$p_1,p_2\ne\ceu{\Break}$
    and~$p_1,p_2\ne\ceu{\Nop}$,
    \[
      \pot'(\ceu{{p_1}\AtOr{p_2}})=\pot'(p_1)+\pot'(p_2)\,;
    \]
    %%
  \item Otherwise, if none of~\eqref{def.pot.first}--\eqref{def.pot.last}
    applies, $\pot(\_)=0$.
  \end{enumerate}
\end{definition}

\begin{definition}
  \label{def.x.rank}
  %%
  The \emph{rank} of a description~$\delta=\<p,n,e>$,
  denoted~$\rank(\delta)$, is a pair of nonnegative integers~$\<i,j>$ such
  that
  \begin{alignat*}{2}
    i&=\pot(p,e) &\quad\text{and}\quad
    j&=
       \begin{cases}
         n  &\text{if~$e=\nil$}\\
         n+1&\text{otherwise\,.}
       \end{cases}
  \end{alignat*}
\end{definition}

\begin{definition}
  \label{def.x.H}
  %%
  A description~$\delta$ is \emph{irreducible} (in symbols, $\delta_\#$) iff
  it is nested-irreducible and its~$\rank(\delta)$ is~$\<i,0>$, for
  some~$i\ge0$.
\end{definition}

An irreducible description~$\delta_\#=\<p,n,e>$ serves as a normal form for
transitions~$\trans$ in general.  Such descriptions cannot be advanced
by~$\nst$, as it is nested-irreducible, and neither by~$\outpush$
nor~$\outpop$, as the second coordinate of its rank is~0, which
implies~$e=\nil$ and~$n=0$.

TODO: Lemas e teorema da terminação.  Sketch da prova.  Mencionar que as
restrições sintáticas são fundamentais.

% - there is always a possible transition until n=0
%
% every cannot restart itself
%     - break disallowed
%     - emit ignored

\subsubsection*{Memory Bounds}

TODO: Acho que sai direto do~$\pot(p)$.

- program is finite
- lexical scope
    - no heap allocation
- no code reentrancy
    - reexecution only due to loops
    - loop reuse nested vars
-
