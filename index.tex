\documentclass[sigplan,protrusion=true,expansion,screen]{acmart}

%\setlength{\belowcaptionskip}{-10pt}

\usepackage[utf8]{inputenc}
\usepackage[T1]{fontenc}
\usepackage{mathtools}
\usepackage{enumerate}
\usepackage{etoolbox}

% Céu.
\newcommand{\CEU}{\textsc{C\'{e}u}\xspace}

% Notes
\usepackage[textsize=footnotesize]{todonotes}
\newcommand{\rc}[1]{\todo[inline,color=green!40]{\textbf{Rodrigo}: #1}}
\newcommand{\fs}[1]{\todo[inline,color=blue!40]{\textbf{Francisco}: #1}}
\newcommand{\gl}[1]{\todo[inline,color=red!40]{\textbf{Guilherme}: #1}}

% Concrete syntax.
\usepackage{listings}
\lstdefinelanguage{ceu}{%
  language=C,
  morekeywords={%
    @const, @pure, @safe, CHAN, FOREVER, PAR, PROC, SIGNAL, abort, and,
    await, bool, call, class, data, define, deterministic, do, each, else,
    emit, end, escape, event, every, finalize, hor, implementation, in,
    input, interface, loop, min, native, new, nohold, not, or, output, par,
    pool, pure, return, s, signal, spawn, tag, then, this, traverse, until,
    var, watching, when, with},
}
\lstset{
  basicstyle=\ttfamily,
  captionpos=b,
  columns=flexible,
  commentstyle=\rmfamily\itshape,
  escapeinside={||},
  frame=tb,
  keepspaces=true,
  keywordstyle=\bfseries,
  language=ceu,
  mathescape=true,
  numbersep=4pt,
  numberstyle=\scriptsize,
  upquote=true,
}
\newcommand{\code}[1]{{\small{\texttt{#1}}}}
%\let\code\lstinline
\usepackage{xspace}
\newcommand{\ax}{\code{[a]}\xspace}
\newcommand{\bx}{\code{[b]}\xspace}
\def\<#1>{\langle#1\rangle}
\let\nil\varepsilon

% Syntactic sets for the abstract syntax.
\def\P{P}                       % programs
\def\N{N}                       % numbers
\def\E{E}                       % events

% Abstract syntax.
\makeatletter
\def\@int{\mathit{int}}
\def\@ext{\mathit{ext}}
\let\ext\@ext
\def\@ceuop#1{\mathop{\normalfont{\texttt{#1}}}}%
\def\@ceubin#1{\mathbin{\normalfont{\texttt{#1}}}}%
\def\ceu{\protect\@ceu}
\def\@ceu#1{%
  \bgroup
  \def\Id{\mathit{id}}%
  \def\Mem{\@ceuop{mem}}%
  \def\AwaitExt{\@ceuop{await}\nolimits_\@ext}%
  \def\AwaitInt{\@ceuop{await}\nolimits_\@int}%
  \def\EmitInt{\@ceuop{emit}\nolimits_\@int}%
  \def\Break{\@ceuop{break}}%
  \def\Ifelse##1##2##3{\@ceuop{if}##1\@ceuop{then}{##2}\@ceuop{else}{##3}}%
  \def\Loop{\@ceuop{loop}}%
  \def\Every{\@ceuop{every}}%
  \def\And{\@ceubin{and}}%
  \def\Or{\@ceubin{or}}%
  \def\Fin{\@ceuop{fin}}%
  \def\AtLoop{\@ceuop{@loop}}%
  \def\AtAnd{\@ceubin{@and}}%
  \def\AtOr{\@ceubin{@or}}%
  \def\CanRun{\@ceuop{@canrun}}%
  \def\Nop{\@ceuop{@nop}}%
  \ensuremath{#1}\ignorespaces
  \egroup
}
\makeatother

% Proof trees (used to typeset proofs and rules).
\usepackage{bussproofs}
\def\labelSpacing{0em}
\def\ScoreOverhang{0em}

% Functions bcast, clear, eval, and isblocked.
\let\op\operatorname
\def\bcast{\op{\mathit{bcast}}}
\def\clear{\op{\mathit{clear}}}
\def\eval{\op{\mathit{eval}}}
\def\isblocked{\op{\mathit{isblocked}}}

% Rule labels.
\makeatletter
\def\@R#1{{\bfseries#1}}
\def\Rtag#1{\tag{\@R{#1}}\label{R:#1}}
\def\R#1{\hyperref[R:#1]{\@R{#1}}}
\makeatother

% Transition arrows.
\makeatletter
\newcommand{\@trans}[2][]{%
  \setbox0=\hbox{$\mathord{\longrightarrow}$}%
  \mathbin{%
    \mathrlap{\longrightarrow}%
    \notblank{#2}{%
      \rlap{\raisebox{-3pt}{\makebox[\wd0][c]{\ensuremath{\scriptstyle#2}}}}%
    }{}%
    \notblank{#1}{%
      \rlap{\raisebox{5pt}{\makebox[\wd0][c]{\ensuremath{\scriptstyle#1}}}}%
    }{}%
    \phantom{\longrightarrow}%
  }%
}
\newcommand{\trans}[1][]{\@trans[#1]{}}
\newcommand{\out}[1][]{\@trans[#1]{\mathit{out\,\;}}}
\newcommand{\nst}[1][]{\@trans[#1]{\mathit{nst\,\;}}}
\makeatother

% Copyright
%\setcopyright{none}
%\setcopyright{acmcopyright}
%\setcopyright{acmlicensed}
%\setcopyright{rightsretained}
%\setcopyright{usgov}
%\setcopyright{usgovmixed}
%\setcopyright{cagov}
%\setcopyright{cagovmixed}

% DOI
%\acmDOI{10.475/123_4}

% ISBN
%\acmISBN{123-4567-24-567/08/06}

%Conference
%\acmConference[LCTES'18]{ACM SIGPLAN/SIGBED}{June 2018}{Philadelphia, USA}
%\acmYear{1997}
%\copyrightyear{2016}

%\acmPrice{15.00}

%\acmBadgeL[http://ctuning.org/ae/ppopp2016.html]{ae-logo}
%\acmBadgeR[http://ctuning.org/ae/ppopp2016.html]{ae-logo}

%%% If you see 'ACMUNKNOWN' in the 'setcopyright' statement below,
%%% please first submit your publishing-rights agreement with ACM (follow link on submission page).
%%% Then please update our instructions page and copy-and-paste the NEW commands into your article.
%%% Please contact us in case of questions; allow up to 10 min for the system to propagate the information.
%%%
%%% The following is specific to LCTES'18 and the paper
%%% 'A Memory-Bounded, Deterministic and Terminating Semantics for the Synchronous Programming Language Céu'
%%% by Guilherme F. Lima, Rodrigo C. M. Santos, Edward Hermann Haeusler, Roberto Ierusalimschy, and Francisco Sant'Anna.
%%%
%\setcopyright{ACMUNKNOWN}
\setcopyright{acmcopyright}
\acmPrice{15.00}
\acmDOI{10.1145/3211332.3211334}
\acmYear{2018}
\copyrightyear{2018}
\acmISBN{978-1-4503-5803-3/18/06}
\acmConference[LCTES'18]{19th ACM SIGPLAN/SIGBED Conference on Languages, Compilers, and Tools for Embedded Systems}{June 19--20, 2018}{Philadelphia, PA, USA}

\begin{document}

\title[A Memory-Bounded, Deterministic and Terminating Semantics for~\CEU]
{A Memory-Bounded, Deterministic and Terminating Semantics for the
       Synchronous Programming Language~\CEU}

\author[R.\,C.\,M.~Santos]{Rodrigo C.\,M.~Santos}
\email{rsantos@inf.puc-rio.br}
\affiliation{PUC-Rio, Brazil}
%
\author[G.\,F.~Lima]{Guilherme F.~Lima}
\email{glima@inf.puc-rio.br}
\affiliation{PUC-Rio, Brazil}
%
\author[F.~Sant'Anna]{Francisco Sant'Anna}
\email{francisco@ime.uerj.br}
\affiliation{UERJ, Brazil}
%
\author[R.~Ierusalimschy]{Roberto Ierusalimschy}
\email{roberto@inf.puc-rio.br}
\affiliation{PUC-Rio, Brazil}
%
\author[E.\,H.~Haeusler]{Edward H. Haeusler}
\email{hermann@inf.puc-rio.br}
\affiliation{PUC-Rio, Brazil}

\begin{abstract}
\CEU is a synchronous programming language for embedded soft real-time systems.
%
It focuses on control-flow safety features, such as safe shared-memory
concurrency and safe abortion of lines of execution, while enforcing
memory-bounded, deterministic, and terminating reactions to the environment.
%
In this work, we present a small-step structural operational semantics for
\CEU and a proof that reactions have the properties enumerated above:
%
that for a given arbitrary timeline of input events, multiple executions of the
same program always react in bounded time and arrive at the same final finite
memory state.
%
%We also discuss some equivalence results and the relation between the proposed
%semantics and its actual implementation.
\end{abstract}

\terms{Languages, design, theory}
\keywords{Determinism, Termination, Operational semantics, Synchronous languages}

\begin{CCSXML}
<ccs2012>
 <concept>
  <concept_id>10003752.10010124.10010131.10010134</concept_id>
  <concept_desc>Theory of computation~Operational semantics</concept_desc>
  <concept_significance>500</concept_significance>
 </concept>
 <concept>
  <concept_id>10011007.10011006.10011008.10011009.10011014</concept_id>
  <concept_desc>Software and its
                engineering~Concurrent programming languages</concept_desc>
  <concept_significance>500</concept_significance>
 </concept>
 <concept>
  <concept_id>10010520.10010553.10010562.10010564</concept_id>
  <concept_desc>Computer systems
                organization~Embedded software</concept_desc>
  <concept_significance>300</concept_significance>
 </concept>
</ccs2012>
\end{CCSXML}
\ccsdesc[500]{Theory of computation~Operational semantics}
\ccsdesc[500]{Software and its engineering~Concurrent programming languages}
\ccsdesc[300]{Computer systems organization~Embedded software}

\maketitle

\section{Introduction}
\label{sec.intro}

\CEU~\cite{ceu.sensys13,ceu.tecs17} is a Esterel-based~\cite{esterel.ieee91}
programming language for embedded soft real-time systems that aims to offer a
concurrent, safe, and expressive alternative to C with the characteristics that
follow:
%
\begin{description}
\item [Reactive:] code only executes in reactions to events.
\item [Structured:] programs use structured control mechanisms, such as
    \code{await} (to suspend a line of execution), and \code{par} (to combine
    multiple lines of execution).
\item [Synchronous:] reactions run atomically and to completion on each line of
    execution, i.e., there's no implicit preemption or real parallelism.
\end{description}
%
Structured reactive programming lets developers write code in direct style,
recovering from the inversion of control imposed by event-driven
execution~\cite{rp.deprecating,rp.rescala,sync_async.cooperative}.
%
Synchronous languages offer a simple run-to-completion execution model that
enables deterministic execution and make formal reasoning tractable.
For this reason, it has been successfully adopted in safety-critical real-time
embedded systems~\cite{rp.twelve}.

Previous work in the context of embedded sensor networks evaluates the
expressiveness of \CEU in comparison to event-driven code in C and attests a
reduction in source code size (around 25\%) with a small increase in memory
usage (around 5--10\%)~\cite{ceu.sensys13}.
%
\CEU has also been used in the context of multimedia
systems~\cite{ceumedia.webmedia16} and games~\cite{ceu.mod15}.
%, and as an
%alternative language in an undergraduate-level course on embedded systems for
%the past 6 years.

\CEU inherits the synchronous and imperative mindset of Esterel but adopts a
simpler semantics with fine-grained execution control.
%
The list that follows summarizes the semantic peculiarities of
\CEU~\cite{ceu.tecs17}:
%
\begin{itemize}
    \item Fine-grained, intra-reaction deterministic execution, which makes
          \CEU fully deterministic.
    \item Stack-based execution for internal events, which provides a limited
          but memory-bounded form of subroutines.
    \item Finalization mechanism for lines of execution, which makes abortion
          safe with regard to external resources.
\end{itemize}

In this work, we present a formal small-step structural operational
semantics for \CEU and prove that it enforces memory-bounded, deterministic,
and terminating reactions to the environment, i.e., that for a given
arbitrary timeline of input events, multiple executions of the same program
always react in bounded time and arrive at the same final finite memory
state.
%
Conceiving a formal semantics for \CEU leads to a number of capabilities and
outcomes as follows:

\begin{enumerate}
\item
    Understanding, finding corner cases, and setting compromises.
    After the semantics was complete and discussed in extent, we found a bug in
    the order of execution of statements in a specific case.
\item
    Explaining core aspects of the language in a reduced, precise, and
    unambiguous way.
    This is particularly important when comparing languages that are similar in
    the surface (e.g., \CEU and Esterel).
\item
    Implementing the language.
    Small-step operational semantics typically describe abstract machines that
    are close to concrete implementations.
    The current implementation of the \CEU scheduler is based on the formal
    semantics.
\item
    Proving properties for particular or all programs in the language.
    We prove that all programs in \CEU are memory bounded, deterministic, and
    react in finite time.
\end{enumerate}

The last item is particularly important in the context of embedded systems:

\begin{description}
\item[Memory Boundedness:]
At compile time, we can ensure that the program fits in the device's restricted
memory and that it will not grow unexpectedly during runtime.
\item[Deterministic Execution:]
We can simulate an entire program execution providing an input history with the
guarantee that it will always have the same behavior.
This can be done in negligible time in a controlled development environment by
multiple developers before deploying the application in the actual device.
\item[Terminating Reactions:]
Real-time applications must guarantee responses within specified deadlines.
A terminating semantics enforces upper bounds for all reactions and guarantees
that programs always progress with the time.
In addition, it also ensures that programs always reach an idle state amenable
to entering in standby mode, which can improve the autonomy of battery-powered
devices.
\end{description}

\section{\CEU}
\label{sec.ceu}

\CEU is a synchronous reactive language inspired by
Esterel~\cite{esterel.ieee91} in which programs advance in a sequence of
discrete reactions to external events.
%
\CEU is designed for control-intensive applications, supporting concurrent
lines of execution, known as \emph{trails}, and broadcast communication
through events.
%
Computations within a reaction (such as expressions, assignments, and
C~calls) execute in no time in accordance to the synchronous
hypothesis~\cite{rp.hypothesis}.
%
The \code{await} is the only \CEU statement that actually ``consumes'' time.
%%
An~\code{await} statement blocks the current trail allowing the program to
advance its other trails; when all trails are blocked, the reaction
terminates and control returns to the environment.

In \CEU, every execution path within a loop must contain at least one
\code{await} statement~\cite{ceu.sensys13,esterel.primer}.
%
This restriction, which is checked statically by the \CEU compiler, ensures
that every reaction runs in bounded time, eventually terminating with all
program trails blocked in \code{await} statements.
%
\CEU has a further restriction which it shares with Esterel and synchronous
languages in general~\cite{esterel.preemption}: computations that actually
take a non-negligible time to run (e.g., cryptography or image processing
algorithms) violate the zero-delay hypothesis, and thus cannot be directly
implemented.

Listing~\ref{lst.syntax} below shows a compact reference of~\CEU.

\bgroup
\def\T<#1>{\langle\mathit{#1}\rangle}
\def\C#1#2{\hfill\rmfamily\itshape\makebox[#1\columnwidth][l]{//~#2}}
\begin{lstlisting}[
  basicstyle=\ttfamily\footnotesize,
  caption={The concrete syntax of \CEU.},
  label={lst.syntax},
]
|\C{1.}{Declarations:}|
input  $\T<type>$ $\T<ids>$;|\C{.6}{declare external input events}|
output $\T<type>$ $\T<ids>$;|\C{.6}{declare external output events}|
event  $\T<type>$ $\T<ids>$;|\C{.6}{declare internal events}|
var    $\T<type>$ $\T<id>$ = $\T<exp>$;|\C{.6}{declare and initialize variable}|

|\C{1.}{Event handling:}|
$\T<id>$ = await $\T<id>$;|\C{.6}{await event and assign the received value}|
$\T<id>$ = await $\T<time>$;|\C{.6}{await time and assign the delayed delta}|
emit $\T<id>$($\T<exp>$);|\C{.6}{emit event passing a value}|

|\C{1.}{Control-flow:}|
$\T<stmt>$ ; $\T<stmt>$|\C{.45}{sequence}|
if $\T<exp>$ then $\T<stmts>$ else $\T<stmts>$ end|\C{.45}{conditional}|
loop do $\T<stmts>$ end|\C{.45}{repetition}|
finalize $\T<stmts>$ with $\T<stmts>$ end|\C{.45}{finalization}|

par/or  do $\T<stmts>$ with $\T<stmts>$ end|\C{.45}{aborts when any side terminates}|
par/and do $\T<stmts>$ with $\T<stmts>$ end|\C{.45}{terminates when all sides terminate}|
par     do $\T<stmts>$ with $\T<stmts>$ end|\C{.45}{never terminates}|

|\C{1.}{Assignment \& integration with C:}|
$\T<id>$ = $\T<exp>$;|\C{.45}{assign value to variable}|
_$\T<id>$($\T<exps>$)|\C{.45}{call C function (id starts with `\_'\,)}|
\end{lstlisting}
\egroup

To make matters concrete, consider the program of Listing~\ref{lst.ceu}.
%
This program continuously turns a LED~on for~2 seconds and off for~1 second,
and terminates after~1 minute of activity with the LED off.
%
The implementation uses a \code{par/or} to run two activities in parallel:
an endless loop that blinks the LED on and off (lines~2--7), and a single
statement that waits for~1 minute before terminating (line~9).
%
The \code{par/or} block stands for a \emph{parallel-or} composition; when
executed it creates~$n$ parallel trails (in this case, $n=2$) and rejoins
them when any of these~$n$ trails terminates, automatically aborting the
other trails.

\begin{lstlisting}[
  numbers=left,
  basicstyle=\ttfamily\footnotesize,
  float=h,
  caption={\CEU program that blinks a LED during 1 minute.},
  label={lst.ceu},
]
par/or do
    loop do
        _led(1);
        await 2s;
        _led(0);
        await 1s;
    end
with
    await 1min;
end
_led(0);
\end{lstlisting}

In \CEU, any identifier prefixed with an underscore (e.g., \code{_led}) is
passed unchanged to the underlying~C compiler.
%
Therefore, access to~C is straightforward and syntactically traceable.
%
To ensure that programs operate synchronously, the compiler environment
should only provide access to~C operations that are assumed run in zero
time, such as non-blocking~I/O and access to \code{struct}'s.


\subsection{External and Internal Events}
\label{sec.ceu.evts}

\CEU defines time as a discrete sequence of reactions to unique external
input events.
%
These external input events are received from the environment; each of them
delimits a new logical unit of time and triggers a corresponding reaction.
%
The life-cycle of a program in \CEU can be summarized as
follows~\cite{ceu.sensys13}.
%
\begin{enumerate}
\item The program initiates the ``boot reaction'' in a single trail
  (parallel constructs may create new trails).
  %
\item Active trails execute until they await or terminate, one after
  another.  This step is called a \emph{reaction chain}, and always runs in
  bounded time.
  %
\item When all its trails are blocked, the program goes idle and the
  environment takes control.
  %
\item On the occurrence of a new external input event, the environment
  awakes \emph{all} trails awaiting that event, and the program goes back to
  step~(ii).
\end{enumerate}

A program must react to an event completely before handling the next one.
%
By the synchronous hypothesis, the time the program spends in step~(ii)
above is conceptually zero (in practice, negligible).
%
Thus, from the point of view of the environment, the program is always
idle---on step~(iii).
%
In practice, if a new external input event occurs while a reaction is being
computed, the event is saved on a queue, which effectively schedules it to
be processed by a subsequent reaction.


\subsubsection*{External events and discrete time}

\begin{figure}[b]
\centering
\includegraphics[width=\columnwidth]{tick}
\caption{The discrete notion of time in \CEU.}
\label{fig.ticks}
\end{figure}

The processing of external input events induces a discrete notion of time in
\CEU.
%
Figure~\ref{fig.ticks} illustrates this notion.
%
The continuous timeline shows an absolute reference clock with ``physical
timestamps'' for the event occurrences (e.g., event~\code{C} occurs at
$17ms521us$).
%
The discrete timeline shows how the same occurring events fit in the logical
notion of time of \CEU.
%
The boot reaction \code{boot-0} happens before any input on program startup.
%
Event~\code{A} ``physically'' occurs during \code{boot-0} but, because time
is discrete, its corresponding reaction can only execute afterwards, at
logical instant~\code{A-1}.
%
Similarly, event~\code{B} occurs during~\code{A-1} and its reaction is
postponed to execute at~\code{B-2}.
%
Event~\code{C} also occurs during~\code{A-1} but its reaction must also wait
for~\code{B-2} to execute and so it is postponed to execute at~\code{C-3}.
%
Event~\code{D} occurs during an idle period and can start immediately at
\code{D-4}.
%
Finally, two instances of event~\code{E} occur during~\code{D-4}; they are
handled in the subsequent reactions~\code{E-5} and~\code{E-6}.

Unique input events imply mutually exclusive reactions, which execute
atomically and never overlap.
%
Automatic mutual exclusion is a prerequisite for deterministic reactions as
we discuss in Section~\ref{sec.sem}.
%
%8<- - - - - - - - - - - - - - - - - - - - - - - - - - - - - - - - - - - - -
% \gl{A não ser que seja desenvolvido (e.g., explicado e comparado com
%   Esterel) eu acho que o parágrafo anterior é dispensável.}
% \fs{Tirei o "simplifies resoning about concurrency".
%     O resto do parágrafo é absoluto e tenta dar a intuição das condições para
%     ter determinismo.}
%- - - - - - - - - - - - - - - - - - - - - - - - - - - - - - - - - - - - ->8

In practice, the synchronous hypothesis for \CEU holds if reaction rate is
faster than the rate of incoming input events.
%
Otherwise, the program continuously accumulates a delay between the real
occurrence time and actual reaction time of events.
%
In the soft real-time systems targeted by \CEU (e.g., sensor networks,
multimedia systems, interactive games, etc.) such delay and postponed
reactions might be tolerated by users as long as they are infrequent and the
application does not take too long to catch up with real time.


\subsubsection*{Internal events as subroutines}

In \CEU, the queue-based processing of events described previously applies
only to external input events, i.e., those events submitted to the program
by the environment.
%
Internal events, which are those events generated internally by the program
via \code{emit} statements, are processed in a stack-based manner.
%
These internal events provide a fine-grained execution control and, because
of their stack-based processing, can be used to implement a limited form of
subroutine, as illustrated in Listing~\ref{lst.sub} below.

\begin{lstlisting}[
  numbers=left,
  basicstyle=\ttfamily\footnotesize,
  float=h,
  caption={A \CEU program with a ``subroutine''.},
  label={lst.sub},
]
event int* inc;     // subroutine |`|inc|'|
par/or do
    loop do         // definitions are loops
        var int* p = await inc;
        *p = *p + 1;
    end
with
    var int v = 1;
    $\cdots$
    emit inc(&v);   // call |`|inc|'|
    _assert(v==2);  // after return
end
\end{lstlisting}
\vskip-\baselineskip

In Listing~\ref{lst.sub}, the ``subroutine'' \code{inc} is defined as a loop
(lines~3--6) that continuously awaits its identifying event (line~4), and
increments the value passed to it by reference (lines~5).
%
A trail in parallel (lines~8--11) invokes the subroutine through the
\code{emit} statement at line~10.
%
Given the stack-based execution mode of internal events, after the emit
statement at line~10 is executed, the calling trail pauses, the subroutine
awakes (line~4), runs its body (yielding \code{v=2}), loops, and awaits the
next ``call'' (line~4, again).
%
Only after this sequence does the calling trail resumes and moves on to
execute the assertion on line~11.

\CEU also supports nested \code{emit} invocations for internal events.
%
For instance, the body of the subroutine \code{inc} in Listing~\ref{lst.sub}
could \code{emit} another internal event after awaking (line~4), creating a
new level in the stack.
%
We can think of the stack as a record of the nested, fine-grained internal
reactions that happen inside the same bigger reaction to some external
event.

This form of subroutine has a significant limitation though: it cannot
express recursion (an \code{emit} to itself is always ignored as a running
trail cannot be waiting on itself).
%
That said, it is this very limitation that brings important safety
properties to subroutines.
%
First, such subroutines are guaranteed to react in bounded time.
%
Second, memory for locals is also bounded, not requiring data stacks.
%
Third, \CEU subroutines can be safely used by the other primitives of the
language, such as parallel compositions and the \code{await} statement,
without breaking the programming model.
%
In particular, after calling a subroutine these primitives wait while
keeping context information, such as locals and the program counter, which
makes the calls behave similarly to those of
coroutines~\cite{lua.coroutines}.
%
Finally, in previous work, we built other advanced control mechanisms on top
of internal events, such as resumable exceptions and reactive
variables~\cite{ceu.rem13}.
%
%In Section~\ref{sec.adv.excpt} we show how to use them to implement
%exceptions.
%
%8<- - - - - - - - - - - - - - - - - - - - - - - - - - - - - - - - - - - - -
\gl{Revisei do início da Seção~2 até aqui.}
%- - - - - - - - - - - - - - - - - - - - - - - - - - - - - - - - - - - - ->8

\subsection{Shared-Memory Concurrency}
    - referenciar warnings

Embedded applications make extensive use of global memory and shared resources,
such as through memory-mapped registers and system calls to device drivers.
Hence, an important goal of \CEU is to ensure a reliable behavior for programs
with concurrent lines of execution sharing memory and interacting with the
environment.

\begin{figure}[h]
\begin{minipage}[h]{0.45\linewidth}
\begin{lstlisting}[numbers=left,xleftmargin=3em]
input void A, B;
var int x = 1;
par/and do
    await A;
    x = x + 1;
with
    await B;
    x = x * 2;
end
\end{lstlisting}
\centering\small{\ax Accesses to \code{x} are never concurrent.}
\end{minipage}
%
\begin{minipage}[h]{0.45\linewidth}
\begin{lstlisting}[numbers=left,xleftmargin=3em]
input void A;
var int y = 1;
par/and do
    await A;
    y = y + 1;
with
    await A;
    y = y * 2;
end
\end{lstlisting}
\centering\small{\bx Accesses to \code{y} are concurrent but deterministic.}
\end{minipage}
%\rule{8.4cm}{0.37pt}
\caption{ Shared-memory concurrency in \CEU:
example \ax is safe because the trails access \code{x} atomically in different
reactions;
example \bx is unsafe because both trails access \code{y} in the same reaction.
\label{lst.shared}
}
\end{figure}

In \CEU, when multiple trails are active during the same reaction, they are
scheduled in lexical order, i.e., in the order they appear in the program
source code.
%
For instance, consider the two examples in Figure~\ref{lst.shared}, both
defining a shared variable (ln. 2), and assigning to it in parallel trails (ln.
5, 8).

In the example \ax, the two assignments to \code{x} can only execute in
reactions to different events \code{A} and \code{B}, which cannot occur
simultaneously by definition (Section~\ref{sec.ceu.evts}).
Hence, for the sequence of events \code{A->B}, \code{x} becomes \code{4}
(\code{(1+1)*2}), while for \code{B->A}, \code{x} becomes \code{3}
(\code{(1*2)+1}).

In the example \bx, the two assignments to \code{y} are simultaneous because
they execute in reaction to the same event \code{A}.
Since \CEU employs lexical order for intra-reaction statements, the execution
is still deterministic, and \code{y} always becomes \code{4} (\code{(1+1)*2}).
%
However, that an (apparently innocuous) change in the order of trails modifies
the behavior of the program.
%
To mitigate this threat, \CEU performs concurrency checks at compile time to
detect conflicting accesses to shared variables:
if a variable is written in a trail segment, then a concurrent trail segment
cannot read or write to that variable~\cite{ceu.sensys13}.
%
Nonetheless, the static checks are optional and do not affect the semantics of
the language.

\subsection{Abortion and Finalization}

The \code{par/or} of \CEU is an \emph{orthogonal abortion
mechanism}~\cite{esterel.preemption} because the two sides in the composition
need not be tweaked with synchronization primitives or state variables in order
to affect each other.
%
In addition, abortion is \emph{immediate} in the sense that it executes
atomically in the current micro reaction.
%
Immediate orthogonal abortion is a distinctive feature of synchronous languages
and cannot be expressed effectively in traditional (asynchronous)
multi-threaded languages~\cite{esterel.preemption,sync_async.threadsstop}.

However, aborting lines of execution that deal with external resources may lead
to inconsistencies.
%
For this reason, \CEU provides a \code{finalize} construct to unconditionally
execute a series of statements even if the enclosing block is aborted.

\CEU also enforces the use of \code{finalize} for system calls that deal with
pointers representing resources, as illustrated in the two examples of
Figure~\ref{lst.fin.ceu}:
%
\begin{itemize}
\item If \CEU \textbf{passes} a pointer to a system call (ln. \ax:5), the
pointer represents a \textbf{local} resource (ln. \ax:2) that requires finalization
(ln. \ax:7).
\item If \CEU \textbf{receives} a pointer from a system call return (ln. \bx:4),
the pointer represents an \textbf{external} resource (ln. \bx:2) that requires
finalization (ln. \bx:6).
\end{itemize}
%
\CEU tracks the interaction of system calls with pointers and requires
finalization clauses to accompany them.
%
In the example in Figure~\ref{lst.fin.ceu}.a, the local variable \code{msg}
(ln. 2) is an internal resource passed as a pointer to \code{\_send\_request}
(ln. 5), which is an asynchronous call that transmits the buffer in the
background.
If the block aborts (ln. 11) before receiving an acknowledge from the
environment (ln. 9), the local \code{msg} goes out of scope and the external
transmission now holds a \emph{dangling pointer}.
The finalization ensures that the transmission also aborts (ln. 7).
%
In the example in Figure~\ref{lst.fin.ceu}.b, the call to \code{\_fopen} (ln.
4) returns an external file resource as a pointer.
If the block aborts (ln. 12) during the \code{await A} (ln. 9), the file
remains open as a \emph{memory leak}.
The finalization ensures that the file closes properly (ln. 6).
%
In both cases, the code does not compile without the \code{finalize}
construct.%
\footnote{
The compiler only forces the programmer to write the finalization clause, but
cannot check if it actually handles the resource properly.
}

The finalization mechanism of \CEU is fundamental to preserve the orthogonality
of the \code{par/or} construct since the clean up code is encapsulated in the
aborted trail itself.

\begin{figure}
\begin{minipage}[t]{0.48\linewidth}
\begin{lstlisting}[numbers=left,xleftmargin=3.5em,mathescape=true]
par/or do
   var _buffer_t msg;
   <...> // prepare msg
   finalize
      _send_request(&msg);
   with
      _send_cancel(&msg);
   end
   await SEND_ACK;
with
   <...>
end
.
\end{lstlisting}
\centering\small{\ax Local resource finalization}
\end{minipage}
%
\begin{minipage}[t]{0.48\linewidth}
\begin{lstlisting}[numbers=left,xleftmargin=3.5em]
par/or do
   var _FILE* f;
   finalize
      f = _fopen(...);
   with
      _fclose(f);
   end
   _fwrite(..., f);
   await A;
   _fwrite(..., f);
with
   <...>
end
\end{lstlisting}
\centering\small{\bx External resource finalization}
\end{minipage}
%\rule{8.4cm}{0.37pt}
\caption{
\CEU enforces the use of finalization to prevent \emph{dangling pointers} for
local resources and \emph{memory leaks} for external resources.
\label{lst.fin.ceu}
}
\end{figure}

\newcommand{\NST}{\1\xrightarrow[\mathit{nst}]\1}
\newcommand{\OUT}{\1\xrightarrow[\mathit{out}]\1}
\newcommand{\LL}{\langle}
\newcommand{\RR}{\rangle}
\newcommand{\DS}{\displaystyle}

\newcommand{\1}{\;}
\newcommand{\2}{\;\;}
\newcommand{\3}{\;\;\;}
\newcommand{\5}{\;\;\;\;\;}

\section{Formal Semantics}
\label{sec.sem}

In this section, we introduce a reduced syntax for \CEU and propose an 
operational semantics to formally describe the language.
We describe a small synchronous kernel highlighting the peculiarities of \CEU, 
in particular, the stack-based execution for internal events.
%
For the sake of simplicity, we focus on the control aspects of the language, 
leaving out side-effects and system calls (which behave like in conventional 
imperative languages).

\subsection{Abstract Syntax}
\label{sec.sem.syntax}

%-
% \begin{lstlisting}[
%   %numbers=left,
%   basicstyle=\ttfamily\footnotesize,
%   float=h,
%   caption={Reduced syntax of \CEU.},
%   label={lst.formal.syntax},
%   mathescape=true
% ]
%                                    // primary expressions
%   p ::= mem(id)                    (any memory access to `id')
%       $|$ awaitExt(id)               (await external event `id')
%       $|$ awaitInt(id)               (await internal event `id')
%       $|$ emitInt(id)                (emit internal event `id')
%       $|$ break                      (loop escape)
%                                    // compound expressions
%       $|$ if mem(id) then p else p   (conditional)
%       $|$ p ; p                      (sequence)
%       $|$ loop p                     (repetition)
%       $|$ every id p                 (event iteration)
%       $|$ p and p                    (par/and)
%       $|$ p or p                     (par/or)
%       $|$ fin p                      (finalization)
%                                    // derived by semantic rules
%       $|$ p @loop p                  (unwinded loop)
%       $|$ p @and q                   (unwinded par/and)
%       $|$ p @or q                    (unwinded par/or)
%       $|$ @canrun(n)                 (can run on stack level `n')
%       $|$ @nop                       (terminated expression)
% \end{lstlisting}
%-

The grammar below defines the syntax of a subset of \CEU that is
sufficient to describe all semantic peculiarities of the language.
\bgroup
\def\lbl#1{\qquad\text{\emph{#1}}}%
\newdimen\X
\X=-1\jot
\begin{alignat*}{2}
  p\Coloneqq
      &\enspace\ceu{\Mem(\Id)}
      &&\lbl{any memory access to~``$\ceu\Id$''}\\[\X]
      %%
  \mid&\enspace\ceu{\AwaitExt(\Id)}
      &&\lbl{await external event~``$\ceu\Id$''}\\[\X]
      %%
  \mid&\enspace\ceu{\AwaitInt(\Id)}
      &&\lbl{await internal event~``$\ceu\Id$''}\\[\X]
      %%
  \mid&\enspace\ceu{\EmitInt(\Id)}
      &&\lbl{emit internal event~``$\ceu\Id$''}\\[\X]
      %%
  \mid&\enspace\ceu{\Break}
      &&\lbl{loop escape}\\[\X]
      %%
  \mid&\enspace\ceu{\Ifelse{b}{p_1}{p_2}}
      &&\lbl{conditional}\\[\X]
      %%
  \mid&\enspace\ceu{p1\,;\,p2}
      &&\lbl{sequence}\\[\X]
      %%
  \mid&\enspace\ceu{\Loop p_1}
      &&\lbl{repetition}\\[\X]
      %%
  \mid&\enspace\ceu{\Every{\Id}\ p_1}
      &&\lbl{event iteration}\\[\X]
      %%
  \mid&\enspace\ceu{p_1\And p_2}
      &&\lbl{par/and}\\[\X]
      %%
  \mid&\enspace\ceu{p_2\Or p_2}
      &&\lbl{par/or}\\[\X]
      %%
  \mid&\enspace\ceu{\Fin p}
      &&\lbl{finalization}\\[\X]
      %%
  \mid&\enspace\ceu{p_1\AtLoop p_2}
      &&\lbl{unwinded loop}\\[\X]
      %%
  \mid&\enspace\ceu{p_1\AtAnd\ p_2}
      &&\lbl{unwinded par/and}\\[\X]
      %%
  \mid&\enspace\ceu{p_1\AtOr\ p_2}
      &&\lbl{unwinded par/or}\\[\X]
      %%
  \mid&\enspace\ceu{\CanRun(n)}
      &&\lbl{can run on stack level~$n$}\\[\X]
      %%
  \mid&\enspace\ceu{\Nop}
      &&\lbl{terminated program}
\end{alignat*}
\egroup

The~$\ceu{\Mem(id)}$ primitive represents all accesses, assignments, and system calls
that affect a memory location identified by~$id$.
%
According to the synchronous hypothesis of \CEU, $\ceu{\Mem}$ expressions are 
considered to be atomic and instantaneous.
%
As the challenging parts of \CEU reside on its control structures, we are not 
concerned here with a precise semantics for side effects, but only with their 
occurrences in programs.
%
%The special notation $nop$ is used to represent an innocuous $mem$ expression 
%(it can be thought as a synonym for $mem(\epsilon)$, where $\epsilon$ is an 
%unused identifier).

We assume that $\ceu{\Mem}$, $\ceu{\AwaitExt}$, and~$\ceu{\AwaitInt}$ and~$\ceu{\EmitInt}$ expressions do not
share identifiers: any identifier is either a variable, an external event,
or an internal event.

Most expressions in the abstract language are mapped to
their counterparts in the concrete language.  The exceptions are the
finalization block~$\ceu{\Fin{p}}$ and the \texttt{@}-expressions, which are
internal expressions that result from the
expansion of awaits, emits, and loops by the transition rules to be discussed.

Regarding other mismatches between the concrete and abstract languages, the concrete
\code{await} and \code{emit} primitives support communication of values between
them, e.g., an ``\code{emit a(10)}'' awakes a ``\code{v=await a}'' setting
variable~\code{v} to~10.
To reproduce this functionality in the formal semantics, we can use a shared
variable to hold the value of an $\ceu{\EmitInt}$ and access it after the
corresponding $\ceu{\AwaitInt}$.
%
Also, a ``\code{finalize $A$ with $B$ end; $C$}'' in the concrete language is
equivalent to ``\ceu{A;\;((\Fin{B})\ \Or\ C)}'' in the abstract language.
In the concrete language, $A$ and~$C$ execute in sequence, and
the finalization code~$B$ is implicitly suspended waiting for~$C$
termination.
In the abstract language, ``$\ceu{\Fin B}$'' suspends forever when reached (it is
an awaiting expression that never awakes).
Hence, we need an explicit \code{or} to execute~$C$ in parallel, whose
termination aborts ``$\ceu{\Fin B}$'', which finally causes~$B$ to
execute (by the semantic rules to be discussed).

\subsection{Operational Semantics}

The core of our semantics describes how a program reacts to a single external 
input event, i.e., starting from an input event, how the program behaves and 
becomes idle again to proceed to a subsequent reaction.
%
We use a set of small-step operational rules, which are built in such a way 
that at most one transition is possible at any time, resulting in deterministic 
reactions.
%
The transition rules map a triple with a program~$p$, a stack level~$n$, and an
emitted event~$e$ to a modified triple as follows:
\[
  \<p,n,e>\trans\<p',n',e'>\,,
\]
where~$p,p'\in\P$ are abstract-language programs, $n,n'\in\N$ are
nonnegative integers representing the current stack level,
and~$e,e'\in\E\cup\{\nil\}$ are the events emitted before and after the
transition (both possibly the empty event~$\nil$).

We will refer to the triples on the left-hand and right-hand sides of
symbol~$\to$ as \emph{descriptions} (denoted~$\delta$).  The triple on the
left-hand side of symbol~$\to$ is called the \emph{input description}, and
the triple on its right-hand side is called the \emph{output description}.

%-
% \begin{align*}
% p, p' &\in\P
%     && (program~as~described~in~Listing~\ref{lst.formal.syntax})
% \\
% n, n' &\in\N
%     && (current~stack~level)
% \\
% e, e' &\in\E \cup \{\epsilon\}
%     && (emitted~event,~possibly~none)
% \end{align*}
%-

At the beginning of a reaction to an input event~$id$, the input description is
initialized with stack level~0 ($n=0$) and with the emitted event
($e=id$).
%, but \code{emitInt} expressions can increase the stack level.
At the end of a reaction, after an arbitrary but finite number of transitions,
the last output description will block with a (possibly) modified program~$p'$, at stack
level~0, and with no event emitted~($\nil$):
\[
  \<p,0,e>\mathbin{\trans[*]}\<p',0,\nil>\,.
\]

We distinguish between two types of transition rules:
    \emph{outermost} transitions $\out$ and
    \emph{nested} transitions $\nst$\,.

\subsubsection*{Outermost transitions}

The~$\out$ rules \R{push} and \R{pop} are non-recursive definitions
that only apply to the program as a whole and manipulate the stack level:
\begin{align*}
  &\AxiomC{$e\ne\nil$}
  \UnaryInfC{$\<p,n,e>\out\<\bcast(p,e),n+1,\nil>$}
  \DisplayProof
  \Rtag{push}\\[2\jot]
  %%
  &\AxiomC{$n>0$}
  \AxiomC{$\ceu{p=\Nop}\vee\isblocked(p,n)$}
  \BinaryInfC{$\<p,n,\nil>\out\<p,n-1,\nil>$}
  \DisplayProof
  \Rtag{pop}
\end{align*}

%-
% { \setlength{\jot}{20pt}
% \begin{eqnarray*}
% & \frac
%     { \DS e \neq \epsilon }
% %   -----------------------------------------------------------
%     { \DS \LL p,n,e \RR \OUT \LL bcast(p),n+1,\epsilon \RR }
%     & \textbf{(push)}   \\
% %%%
% & \frac
%     { \DS n>0, \2 ((p=@nop) \vee isblocked(n,p)) }
% %   -----------------------------------------------------------
%     { \DS \LL p,n,\epsilon \RR \OUT \LL p,n-1,\epsilon \RR }
%     & \textbf{(pop)}    \\
% %%%
% %& \LL p,0,\epsilon \RR \1\xrightarrow\1 \bot
%     %& \textbf{(end)}    %\\
% \end{eqnarray*}
% }
%-

Rule \R{push} matches whenever there is an emitted event in the input
description,
and instantly broadcasts the event to the program, which means
    (a)~awaking active $\ceu{\AwaitExt}$ or $\ceu{\AwaitInt}$ expressions altogether (see function~$\bcast$ in
        Figure~\ref{fig.bcast}),
    (b)~creating a nested reaction by increasing the stack level, and, at the same time, and
    (c)~consuming the event ($e$ becomes~$\nil$).
%
Rule \R{push} is the only rule in the semantics that matches an
emitted event and also immediately consumes it.

Rule \R{pop} only decreases the stack level, not affecting the
program, and only applies if the program is blocked (see function~$\isblocked$ in
Figure~\ref{fig.isblocked}).
This condition ensures that an $\ceu{\EmitInt}$ only resumes after its internal
reaction completes and blocks, as discussed in Section~\ref{sec.ceu.evts}.

At the beginning of the reaction, an external event is emitted, which
will trigger rule \textbf{push}, which will immediately raise the stack level
to~1.
At the end of the reaction, the program will block or terminate and
successive applications of
rule~\R{pop} will eventually lead to a description containing this
same program at stack level~0.
(Rule \R{pop} is the only rule that decreases the stack level.)

\subsubsection*{Nested transitions}

The~$\nst$ rules are recursive definitions with the following general
format:
\[
\<p,n,\nil>\nst\<p',n,e>.
\]
%
%-
% \begin{align*}
% \LL p, n,\epsilon \RR &\NST
% \LL p',n,e        \RR
%     & \textbf{(rule-inner)}
% \end{align*}
%-
%
Nested transitions do not affect the stack level and never have an emitted
event as a precondition.  The distinction between~$\out$ and~$\nst$ prevents
rules \R{push} and \R{pop} from matching and, consequently, from
inadvertently modifying the current stack level before the nested reaction
is complete.

A complete reaction consists of the following series of transitions:
\begin{align*}
  \<p,0,e_\ext>\out[push]\<p_1,1,\nil>
  \Big[\null\nst[*]\null\out\null\Big]\!\!\ast
  \null\nst[*]\null\out[pop]\<p',0,\nil>\,.
\end{align*}
%
%-
% \begin{align*}
% a) &\5\5
%     \LL p,0,ext \RR
%         \1\xrightarrow[out]{push}\1
%     \LL q,1,\epsilon \RR
% \\
% b) &\5\5 \1[ \1\xrightarrow[in]{*}\1
%     \LL r,i,e \RR
%         \1\xrightarrow[out]\1
%     \LL s,j,\epsilon \RR \1]*
% \\
% c) &\5\5 \1\xrightarrow[in]{*}\1
%     \LL t,k,\epsilon \RR
%         \1\xrightarrow[out]{pop}\1
%     \LL u,0,\epsilon \RR
% \end{align*}
%-
%
First, a~$\out[push]$ starts a nested reaction at level~1.
Then, a series of alternations between zero or more~$\nst$ transitions (nested reactions) and a
single~$\out$ transition (stack operation) takes place.
Finally, a last~$\out[pop]$ transition decrements the
stack level to~0 and terminates the reaction.

The~$\nst$ transition rules for atomic expressions are defined as follows:
%
{ \setlength{\jot}{20pt}
\begin{align*}
\LL mem(id), n, \epsilon \RR &\NST
\LL @nop, n, \epsilon \RR
    & \textbf{(mem)}        \\
%%%
\LL emit(id), n, \epsilon \RR &\NST
\LL @canrun(n), n, id \RR
    & \textbf{(emitInt)}    \\
%%%
\LL @canrun(n), n, \epsilon \RR &\NST
\LL @nop, n, \epsilon \RR
    & \textbf{(canrun)}     \\
\end{align*}
}
%
A $mem$ operation becomes a $@nop$ which indicates the memory access (rule
\textbf{mem}).
An $emitInt(id)$ generates an event $id$ and transits to a $@canrun(n)$ which
can only resume at level $n$ (rule \textbf{emitInt}).
Since all $\NST$ rules can only transit with $e=\epsilon$, an $emitInt$ causes
rule \textbf{push} to execute at the outer level, creating a new level $n+1$ on
the stack.
Also, with the new stack level, the resulting $@canrun(n)$ itself cannot
transit, providing the desired stack-based semantics for internal events.

Proceeding to compound expressions, the rules for conditionals and sequences 
are straightforward:
%
{ \setlength{\jot}{20pt}
\begin{eqnarray*}
& \frac
    { \DS val(id) \neq 0 }
%   -----------------------------------------------------------
    { \DS \LL (if~mem(id)~then~p~else~q),n,\epsilon \RR \NST
          \LL p, n, \epsilon \RR }
    & \textbf{(if-true)}       \\
%%%
& \frac
    { \DS val(id,n) = 0 }
%   -----------------------------------------------------------
    { \DS \LL (if~mem(id)~then~p~else~q),n,\epsilon \RR \NST
          \LL q,n,\epsilon \RR }
    & \textbf{(if-false)}       \\
%%%
& \frac
    { \DS \LL p,n,\epsilon \RR \NST \LL p',n,e \RR }
%   -----------------------------------------------------------
    { \DS \LL (p~;~q), n, \epsilon \RR \NST \LL (p'~;~q), n, e \RR }
    & \textbf{(seq-adv)}      \\
%%%
& \LL (@nop~;~q),n,\epsilon \RR \NST  \LL q,n,\epsilon \RR
    & \textbf{(seq-nop)}      \\
%%%
& \LL (break~;~q),n,\epsilon \RR \NST \LL break,n,\epsilon \RR
    & \textbf{(seq-brk)}
\end{eqnarray*}
}
%
Given that our semantics focuses on control, rules \textbf{if-true} and 
\textbf{if-false} are the only to query $mem$ expressions.
%
Function $val$ receives a memory identifier and returns its current value.
%
%Although the value here is arbitrary, it is unique in a reaction, because a 
%given expression can execute only once within it (remember that $loops$ must 
%contain $awaits$ which, from rule \textbf{await}, cannot awake in the same 
%reaction they are reached).
%For all other rules, we omit these values (e.g., \textbf{seq-nop}).

%As determined for nested rules, compound expressions also can only have
%$\epsilon$ as a precondition and they never modify $n$.
%However, they can still emit an event to nest another reaction.
%For instance, in rule \textbf{seq-adv}, if the sub-expression $p$ emits event
%$e$, the whole composition also emits $e$.
%However, rules \textbf{push} and \textbf{pop} can only match at the outermost
%level.

The rules for loops are analogous to sequences, but use \code{`@'} as 
separators to properly bind breaks to their enclosing loops:
%
{ \setlength{\jot}{20pt}
\begin{eqnarray*}
& \LL (loop~p),n,\epsilon \RR \NST \LL (p~@loop~p), n, \epsilon \RR
    & \textbf{(loop-expd)}       \\
%%%
& \frac
    { \DS \LL p,n,\epsilon \RR \NST \LL p',n,e \RR }
% -----------------------------------------------------------
    { \DS \LL (p~@loop~q),n,\epsilon \RR \NST \LL (p'~@loop~q), n, e \RR }
    & \textbf{(loop-adv)}    \\
%%%
& \LL (@nop~@loop~p), n, \epsilon \RR \NST \LL (loop~p), n, \epsilon \RR
    & \textbf{(loop-nop)}    \\
%%%
& \LL (break~@loop~p), n, \epsilon \RR \NST \LL @nop, n, \epsilon \RR
    & \textbf{(loop-brk)}
\end{eqnarray*}
}

%
When a program encounters a $loop$, it first expands its body in sequence with 
itself (rule \textbf{loop-expd}).
Rules \textbf{loop-adv} and \textbf{loop-nop} are similar to rules 
\textbf{seq-adv} and \textbf{seq-nop}, advancing the loop until they reach a 
$@nop$.
However, what follows the loop is the loop itself (rule \textbf{loop-nop}).
Note that if we used \code{`;'} as a separator in loops, rules 
\textbf{loop-brk} and \textbf{seq-brk} would conflict.
%
Rule \textbf{loop-brk} escapes the enclosing loop, transforming everything into 
a $@nop$.
%Rule \textbf{loop-brk} escapes the enclosing loop, transforming everything 
%into a $clear(p)$.
%We cannot simply transform the loop into a $nop$ because its body may be a 
%parallel composition containing finalization blocks.

Proceeding to parallel compositions, the semantic rules for $and$ and $or$ 
always force transitions on their left branches $p$ to occur before their right 
branches $q$:
%
{ \setlength{\jot}{20pt}
\begin{eqnarray*}
& \LL (p~and~q),n,\epsilon \RR \NST \LL (p~@and~(@canrun(n)~;~q)),n,\epsilon \RR
    & \textbf{(and-expd)}       \\
%%%
& \frac
    { \DS \LL p,n,\epsilon \RR \NST \LL p',n,e \RR }
%   -----------------------------------------------------------
    { \DS \LL (p~@and~q),n,\epsilon \NST \LL (p'~@and~q),n,e \RR }
    & \textbf{(and-adv1)}      \\
%%%
& \frac
    { \DS isblocked(n,p) \1,\2 \LL q,n,\epsilon \RR \NST \LL q',n,e \RR }
%   -----------------------------------------------------------
    { \DS \LL (p~@and~q),n,\epsilon \RR \NST \LL (p~@and~q'), n, e \RR }
    & \textbf{(and-adv2)}      \\
%%%
& \LL (p~or~q), n, \epsilon \RR \NST \LL (p~@or~(@canrun(n)~;~q)), n, \epsilon \RR
    & \textbf{(or-expd)}       \\
%%%
& \frac
    { \DS \LL p,n,\epsilon \RR \NST \LL p',n,e \RR }
%   -----------------------------------------------------------
    { \DS \LL (p~@or~q),n,\epsilon \RR \NST \LL (p'~@or~q), n, e \RR }
    & \textbf{(or-adv1)}   \\
%%%
& \frac
    { \DS isblocked(n,p) \1,\2 \LL q,n,\epsilon \RR \NST \LL q',n,e \RR }
%   -----------------------------------------------------------
    { \DS \LL (p~@or~q),n,\epsilon \RR \NST \LL (p~@or~q'), n, e \RR }
    & \textbf{(or-adv2)}   %\\
\end{eqnarray*}
}
%
Rules \textbf{and-expd} and \textbf{or-expd} insert a $@canrun(n)$ at the
beginning of the right branch.
This ensures that an $emitInt$ on the left branch, which transits to a
$@canrun(n)$, still resumes before the right branch starts.
%
The deterministic behavior of the semantics relies on the \emph{isblocked} 
predicate (see Figure~\ref{fig.isblocked}) which appears in rules
\textbf{and-adv2} and \textbf{or-adv2}.
These rules require the left branch $p$ to be blocked in order to allow the 
right branch transition from $q$ to $q'$.

For a parallel $@and$, if one of the sides terminates, the composition is
simply substituted by the other side (rules \textbf{and-nop1} and
\textbf{and-nop2}.
%
For a parallel $or$, if one of the sides terminates, the whole composition 
terminates, also applying the $clear$ function to properly finalize the aborted 
side (rules \textbf{or-nop1} and \textbf{or-nop2}):
%
{ \setlength{\jot}{20pt}
\begin{eqnarray*}
& \LL (@nop~@and~q), n, \epsilon \RR \NST \LL q,n,\epsilon \RR
    & \textbf{(and-nop1)}   \\
%%%
& \frac
    { \DS isblocked(n,p) }
%   -----------------------------------------------------------
    { \DS \LL (p~@and~@nop), n, \epsilon \RR \NST \LL p,n,\epsilon \RR }
    & \textbf{(and-nop2)}   \\
%%%
& \LL (@nop~@or~q), n, \epsilon \RR \NST \LL clear(q),n,\epsilon \RR
    & \textbf{(or-nop1)}   \\
%%%
& \frac
    { \DS isblocked(n,p) }
%   -----------------------------------------------------------
    { \DS \LL (p~@or~@nop), n, \epsilon \RR \NST \LL clear(p),n,\epsilon \RR }
    & \textbf{(or-nop2)}   %\\
\end{eqnarray*}
}
%
The $clear$ function (see Figure~\ref{fig.formal.clear}) concatenates all
active $fin$ bodies of the side being aborted, so that they execute before the
composition rejoins.
Note that there are no transition rules for $fin$ expressions.
This is because once reached, a $fin$ expression halts and will only execute 
when it is aborted by a trail in parallel and is expanded by the $clear$ 
function.
%In Section~\ref{sec.formal.fins}, we show how to map a finalization block in 
%the concrete language to a $fin$ in the formal semantics.
%
Note also that there is a syntactic restriction that $fin$ bodies cannot
contain awaiting expressions ($awaitExt$, $awaitInt$, $every$, and $fin$),
i.e., they are guaranteed to execute entirely within a reaction.

Finally, a $break$ in one of the sides in parallel escapes the closest 
enclosing $loop$, properly aborting the other side by applying the $clear$ 
function:
%
{ \setlength{\jot}{20pt}
\begin{eqnarray*}
& \LL (break~@and~q), n, \epsilon \RR \NST \LL (clear(q)~;~break),n,\epsilon \RR
    & \textbf{(and-brk1)}   \\
%%%
& \frac
    { \DS isblocked(n,p) }
%   -----------------------------------------------------------
    { \DS \LL (p~@and~break), n, \epsilon \RR \NST \LL (clear(p)~;~break),n,\epsilon \RR }
    & \textbf{(and-brk2)}   \\
%%%
& \LL (break~@or~q),n,\epsilon \RR \NST \LL (clear(q)~;~break),n,\epsilon \RR
    & \textbf{(or-brk1)}   \\
%%%
& \frac
    { \DS isblocked(n,p) }
%   -----------------------------------------------------------
    { \DS \LL (p~@or~break),n,\epsilon \RR \NST \LL (clear(p)~;~break),n,\epsilon \RR }
    & \textbf{(or-brk2)}   %\\
\end{eqnarray*}
}
%
A reaction eventually blocks in $awaitExt$, $awaitInt$, $every$, $fin$, and
$@canrun$ expressions in parallel trails.
%
If no trails are blocked in $@canrun$ expressions, it means that the program 
cannot advance in the current reaction.
%
However, $@canrun$ expressions can still resume in lower stack indexes and will
eventually resume in the current reaction (see rule \textbf{pop}).

\begin{figure}
{\small
\begin{align*}
  bcast(e, awaitExt(e)) &= @nop                         \\
  bcast(e, awaitInt(e)) &= @nop                         \\
  bcast(e, every~e~p)   &= p;~every~e~p                 \\
  bcast(e, @canrun(n))  &= @canrun(n)                   \\
  bcast(e, fin~p)       &= fin~p                        \\
  bcast(e, p~;~q)       &= bcast(e,p)~;~q               \\
  bcast(e, p~@loop~q)   &= bcast(e,p)~@loop~q           \\
  bcast(e, p~@and~q)    &= bcast(e,p)~@and~bcast(e,q)   \\
  bcast(e, p~@or~q)     &= bcast(e,p)~@or~bcast(e,q)    \\
  bcast(e, \_)          &= \bot \2 (mem,emitInt,break,if,  \\
                                 & \5\5 loop,and,or,@nop) %\\
\end{align*}
}
\caption{
The function $bcast$ awakes awaiting trails matching the event by converting
$awaitExt$ and $awaitInt$ to $@nop$ expressions, and by unwinding $every$
expressions.
\label{fig.bcast}
}
\end{figure}

\begin{figure}
{\small
\begin{align*}
  isblocked(n, \1 awaitExt(id)) &= true                                   \\
  isblocked(n, \1 awaitInt(id)) &= true                                   \\
  isblocked(n, \1 every~e~p)    &= true                                   \\
  isblocked(n, \1 @canrun(m))   &= (n > m)                                \\
  isblocked(n, \1 fin~p)        &= true                                   \\
  isblocked(n, \1 p~;~q)        &= isblocked(n,p)                         \\
  isblocked(n, \1 p~@loop~q)    &= isblocked(n,p)                         \\
  isblocked(n, \1 p~@and~q)     &= isblocked(n,p) \wedge isblocked(n,q)   \\
  isblocked(n, \1 p~@or~q)      &= isblocked(n,p) \wedge isblocked(n,q)   \\
  isblocked(n, \1 \_)           &= false \2 (mem,emitInt,break,if,        \\
                                & \5\5\5\1 loop,and,or,@nop)   %\\
\end{align*}
}
\caption{
The predicate $isblocked$ is true only if all branches in parallel are blocked
waiting for events, finalization clauses, or certain stack levels.
\label{fig.isblocked}
}
\end{figure}

\begin{figure}[b]
{\small
\begin{align*}
  clear( awaitExt(e) ) &= @nop                  \\
  clear( awaitInt(e) ) &= @nop                  \\
  clear( every~e~p )   &= @nop                  \\
  clear( @canrun(n) )  &= @nop                  \\
  clear( fin~p )       &= p                     \\
  clear( p~;~q )       &= clear(p)              \\
  clear( p~@loop~q )   &= clear(p)              \\
  clear( p~@and~q )    &= clear(p)~;~clear(q)   \\
  clear( p~@or~q )     &= clear(p)~;~clear(q)   \\
  clear( \_ )          &= \bot \2 (mem,emitInt,break,if, \\
                                  & \5\5 loop,and,or,@nop) %\\ 
\end{align*}
}
\caption{
The function $clear$ extracts $fin$ expressions in parallel and put their 
bodies in sequence.
\label{fig.formal.clear}
}
\end{figure}

\subsection{Determinism, Termination, and Memory Bounds}

- informal discussion

\subsubsection*{Determinism}

The proof for determinism relies on the fact all semantic rules are mutually
exclusive, i.e., their preconditions are unique in the set of rules.
This can be verified by direct inspection of rules.

Rule \textbf{push} is the only one with $e \neq \epsilon$ as a precondition,
and is trivially mutually exclusive with all other rules.

Rule \textbf{pop} either has $p=@nop$ or $isblocked(n,p)$ as preconditions.
%
Note that rule \textbf{pop} only applies syntactically to top-level
transitions.
For instance, it can never match $\NST$ rules for subprograms as in rule
\textbf{seq-adv}.
%
Hence, for the first case, rule \textbf{pop} only applies, and is the only one
to apply, to $nop$ as the whole program (i.e., a $nop$ not surrounded by other
expressions, such as in rule \textbf{seq-nop}).
%
For the second case, we need to show that given $\LL p,n,\epsilon \RR$, no
$\NST$ transitions apply with $isblocked(n,p)$ and vice versa.
Except for $@canrun$, there are no $\NST$ transitions for the other blocking
expressions ($awaitExt$, $awaitInt$, $every$, and $fin$).
However, considering the precondition $\LL p,n,\epsilon \RR$,
$isblocked(n,@canrun(n))$ is false.
Hence, given the preconditions for rule \textbf{pop}, no $\NST$ transitions can
occur.
Conversely, if a $\NST$ transition is possible, then $isblocked(n,p)$ must be
false.
Again, except for $@canrun$, all other transitions do not involve blocking
expressions, hence, for these transitions, $isblocked(n,p)$ must be false.
For rule \textbf{canrun}, a transition can only occur if the current stack
level matches $@canrun(n)$.
In this case, $isblocked(n,@canrun(n))$ is false.

Finally, we need to show that $\NST$ transitions are mutually exclusive among
themselves.
%
Note that most rules have unique syntactic prefixes, e.g., $(@nop~@and~q)$
(rule \textbf{and-nop1}) is trivially mutually exclusive with $(@nop~@loop~p)$
(rule \textbf{or-nop1}).
%
The only exceptions are rules \textbf{and-adv1} vs. \textbf{and-adv2}, and
\textbf{or-adv1} vs. \textbf{or-adv2}.
In both cases, we need to show that if the left branch can advance, then it
cannot be blocked and vice-versa, i.e., that 
$\LL p,n,\epsilon \RR \NST \LL p',n,e \RR$ and $isblocked(n,p)$ are mutually
exclusive, which is exactly the same reasoning for rule \textbf{pop} above.

\subsubsection*{Termination}

- there is always a possible transition until n=0

every cannot restart itself
    - break disallowed
    - emit ignored

\subsubsection*{Memory Bounds}

- program is finite
- lexical scope
    - no heap allocation
- no code reentrancy
    - reexecution only due to loops
    - loop reuse nested vars
- 

\section{Related Work}

\CEU follows the lineage of imperative synchronous languages initiated by
Esterel~\cite{esterel.ieee91}.
These languages typically define time as a discrete sequence of logical
``ticks'' in which multiple simultaneous input events can be
active~\cite{ceu.tecs17}.
%
The presence of multiple inputs requires careful static analysis to detect and
reject programs with \emph{causality cycles} and \emph{schizophrenia problems}%
~\cite{esterel.constructive}.
%
In contrast, \CEU defines time as a discrete sequence of reactions to
unique input events, which is a prerequisite for the concurrency checks that
enable safe shared-memory concurrency.

Also, in most synchronous languages, the behavior of internal and external
events is equivalent.
In \CEU, internal events introduce stack-based micro reactions within external
reactions, providing more fine-grained control for intra-reaction execution.
%
This allows for memory-bounded subroutines that execute multiple times in the
same external reaction.
%
The synchronous languages Statecharts~\cite{statecharts.variants} and
Statemate~\cite{statecharts.statemate} also distinguish internal from external
events.
In the former, \emph{``reactions to external and internal events (...) can be
sensed only after completion of the step''}.
In the latter, \emph{``the receiving state (of the internal event) acts here as
a function''}.
Although the descriptions suggest a stack-based semantics, we are not aware of
formalizations for these ideas for a deeper comparison with \CEU.

Like \CEU, many synchronous languages%
~\cite{rp.rc,wsn.protothreads,wsn.sol,rp.synchc,rp.pretc}
also rely on lexical scheduling to preserve determinism.
%
In contrast, in Esterel, the execution order for operations within a reaction
is non-deterministic: \emph{``if there is no control dependency, as in
\code{(call~f1()~||~call~f2())},
the order is unspecified and it would be an error to rely on
it''}~\cite{esterel.primer}.
%
For this reason, Esterel, does not support shared-memory concurrency:
\emph{``if a variable is written by some thread, then it can neither be read
nor be written by concurrent threads''}~\cite{esterel.primer}.

Regarding the integration with the $C$ language-based environment, \CEU
supports a finalization mechanism for external resources.
%
In addition, \CEU also tracks pointers representing resources that cross $C$
boundaries and forces the programmer to provide associated finalizers.

\section{Conclusion}
\label{sec.conclusion}

The programming language \CEU aims to offer a concurrent, safe, and realistic
alternative to C for embedded soft real-time systems, such as sensor networks
and multimedia systems.
%
%Its synchronous semantics enables safe shared-memory concurrency and
%safe abortion of lines of execution, while enforcing memory-bounded,
%deterministic, and terminating reactions to the environment.
%
\CEU inherits the synchronous and imperative mindset of Esterel but adopts a
simpler semantics with fine-grained execution control, which makes the
language fully deterministic.
%
In addition, its stack-based execution for internal events provides a limited
but memory-bounded form of subroutines.
%
\CEU also provides a finalization mechanism for resources when interacting with
the external environment.

We propose a small-step structural operational semantics for \CEU and a proof
that reactions are deterministic, terminate in finite time, and use bounded
memory, i.e., that for a given arbitrary timeline of input events, multiple
executions of the same program always react in bounded time and arrive at the
same final finite memory state.


\balance
\bibliographystyle{ACM-Reference-Format}
\bibliography{my,other}

\appendix
\clearpage
\nobalance
\appendix
\section{Proofs}
\label{sec.proofs}

\subsection*{Determinism}


\begin{lemma}\label{lem.det-out}
  If~$\delta\out\delta_1$ and~$\delta\out\delta_2$ then~$\delta_1=\delta_2$.
\end{lemma}
\begin{proof}
  The lemma is vacuously true if~$\delta$ cannot be advanced by~$\out$
  transitions.  Suppose that is not the case and let~$\delta=\<p,n,e>$,
  $\delta_1=\<p_1,n_1,e_1>$ and~$\delta_2=\<p_2,n_2,e_2>$.  Then, there are
  two possibilities.
  \begin{case}
    $e\ne\nil$.  Both transitions are applications of~\R{push}.
    Hence~$p_1=p_2=\bcast(p,e)$, $n_1=n_2=n+1$, and~$e_1=e_2=\nil$.
  \end{case}
  \begin{case}
    $e=\nil$.  Both transitions are applications of~\R{pop}.
    Hence~$p_1=p_2=p$, $n_1=n_2=n-1$, and~$e_1=e_2=\nil$.\qedhere
  \end{case}
\end{proof}


\begin{theorem}\label{thm.det-out-pop-n}
  If~$\delta\out[n]\delta_1$ and~$\delta\out[n]\delta_2$
  then~$\delta_1=\delta_2$.
\end{theorem}
\begin{proof}
  By induction on~$n$.
  %%
  The theorem is trivially true for~$n=0$ and follows directly from
  Lemma~\ref{lem.det-out} for~$n=1$.  (The theorem is vacuously true
  for~$\outpush$ transitions if~$n>1$ since, by definition,
  $\outpush$~transitions cannot be applied more than once in a row and
  cannot occur after a~$\outpop$ transition.)
  %%
  Suppose~$\delta\out[1]\delta_1'\out[n-1]\delta_1$
  and~$\delta\out[1]\delta_2'\out[n-1]\delta_2$, for some~$n>1$
  and~$\delta_1'$, $\delta_2'\in\Delta$.
  %%
  By Lemma~\ref{lem.det-out}, $\delta_1'=\delta_2'$.  By the induction
  hypothesis, $\delta_1=\delta_2$.\qedhere
\end{proof}


\begin{lemma}\label{lem.det-nst}
  If~$\delta\nst\delta_1$ and~$\delta\nst\delta_2$ then~$\delta_1=\delta_2$.
\end{lemma}
\begin{proof}
  By induction on the structure of~$\nst$ derivations.  The lemma is
  vacuously true if~$\delta$ cannot be advanced by~$\nst$ transitions.
  Suppose that is not the case and let~$\delta=\<p,n,e>$,
  $\delta_1=\<p_1,n_1,e_1>$ and~$\delta_2=\<p_2,n_2,e_2>$.  Then, by the
  hypothesis of the lemma, there are derivations~$\pi_1$ and~$\pi_2$ such
  that
  \begin{align*}
    \pi_1&\Vdash\<p,n,e>\nst\<p_1,n_1,e_1>\\
    \pi_2&\Vdash\<p,n,e>\nst\<p_2,n_2,e_2>
  \end{align*}
  i.e., the conclusion of~$\pi_1$ is~$\<p,n,e>\nst\<p_1,n_1,e_1>$ and the
  conclusion of~$\pi_2$ is~$\<p,n,e>\nst\<p_2,n_2,e_2>$.

  By definition of~$\nst$, we have that~$e=\nil$ and $n_1=n_2=n$.  It
  remains to be shown that~$p_1=p_2$ and~$e_1=e_2$.

  Depending on the structure of program~$p$, the following~11 cases are
  possible.  (Note that~$p$ cannot be an~$\ceu{\AwaitExt}$,
  $\ceu{\AwaitInt}$, $\ceu{\Break}$, $\ceu{\Every}$, $\ceu{\Fin}$,
  or~$\ceu{\Nop}$ expression as there are no~$\nst$ rules to transition such
  programs.)

  \begin{case}
    $p=\ceu{\Mem(\Id)}$.
    %%
    Then derivations~$\pi_1$ and~$\pi_2$ are instances of rule~\R{mem},
    i.e., their conclusions are obtained by an application of this rule.
    Hence~$p_1=p_2=\ceu{\Nop}$ and~$e_1=e_2=\nil$.
  \end{case}

  \begin{case}
    $p=\ceu{\EmitInt(e')}$.
    %%
    Then~$\pi_1$ and~$\pi_2$ are instances of~\R{emit-int}.
    Hence~$p_1=p_2=\ceu{\CanRun(n)}$ and~$e_1=e_2=e'$.
  \end{case}

  \begin{case}
    $p=\ceu{\CanRun(n)}$.
    %%
    Then~$\pi_1$ and~$\pi_2$ are instances of~\R{can-run}.
    Hence~$p_1=p_2=\ceu{\Nop}$ and~$e_1=e_2=\nil$.
  \end{case}

  \begin{case}
    $p=\ceu{\IfElse{\Mem(\Id)}{p'}{p''}}$.
    %%
    There are two subcases.
    \begin{subcase}
      $\eval(\ceu{\Mem(\Id)})$~is true.
      %%
      Then~$\pi_1$ and~$\pi_2$ are instances of~\R{if-true}.
      Hence~$p_1=p_2=p'$ and~$e_1=e_2=\nil$.
    \end{subcase}
    \begin{subcase}
      $\eval(\ceu{\Mem(\Id)})$ is false.
      %%
      Then~$\pi_1$ and~$\pi_2$ are instances of~\R{if-false}.
      Hence~$p_1=p_2=p''$ and~$e_1=e_2=\nil$.
    \end{subcase}
  \end{case}

  \begin{case}
    $p=\ceu{p';\,p''}$.
    %%
    There are three subcases.
    \begin{subcase}
      $p'=\ceu{\Nop}$.
      %%
      Then~$\pi_1$ and~$\pi_2$ are instances of~\R{seq-nop}.
      Hence~$p_1=p_2=p''$ and~$e_1=e_2=\nil$.
    \end{subcase}
    \begin{subcase}
      $p'=\ceu{\Break}$.
      %%
      Then~$\pi_1$ and~$\pi_2$ are instances of~\R{seq-brk}.
      Hence~$p_1=p_2=\ceu{\Break}$ and~$e_1=e_2=\nil$.
    \end{subcase}
    \begin{subcase}
      $p'\ne\ceu{\Nop},\ceu{\Break}$.
      %%
      Then~$\pi_1$ and~$\pi_2$ are instances of~\R{seq-adv}.
      Thus there are derivations~$\pi_1'$ and~$\pi_2'$ such that
      \begin{align*}
        \pi_1'&\Vdash\<p',n,\nil>\nst\<p_1',n,e_1'>\\
        \pi_2'&\Vdash\<p',n,\nil>\nst\<p_2',n,e_2'>
      \end{align*}
      for some~$p_1',p_2'\in\P$ and~$e_1',e_2'\in\E$.  By the induction
      hypothesis, $p_1'=p_2'$ and~$e_1'=e_2'$.
      Hence~$p_1=\ceu{p_1';p''}=\ceu{p_2';p''}=p_2$ and~$e_1=e_1'=e_2'=e_2$.
    \end{subcase}
  \end{case}

  \begin{case}
    $p=\ceu{\Loop{p'}}$.
    %%
    Then~$\pi_1$ and~$\pi_2$ are instances of~\R{loop-expd}.
    Hence~$p_1=p_2=\ceu{p'\AtLoop{p'}}$ and~$e_1=e_2=\nil$.
  \end{case}

  \begin{case}
    $p=\ceu{p'\AtLoop{p''}}$.  There are three subcases.
    \begin{subcase}
      $p'=\ceu{\Nop}$.
      %%
      Then~$\pi_1$ and~$\pi_2$ are instances of~\R{loop-nop}.
      Hence~$p_1=p_2=\ceu{\Loop{p''}}$ and~$e_1=e_2=\nil$.
    \end{subcase}
    \begin{subcase}
      $p'=\ceu{\Break}$.
      %%
      Then~$\pi_1$ and~$\pi_2$ are instances of~\R{loop-brk}.
      Hence~$p_1=p_2=\ceu{\Nop}$ and~$e_1=e_2=\nil$.
    \end{subcase}
    \begin{subcase}
      $p'\ne\ceu{\Nop},\ceu{\Break}$.
      %%
      Then~$\pi_1$ and~$\pi_2$ are instances of~\R{loop-adv}.
      Thus there are derivations~$\pi_1'$ and~$\pi_2'$ such that
      \begin{align*}
        \pi_1'&\Vdash\<p',n,\nil>\nst\<p_1',n,e_1'>\\
        \pi_2'&\Vdash\<p',n,\nil>\nst\<p_2',n,e_2'>
      \end{align*}
      for some~$p_1',p_2'\in\P$ and~$e_1',e_2'\in\E$.  By the induction
      hypothesis, $p_1'=p_2'$ and~$e_1'=e_2'$.
      Hence~$p_1=\ceu{p_1'\AtLoop{p''}}=\ceu{p_2'\AtLoop{p''}}=p_2$
      and~$e_1=e_1'=e_2'=e_2$.
    \end{subcase}
  \end{case}

  \begin{case}
    $p=\ceu{p'\And{p''}}$.
    %%
    Then~$\pi_1$ and~$\pi_2$ are instances of~\R{and-expd}.
    Hence~$p_1=p_2=\ceu{{p'}\AtAnd{(\CanRun(n);\,p'')}}$ and~$e_1=e_2=\nil$.
  \end{case}

  \begin{case}
    $p=\ceu{p'\AtAnd{p''}}$.  There are two subcases.
    \begin{subcase}
      $\isblocked(p',n)$~is false.  There are three subcases.
      \begin{subsubcase}
        $p'=\ceu{\Nop}$.
        %%
        Then~$\pi_1$ and~$\pi_2$ are instances of~\R{and-nop1}.
        Hence~$p_1=p_2=p''$ and~$e_1=e_2\nil$.
      \end{subsubcase}
      \begin{subsubcase}
        \label{lem.det-nst.and-brk1}
        $p'=\ceu{\Break}$.
        %%
        Then~$\pi_1$ and~$\pi_2$ are instances of~\R{and-brk1}.
        Hence~$p_1=p_2=\ceu{\clear(p'');\Break}$ and~$e_1=e_2\nil$.
      \end{subsubcase}
      \begin{subsubcase}\label{lem.det-nst.and-adv1}
        $p'\ne\ceu{\Nop},\ceu{\Break}$.
        %%
        Then~$\pi_1$ and~$\pi_2$ are instances of~\R{and-adv1}.
        Thus there are derivations~$\pi_1'$ and~$\pi_2'$ such that
        \begin{align*}
          \pi_1'&\Vdash\<p',n,\nil>\nst\<p_1',n,e_1'>\\
          \pi_2'&\Vdash\<p',n,\nil>\nst\<p_2',n,e_2'>
        \end{align*}
        for some~$p_1',p_2'\in\P$ and~$e_1',e_2'\in\E$.  By the induction
        hypothesis, $p_1'=p_2'$ and~$e_1'=e_2'$.
        Hence~$p_1=\ceu{{p_1'}\And{p''}}=\ceu{{p_2'}\And{p''}}=p_2$
        and~$e_1=e_1'=e_2'=e_2$.
      \end{subsubcase}
    \end{subcase}
    \begin{subcase}
      $\isblocked(p',n)$~is true.  There are three subcases.
      \begin{subsubcase}
        $p''=\ceu{\Nop}$.
        %%
        Then~$\pi_1$ and~$\pi_2$ are instances of~\R{and-nop2}.
        Hence~$p_1=p_2=p'$ and~$e_1=e_2\nil$.
      \end{subsubcase}
      \begin{subsubcase}\label{lem.det-nst.and-brk2}
        $p''=\ceu{\Break}$.
        %%
        Then~$\pi_1$ and~$\pi_2$ are instances of~\R{and-brk2}.
        Hence~$p_1=p_2=\ceu{\clear(p');\Break}$ and~$e_1=e_2=\nil$.
      \end{subsubcase}
      \begin{subsubcase}\label{lem.det-nst.and-adv2}
        $p''\ne\ceu{\Nop},\ceu{\Break}$.
        %%
        Then~$\pi_1$ and~$\pi_2$ are instances of~\R{and-adv2}.  Thus there
        are derivations~$\pi_1''$ and~$\pi_2''$ such that
        \begin{align*}
          \pi_1''&\Vdash\<p'',n,\nil>\nst\<p_1'',n,e_1''>\\
          \pi_2''&\Vdash\<p'',n,\nil>\nst\<p_2'',n,e_2''>
        \end{align*}
        for some~$p_1'',p_2''\in\P$ and~$e_1'',e_2''\in\E$.  By the
        induction hypothesis, $p_1''=p_2''$ and~$e_1''=e_2''$.
        Hence~$p_1=\ceu{{p'}\And{p_1''}}=\ceu{{p'}\And{p_2''}}=p_2$
        and~$e_1=e_1''=e_2''=e_2$.
      \end{subsubcase}
    \end{subcase}
  \end{case}

  \begin{case}
    $p=\ceu{p'\Or{p''}}$.
    %%
    Then~$\pi_1$ and~$\pi_2$ are instances of~\R{or-expd}.
    Hence~$p_1=p_2=\ceu{{p'}\AtOr{(\CanRun(n);\,p'')}}$ and~$e_1=e_2=\nil$.
  \end{case}

  \begin{case}
    $p=\ceu{p'\AtOr{p''}}$.  There are two subcases.
    \begin{subcase}
      $\isblocked(p',n)$~is false.  There are three subcases.
      \begin{subsubcase}
        $p'=\ceu{\Nop}$.
        %%
        Then~$\pi_1$ and~$\pi_2$ are instances of~\R{or-nop1}.
        Hence~$p_1=p_2=\clear(p'')$ and~$e_1=e_2=\nil$.
      \end{subsubcase}
      \begin{subsubcase}
        $p'=\ceu{\Break}$.
        %%
        Similar to Case~\ref{lem.det-nst.and-brk1}.
      \end{subsubcase}
      \begin{subsubcase}
        $p'\ne\ceu{\Nop},\ceu{\Break}$.
        %%
        Similar to Case~\ref{lem.det-nst.and-adv1}.
      \end{subsubcase}
    \end{subcase}
    \begin{subcase}
      $\isblocked(p',n)$~is true.  There are three subcases.
      \begin{subsubcase}
        $p''=\ceu{\Nop}$.
        %%
        Then~$\pi_1$ and~$\pi_2$ are instances of~\R{or-nop1}.
        Hence~$p_1=p_2=\clear(p')$ and~$e_1=e_2=\nil$.
      \end{subsubcase}
      \begin{subsubcase}
        $p''=\ceu{\Break}$.
        %%
        Similar to Case~\ref{lem.det-nst.and-brk2}.
      \end{subsubcase}
      \begin{subsubcase}
        $p''\ne\ceu{\Nop},\ceu{\Break}$.
        %%
        Similar to Case~\ref{lem.det-nst.and-adv2}.
        %%
        \qedhere
      \end{subsubcase}
    \end{subcase}
  \end{case}
\end{proof}


\begin{theorem}\label{thm.det-nst-n}
  If~$\delta\nst[n]\delta_1$ and~$\delta\nst[n]\delta_2$
  then~$\delta_1=\delta_2$.
\end{theorem}
\begin{proof}
  Similar to the proof of Theorem~\ref{thm.det-out-pop-n}.
\end{proof}

\subsection*{Termination}


\begin{definition}\label{def.Hnst}
  A description~$\delta=\<p,n,e>$ is \emph{nested-irre\-ducible}
  iff~$e\ne\nil$ or~$p=\ceu{\Nop},\ceu{\Break}$ or~$\isblocked(p,n)$~is
  true.  Nested-irreducible descriptions serve as normal forms for~$\nst$
  transitions: they embody the result of an exhaustive number of~$\nst$
  applications.  We will write~$\delta_\Hnst$ to indicate that
  description~$\delta$ is nested-irreducible.
\end{definition}

The next lemma justifies the use of qualifier ``irreducible'' in
Definition~\ref{def.Hnst}.


\begin{lemma}\label{lem.irr-nst-n}
  If~$\delta\nst[n]\delta_\Hnst'$ then, for all~$i\ne{n}$, there is
  no~$\delta_\Hnst''$ such that~$\delta\nst[i]\delta''_\Hnst$.
\end{lemma}
\begin{proof}
  By contradiction on the hypothesis that there is such~$i$.
  %%
  Let~$\delta\nst[n]\delta'_\Hnst$, for some~$n\ge0$.
  There are two cases.
  \begin{case}\label{lem.irr-nst-n-case1}
    Suppose there are~$i>n$ and~$\delta''_\Hnst$ such
    that~$\delta\nst[i]\delta''$.
    %%
    Then, by definition of~$\nst[i]$,
    \begin{equation}\label{lem.irr-nst-n-eq1}
      \delta\nst[n]\delta'\nst[n+1]\delta_1'\nst[n+2]\cdots\nst[i]\delta''.
    \end{equation}
    Since~$\delta'=\<p',n,e'>$ is nested-irreducible, $e'=\nil$
    or~$p=\ceu{\Nop},\ceu{\Break}$ or~$\isblocked(p',n)$.  In any of these
    cases, by the definition of~$\nst$, there is no~$\delta_1'$ such
    that~$\delta'\nst[1]\delta_1'$, which
    contradicts~\eqref{lem.irr-nst-n-eq1}.  Therefore, no such~$i$ can
    exist.
  \end{case}
  \begin{case}
    Suppose there are~$i<n$ and~$\delta''_\Hnst$ such
    that~$\delta\nst[i]\delta''$.  Then, since~$n>i$, by
    Case~\ref{lem.irr-nst-n-case1}, $\delta'$~cannot exist, which is absurd.
    Therefore, the assumption that there is such~$i$ is false.\qedhere
  \end{case}
\end{proof}

The next lemma establishes some basic properties of sequences of~$\nst$
transitions.


\begin{lemma}\label{lem.props-nst-n}
  If~$\<p_1,n,e>\nst[n]\<p_1',n,e'>$ then, for any~$p_2$:
  \begin{enumerate}[(a)]
  \item\label{lem.props-nst-n.a}
    $\<\ceu{p_1;\,p_2},n,e>\nst[n]\<p_1';p_2,n,e'>$;
    %%
  \item\label{lem.props-nst-n.b}
    $\<\ceu{p_1\AtLoop{p_2}},n,e>\nst[n]\<\ceu{p_1'\AtLoop{p_2}},n,e'>$;
    %%
  \item\label{lem.props-nst-n.c}
    $\<\ceu{{p_1}\AtAnd{p_2}},n,e>\nst[n]\<\ceu{{p_1'}\AtAnd{p_2}},n,e'>$;
    %%
  \item\label{lem.props-nst-n.d}
    $\<\ceu{{p_1}\AtOr{p_2}},n,e>\nst[n]\<\ceu{{p_1}'\AtOr{p_2}},n,e'>$.
  \end{enumerate}
  If~$\<p_2,n,e>\nst[n]\<p_2',n,e'>$, for any~$p_1$ such
  that~$\isblocked(p_1,n)$:
  \begin{enumerate}[(a)]
    \setcounter{enumi}{4}
  \item\label{lem.props-nst-n.e}
    $\<\ceu{{p_1}\AtAnd{p_2}},n,e>\nst[n]\<\ceu{{p_1}\AtAnd{p_2'}},n,e'>$;
    %%
  \item\label{lem.props-nst-n.f}
    $\<\ceu{{p_1}\AtOr{p_2}},n,e>\nst[n]\<\ceu{{p_1}\AtOr{p_2'}},n,e'>$.
  \end{enumerate}
\end{lemma}
\begin{proof}
  By induction on~$n$.
  %%
  \begin{enumerate}[(a)]
  \item The lemma is trivially true for~$n=0$, as~$p_1=p_1'$, and follows
    directly from~\R{seq-adv} for~$n=1$.  Suppose
    \begin{equation}
      \label{lem.props-nst-n.a.eq1}
      \<p_1,n,e>\nst[1]\<p_1'',n,e''>\nst[n-1]\<p_1',n,e'>\,,
    \end{equation}
    for some~$n>1$.  Then~$\<p_1'',n,e''>$ is not nested-irreducible, i.e.,
    $e=\nil$ and~$p\ne{\ceu{\Nop},\ceu{\Break}}$ and~$\isblocked(p_1'',n)$
    is false.  By~\eqref{lem.props-nst-n.a.eq1} and by~\R{seq-adv},
    \begin{equation}
      \label{lem.props-nst-n.a.eq2}
      \<\ceu{p_1;\,p_2},n,e>\nst[1]\<\ceu{p_1'';\,p_2},n,e''>\,.
    \end{equation}
    From~\eqref{lem.props-nst-n.a.eq1}, by the induction hypothesis,
    \begin{equation}
      \label{lem.props-nst-n.a.eq3}
      \<\ceu{p_1'';\,p_2},n,e''>\nst[n-1]\<\ceu{p_1';\,p_2},n,e'>\,.
    \end{equation}
    From~\eqref{lem.props-nst-n.a.eq2} and~\eqref{lem.props-nst-n.a.eq3},
    \[
      \<\ceu{p_1;\,p_2},n,e>\nst[n]\<\ceu{p_1';\,p_2},n,e'>\,.
    \]

  \item Similar to Case~\eqref{lem.props-nst-n.a}.
    %%
    % The lemma is trivially true for~$n=0$, as~$p_1=p_1'$, and follows
    % directly from~\R{loop-adv} for~$n=1$.  Suppose
    % \begin{equation}
    %   \label{lem.props-nst-n.b.eq1}
    %   \<p_1,n,e>\nst[1]\<p_1'',n,e''>\nst[n-1]\<p_1',n,e'>\,,
    % \end{equation}
    % for some~$n>1$.  Then~$\<p_1'',n,e''>$ is not nested-irreducible.
    % By~\eqref{lem.props-nst-n.b.eq1} and by~\R{loop-adv},
    % \begin{equation}
    %   \label{lem.props-nst-n.b.eq2}
    %   \<\ceu{p_1\AtLoop{p_2}},n,e>\nst[1]\<\ceu{p_1''\Loop{p_2}},n,e''>\,.
    % \end{equation}
    % From~\eqref{lem.props-nst-n.b.eq1}, by the induction hypothesis,
    % \begin{equation}
    %   \label{lem.props-nst-n.b.eq3}
    %   \<\ceu{p_1''\AtLoop{p_2}},n,e''>
    %   \nst[n-1]\<\ceu{p_1'\AtLoop{p_2}},n,e'>\,.
    % \end{equation}
    % From~\eqref{lem.props-nst-n.b.eq2} and~\eqref{lem.props-nst-n.b.eq3},
    % \[
    %   \<\ceu{p_1\AtLoop{p_2}},n,e>\nst[n]\<\ceu{p_1'\AtLoop{p_2}},n,e'>\,.
    % \]
    %%

  \item Similar to Case~\eqref{lem.props-nst-n.a}.
    %%
    % The lemma is trivially true for~$n=0$, as~$p_1=p_1'$, and follows
    % directly from~\R{and-adv1} for~$n=1$.  Suppose
    % \begin{equation}
    %   \label{lem.props-nst-n.c.eq1}
    %   \<p_1,n,e>\nst[1]\<p_1'',n,e''>\nst[n-1]\<p_1',n,e'>\,,
    % \end{equation}
    % for some~$n>1$.  Then~$\<p_1'',n,e''>$ is not nested-irreducible.
    % By~\eqref{lem.props-nst-n.c.eq1} and by~\R{and-adv1},
    % \begin{equation}
    %   \label{lem.props-nst-n.c.eq2}
    %   \<\ceu{{p_1}\AtAnd{p_2}},n,e>
    %   \nst[1]\<\ceu{{p_1}''\AtAnd{p_2}},n,e''>\,.
    % \end{equation}
    % From~\eqref{lem.props-nst-n.c.eq1}, by the induction hypothesis,
    % \begin{equation}
    %   \label{lem.props-nst-n.a.eq3}
    %   \<\ceu{{p_1''}\AtAnd{p_2}},n,e''>
    %   \nst[n-1]\<\ceu{{p_1'}\AtAnd{p_2}},n,e'>\,.
    % \end{equation}
    % From~\eqref{lem.props-nst-n.c.eq2} and~\eqref{lem.props-nst-n.c.eq3},
    % \[
    %   \<\ceu{{p_1}\AtAnd{p_2}},n,e>
    %   \nst[n]\<\ceu{{p_1'}\AtAnd{p_2}},n,e'>\,.
    % \]
    %%

  \item Similar to Case~\eqref{lem.props-nst-n.a}.
    %%
    % The lemma is trivially true for~$n=0$, as~$p_1=p_1'$, and follows
    % directly from~\R{or-adv1} for~$n=1$.  Suppose
    % \begin{equation}
    %   \label{lem.props-nst-n.d.eq1}
    %   \<p_1,n,e>\nst[1]\<p_1'',n,e''>\nst[n-1]\<p_1',n,e'>\,,
    % \end{equation}
    % for some~$n>1$.  Then~$\<p_1'',n,e''>$ is not nested-irreducible.
    % By~\eqref{lem.props-nst-n.d.eq1} and by~\R{or-adv1},
    % \begin{equation}
    %   \label{lem.props-nst-n.d.eq2}
    %   \<\ceu{{p_1}\AtOr{p_2}},n,e>
    %   \nst[1]\<\ceu{{p_1}''\AtOr{p_2}},n,e''>\,.
    % \end{equation}
    % From~\eqref{lem.props-nst-n.d.eq1}, by the induction hypothesis,
    % \begin{equation}
    %   \label{lem.props-nst-n.a.eq3}
    %   \<\ceu{{p_1''}\AtOr{p_2}},n,e''>
    %   \nst[n-1]\<\ceu{{p_1'}\AtOr{p_2}},n,e'>\,.
    % \end{equation}
    % From~\eqref{lem.props-nst-n.d.eq2} and~\eqref{lem.props-nst-n.d.eq3},
    % \[
    %   \<\ceu{{p_1}\AtOr{p_2}},n,e>
    %   \nst[n]\<\ceu{{p_1'}\AtOr{p_2}},n,e'>\,.
    % \]
    %%

  \item The lemma is trivially true for~$n=0$, as~$p_2=p_2'$, and follows
    directly from~\R{and-adv2} for~$n=1$.  Suppose
    \begin{equation}
      \label{lem.props-nst-n.e.eq1}
      \<p_2,n,e>\nst[1]\<p_2'',n,e''>\nst[n-1]\<p_2',n,e'>\,,
    \end{equation}
    for some~$n>1$.  Then~$\<p_2'',n,e''>$ is not nested-irreducible.
    By~\eqref{lem.props-nst-n.e.eq1} and by~\R{and-adv2},
    \begin{equation}
      \label{lem.props-nst-n.e.eq2}
      \<\ceu{{p_1}\AtAnd{p_2}},n,e>
      \nst[1]\<\ceu{{p_1}\AtAnd{p_2''}},n,e''>\,.
    \end{equation}
    From~\eqref{lem.props-nst-n.e.eq1}, by the induction hypothesis,
    \begin{equation}
      \label{lem.props-nst-n.e.eq3}
      \<\ceu{{p_1}\AtAnd{p_2''}},n,e''>
      \nst[n-1]\<\ceu{{p_1}\AtOr{p_2'}},n,e'>\,.
    \end{equation}
    From~\eqref{lem.props-nst-n.e.eq2} and~\eqref{lem.props-nst-n.e.eq3},
    \[
      \<\ceu{{p_1}\AtAnd{p_2}},n,e>
      \nst[n]\<\ceu{{p_1}\AtAnd{p_2'}},n,e'>\,.
    \]

  \item Similar to Case~\eqref{lem.props-nst-n.e}.\qedhere
  \end{enumerate}
\end{proof}

The syntactic restrictions discussed in Section~\ref{?}, regarding the body
of~$\ceu{\Fin}$ and~$\ceu{\Loop}$ expressions, are formalized by the
following assumptions.

\begin{assumption}\label{ass.term-nst-fin}
  If~$\delta=\<\ceu{\clear(p)},n,\nil>$ then there is a
  description~$\delta'=\<\ceu{\Nop},n,\nil>$ such
  that~$\delta\nst[*]\delta'$.
\end{assumption}

\begin{assumption}\label{ass.term-nst-loop}
  If~$\delta=\<\ceu{\Loop{p}},n,\nil>$ then there is a
  description~$\delta'=\<p',n,e>$ such that~$\delta\nst[*]\delta'$ where
  either~$p'=\ceu{{\Break}\AtLoop{p}}$ or~$\isblockedext(p',n)$.
\end{assumption}


\begin{theorem}\label{thm.term-nst-*}
  For any~$\delta$ there is a~$\delta'_\Hnst$ such
  that~$\delta\nst[*]\delta'_\Hnst$.
\end{theorem}
\begin{proof}
  By induction on the structure of programs.
  %%
  Let~$\delta=\<p,n,\nil>$.  The theorem is trivially true if~$\delta$ is
  nested-irreducible, as by definition~$\delta\nst[0]\delta$.  Suppose that
  is not the case.  Then, depending on the structure of~$p$, there are~11
  possibilities.  In each one of them, we show that such~$\delta'_\Hnst$
  indeed exists.
  \begin{case}
    $p=\ceu{\Mem(\Id)}$.
    %%
    Then, by~\R{mem},
    \[
      \<\ceu{\Mem(\Id)},n,\nil>\nst[1]\<\ceu{\Nop},n,\nil>_\Hnst\,.
    \]
  \end{case}

  \begin{case}
    $p=\ceu{\EmitInt(e)}$.
    %%
    Then, by~\R{emit-int},
    \[
      \<\ceu{\EmitInt(e)},n,\nil>\nst[1]\<\ceu{\CanRun(n)},n,e>_\Hnst\,.
    \]
  \end{case}

  \begin{case}
    $p=\ceu{\CanRun(n)}$.
    %%
    Then, by~\R{can-run},
    \[
      \<\ceu{\CanRun(n)},n,\nil>\nst[1]\<\ceu{\Nop},n,\nil>_\Hnst\,.
    \]
  \end{case}

  \begin{case}
    $p=\ceu{\IfElse{\Mem(\Id)}{p'}{p''}}$.
    %%
    There are two subcases.
    \begin{subcase}
      $\eval(\ceu{\Mem(\Id)})$~is true.
      %%
      Then, by~\R{if-true} and by the induction hypothesis, there is
      a~$\delta'$ such that
      \begin{align*}
        \<\ceu{\IfElse{\Mem(\Id)}{p'}{p''}},n,\nil>
        &\nst[1]\<p',n,e>\\
        &\nst[*]\delta'_\Hnst\,.
      \end{align*}
    \end{subcase}
    \begin{subcase}
      $\eval(\ceu{\Mem(\Id)})$~is false.
      %%
      Then, by~\R{if-false} and by the induction hypothesis, there is
      a~$\delta'$ such that
      \begin{align*}
        \<\ceu{\IfElse{\Mem(\Id)}{p'}{p''}},n,\nil>
        &\nst[1]\<p'',n,e>\\
        &\nst[*]\delta'_\Hnst\,.
      \end{align*}
    \end{subcase}
  \end{case}

  \begin{case}
    $p=\ceu{p';\,p''}$.
    %%
    There are three subcases.
    \begin{subcase}
      \label{thm.term-nst-*.seq-nop}
      $p'=\ceu{\Nop}$.
      %%
      Then, by~\R{seq-nop} and by the induction hypothesis, there is
      a~$\delta'$ such that
      \[
        \<\ceu{\Nop;\,p''},n,\nil>
        \nst[1]\<p'',n,e>\nst[*]\delta'_\Hnst\,.
      \]
    \end{subcase}
    \begin{subcase}
      \label{thm.term-nst-*.seq-brk}
      $p'=\ceu{\Break}$.
      %%
      Then, by~\R{seq-brk},
      \[
        \<\ceu{\Break;\,p''},n,\nil>\nst[1]\<\ceu{\Break},n,\nil>_\Hnst\,.
      \]
    \end{subcase}
    \begin{subcase}
      \label{thm.term-nst-*.seq-adv}
      $p'\ne\ceu{\Nop},\ceu{\Break}$.
      %%
      Then, by the induction hypothesis, there are~$p_1'$ and~$e$ such that
      \[
        \<p',n,\nil>\nst[*]\<p_1',n,e>_\Hnst\,.
      \]
      By item~\eqref{lem.props-nst-n.a} of Lemma~\ref{lem.props-nst-n},
      \begin{equation}
        \label{thm.term-nst-*.seq-adv.eq1}
        \<\ceu{p';\,p''},n,\nil>\nst[*]\<\ceu{p_1';\,p''},n,e>\,.
      \end{equation}
      It remains to be shown that~$\<\ceu{p_1';\,p''},n,e>$ is
      nested-irreducible.  There are four possibilities following from the
      fact that the simpler~$\<p_1',n,e>$ is nested-irreducible.
      %%
      \begin{subsubcase}
        $e\ne\nil$.  Then, by the definition of~$\Hnst$,
        description~$\<\ceu{p_1';\,p''},n,e>$ is nested-irreducible.
      \end{subsubcase}
      \begin{subsubcase}
        $p_1'=\ceu{\Nop}$.
        %%
        From~\eqref{thm.term-nst-*.seq-adv.eq1},
        \[
          \<\ceu{p';\,p''},n,\nil>\nst[*]\<\ceu{\Nop;\,p''},n,e>\,.
        \]
        From this point on, this case is similar to
        Case~\ref{thm.term-nst-*.seq-nop}.
      \end{subsubcase}
      \begin{subsubcase}
        $p_1'=\ceu{\Break}$.
        %%
        From~\eqref{thm.term-nst-*.seq-adv.eq1},
        \[
          \<\ceu{p';\,p''},n,\nil>\nst[*]\<\ceu{\Break;\,p''},n,e>\,.
        \]
        From this point on, this case is similar to
        Case~\ref{thm.term-nst-*.seq-brk}.
      \end{subsubcase}
      \begin{subsubcase}
        $\isblocked(p_1',n)$~is true.
        %%
        Then, by definition,
        \[
          \isblocked(\ceu{p_1';p''},n)=\isblocked(p_1',n)=\mathit{true}\,.
        \]
        Hence, from~\eqref{thm.term-nst-*.seq-adv.eq1} and by the
        definition~$\Hnst$, description~$\<\ceu{p_1';\,p''},n,e>$ is
        nested-irreducible.
      \end{subsubcase}
    \end{subcase}
  \end{case}

  \begin{case}
    \label{thm.term-nst-*.loop}
    $p=\ceu{\Loop{p'}}$.
    %%
    Then, by Assumption~\ref{ass.term-nst-loop},
    \begin{equation}\label{thm.term-nst-*.loop-expd.eq1}
      \<\ceu{\Loop{p'}},n,\nil>\nst[*]\<p_1',n,e>\,,
    \end{equation}
    for some~$e$ and~$p_1'$ such that either~$p_1'=\ceu{\Break\AtLoop{p'}}$
    or~$\isblockedext(p_1',n)$.
    \begin{subcase}
      $p_1'=\ceu{\Break\AtLoop{p'}}$.
      %%
      From~\eqref{thm.term-nst-*.loop-expd.eq1}, by~\R{loop-brk},
      \begin{align*}
        \<\ceu{\Loop{p'}},n,\nil>
        &\nst[*]\<\ceu{\Break\AtLoop{p'}},n,e>\\
        &\nst[1]\<\ceu{\Nop},n,e>_\Hnst\,.
      \end{align*}
    \end{subcase}
    \begin{subcase}
      $\isblockedext(p_1',n)$ is true.  Then, by definition,
      $\isblockedext(p_1',n)$ implies~$\isblocked(p_1',n)$.  Hence,
      from~\eqref{thm.term-nst-*.loop-expd.eq1} and by the definition
      of~$\Hnst$, $\<p_1',n,e>_\Hnst$.
    \end{subcase}
  \end{case}

  \begin{case}
    $p=\ceu{p'\AtLoop{p''}}$.  There are three subcases.
    \begin{subcase}
      $p'=\ceu{\Nop}$.
      %%
      Then, by~\R{loop-nop},
      \[
        \<\ceu{\Nop\AtLoop{p''}},n,\nil>
        \nst[1]\<\ceu{\Loop{p''}},n,\nil>\,.
      \]
      From this point on, this case is similar to
      Case~\ref{thm.term-nst-*.loop}.
    \end{subcase}
    \begin{subcase}
      $p'=\ceu{\Break}$.  Then, by~\R{loop-brk},
      \[
        \<\ceu{\Break\AtLoop{p''}},n,\nil>
        \nst[1]\<\ceu{\Nop},n,\nil>_\Hnst\,.
      \]
    \end{subcase}
    \begin{subcase}
      $p'\ne\ceu{\Nop},\ceu{\Break}$.  Then, by the induction hypothesis,
      there are~$p_1'$ and~$e$ such that
      \[
        \<p',n,\nil>\nst[*]\<p'_1,n,e>_\Hnst\,.
      \]
      By item~\eqref{lem.props-nst-n.b} of Lemma~\ref{lem.props-nst-n},
      \[
        \<\ceu{p'\AtLoop{p''}},n,\nil>
        \nst[*]\<\ceu{p_1'\AtLoop{p''}},n,e>\,.
      \]
      It remains to be show that~$\<\ceu{p_1'\AtLoop{p''}},n,e>$ is
      nested-irreducible.  The rest of this proof is similar to that of
      Case~\ref{thm.term-nst-*.seq-adv}.
    \end{subcase}
  \end{case}

  \begin{case}
    $p=\ceu{{p'}\And{p''}}$.
    %%
    Then, by~\R{and-expd},
    \[
      \<\ceu{{p'}\And{p''}},n,\nil>
      \nst[1]\<\ceu{{p'}\AtAnd{(\CanRun(n);\,p'')}},n,\nil>\,.
    \]
    From this point on, this case is similar to
    Case~\ref{thm.term-nst-*.and}.
  \end{case}

  \begin{case}\label{thm.term-nst-*.and}
    $p=\ceu{{p'}\AtAnd{p''}}$.
    %%
    There are two subcases.
    \begin{subcase}
      $\isblocked(p',n)$ is false.
      %%
      There are three subcases.
      \begin{subsubcase}
        \label{thm.term-nst-*.and.nop1}
        $p'=\ceu{\Nop}$.
        %%
        Then, by~\R{and-nop1} and by the induction hypothesis, there
        is a~$\delta'$ such that
        \[
          \<\ceu{{\Nop}\AtAnd{p''}},n,\nil>
          \nst[1]\<p'',n,\nil>
          \nst[*]\delta'_\Hnst\,.
        \]
      \end{subsubcase}
      \begin{subsubcase}
        \label{thm.term-nst-*.and.brk1}
        $p'=\ceu{\Break}$.
        %%
        Then, by~\R{and-brk1},
        \begin{equation}
          \label{thm.term-nst-*.and.brk1.eq1}
          \<\ceu{{\Break}\AtAnd{p''}},n,\nil>
          \nst[1]\<\ceu{\clear(p'');\,\Break},n,\nil>\,.
        \end{equation}
        From~\eqref{thm.term-nst-*.and.brk1.eq1}, by
        Assumption~\ref{ass.term-nst-fin} and~\R{seq-nop},
        \begin{align*}
          \<\ceu{\clear(p'');\,\Break},n,\nil>
          &\nst[*]\<\ceu{\Nop;\,\Break},n,\nil>\\
          &\nst[1]\<\ceu{\Break},n,\nil>_\Hnst\,.
        \end{align*}
      \end{subsubcase}
      \begin{subsubcase}
        \label{thm.term-nst-*.and.adv1}
        $p'\ne\ceu{\Nop},\ceu{\Break}$.
        %%
        Then, by the induction hypothesis, there are~$p_1'$ and~$e$ such
        that
        \[
          \<p',n,\nil>\nst[*]\<p_1',n,e>_\Hnst\,.
        \]
        By item~\eqref{lem.props-nst-n.c} of Lemma~\ref{lem.props-nst-n},
        \[
          \<\ceu{{p'}\AtAnd{p''}},n,\nil>
          \nst[*]\<\ceu{{p_1'}\AtAnd{p''}},n,e>\,.
        \]
        It remains to be show that~$\<\ceu{{p_1'}\AtAnd{p''}},n,e>$ leads to
        an nested-irreducible description.  There are four possibilities
        following from the fact that the simpler~$\<p_1',n,e>$ is
        nested-irreducible.
        \begin{enumerate}
        \item If~$e\ne\nil$ then, by
          definition,~$\<\ceu{{p_1'}\AtAnd{p''}},n,e>_\Hnst$.
        \item If~$p_1'=\ceu{\Nop}$, this case is similar to
          Case~\ref{thm.term-nst-*.and.nop1}.
        \item If~$p_1'=\ceu{\Break}$, this case is similar to
          Case~\ref{thm.term-nst-*.and.brk1}.
        \item If~$\isblocked(p_1',n)$, this case is similar to
          Case~\ref{thm.term-nst-*.and2}.
        \end{enumerate}
      \end{subsubcase}
    \end{subcase}
    \begin{subcase}
      \label{thm.term-nst-*.and2}
      $\isblocked(p',n)$ is true.
      %%
      There are three subcases.
      \begin{subsubcase}
        \label{thm.term-nst-*.and.nop2}
        $p''=\ceu{\Nop}$.
        %%
        Then, by~\R{and-nop2},
        \[
          \<\ceu{{p'}\AtAnd{\Nop}},n,\nil>\nst[1]\<p',n,\nil>_\Hnst\,.
        \]
      \end{subsubcase}
      \begin{subsubcase}
        \label{thm.term-nst-*.and.brk2}
        $p''=\ceu{\Break}$.
        %%
        Then, by~\R{and-brk2},
        \[
          \<\ceu{{p'}\AtAnd{\Break}},n,\nil>
          \nst[1]\<\ceu{\clear(p');\,\Break},n,\nil>\,.
        \]
        From this point on, this case is similar to
        Case~\ref{thm.term-nst-*.and.brk1}.
      \end{subsubcase}
      \begin{subsubcase}
        \label{thm.term-nst-*.and.adv2}
        $p''\ne\ceu{\Nop},\ceu{\Break}$.
        %%
        Then, by the induction hypothesis, there are~$p_1''$ and~$e$ such
        that
        \[
          \<p'',n,\nil>\nst[*]\<p_1'',n,e>_\Hnst\,.
        \]
        By item~\eqref{lem.props-nst-n.e} of Lemma~\ref{lem.props-nst-n},
        \[
          \<\ceu{{p'}\AtAnd{p''}},n,\nil>
          \nst[*]\<\ceu{{p'}\AtAnd{p_1''}},n,e>\,.
        \]
        It remains to be show that~$\<\ceu{{p'}\AtAnd{p_1''}},n,e>$ leads to
        an nested-irreducible description.  There are four possibilities
        following from the fact that the simpler~$\<p_1'',n,e>$ is
        nested-irreducible.
        \begin{enumerate}
        \item If~$e\ne\nil$ then, by definition,
          $\<\ceu{{p'}\AtAnd{p_1''}},n,e>_\Hnst$.
        \item If~$p_1''=\ceu{\Nop}$, this case is similar to
          Case~\ref{thm.term-nst-*.and.nop2}.
        \item If~$p_1''=\ceu{\Break}$, this case is similar to
          Case~\ref{thm.term-nst-*.and.brk2}.
        \item If~$\isblocked(p_1'',n)$ then, as both sides are blocked, by
          definition, $\<\ceu{{p'}\AtAnd{p_1''}},n,e>_\Hnst$.
        \end{enumerate}
      \end{subsubcase}
    \end{subcase}
  \end{case}

  \begin{case}
    $p=\ceu{{p'}\Or{p''}}$.
    %%
    Then, by~\R{or-expd},
    \[
      \<\ceu{{p'}\Or{p''}},n,\nil>
      \nst[1]\<\ceu{{p'}\AtOr{(\CanRun(n);\,p'')}},n,\nil>\,.
    \]
    From this point on, this case is similar to
    Case~\ref{thm.term-nst-*.or}.
  \end{case}

  \begin{case}
    \label{thm.term-nst-*.or}
    $p=\ceu{{p'}\AtOr{p''}}$.
    %%
    There are two subcases.
    \begin{subcase}
      $\isblocked(p',n)$~is false.
      %%
      There are three subcases.
      \begin{subsubcase}
        \label{thm.term-nst-*.or.nop1}
        $p'=\ceu{\Nop}$.  Then, by~\R{or-nop1},
        \begin{equation}
          \label{thm.term-nst-*.or.nop1.eq1}
          \<\ceu{{\Nop}\AtOr{p''}},n,\nil>
          \nst[1]\<\ceu{\clear(p'')},n,\nil>\,.
        \end{equation}
        From~\eqref{thm.term-nst-*.or.nop1.eq1}, by
        Assumption~\ref{ass.term-nst-fin},
        \[
          \<\ceu{\clear(p'')},n,\nil>\nst[*]\<\ceu{\Nop},n,\nil>_\Hnst\,.
        \]
      \end{subsubcase}
      \begin{subsubcase}
        \label{thm.term-nst-*.or.brk1}
        $p'=\ceu{\Break}$.
        %%
        Similar to Case~\ref{thm.term-nst-*.and.brk1}.
      \end{subsubcase}
      \begin{subsubcase}
        $p'\ne\ceu{\Nop},\ceu{\Break}$.
        %%
        Similar to Case~\ref{thm.term-nst-*.and.adv1}.
      \end{subsubcase}
    \end{subcase}
    \begin{subcase}
      \label{thm.term-nst-*.or.adv1}
      $\isblocked(p',n)$~is true.
      %%
      There are three subcases.
      \begin{subsubcase}
        $p''=\ceu{\Nop}$.
        %%
        Then, by~\R{or-nop2},
        \begin{equation}
          \label{thm.term-nst-*.or.nop2.eq1}
          \<\ceu{p'\AtOr{\Nop}},n,\nil>
          \nst[1]\<\clear(p'),n,\nil>\,.
        \end{equation}
        From~\eqref{thm.term-nst-*.or.nop2.eq1}, by
        Assumption~\ref{ass.term-nst-fin},
        \[
          \<\ceu{\clear(p')},n,\nil>\nst[*]\<\ceu{\Nop},n,\nil>_\Hnst\,.
        \]
      \end{subsubcase}
      \begin{subsubcase}
        $p''=\ceu{\Break}$.
        %%
        Similar to Case~\ref{thm.term-nst-*.and.brk2}.
      \end{subsubcase}
      \begin{subsubcase}
        $p''\ne\ceu{\Nop},\ceu{\Break}$.
        %%
        Similar to Case~\ref{thm.term-nst-*.and.adv2}.\qedhere
      \end{subsubcase}
    \end{subcase}
  \end{case}
\end{proof}


\begin{definition}
  \label{def.pot}
  %%
  The potency of a program~$p$ in reaction to event~$e$,
  denoted~$\pot(p,e)$, is the maximum number of~$\ceu{\EmitInt}$ expressions
  that can be executed during a reaction of~$p$ to~$e$.

  More formally,
  \[
    \pot(p,e)=\pot'(\bcast(p,e))\,,
  \]
  where~$\pot'$ is an auxiliary function that counts the number of
  reachable~$\ceu{\EmitInt}$ in the program resulting from the broadcast of
  event~$e$ to~$p$.

  The auxiliary function~$\pot'$ is defined by the following clauses:
  \begin{enumerate}[(a)]
    \item$\pot'(\ceu{\EmitInt}(e))=1$;
    \item$\pot'(\ceu{\IfElse{\Mem(\Id)}{p_1}{p_2}})
      =\max\{\pot'(p_1),\pot'(p_2)\}$;
    \item$\pot'(\ceu{\Loop{p_1}})=\pot'(p_1)$;
    \item If~$p_1\ne\ceu{\Break},\ceu{\AwaitExt(e)}$,
      \begin{align*}
        \pot'(\ceu{p_1;\,p_2})&=\pot'(p_1)+\pot'(p_2)\\
        %%
        \pot'(\ceu{p_1\AtLoop{p_2}})&=
        \begin{cases}
          \pot'(p_1)              &\text{if~(\dag)}\\
          \pot'(p_1)+\pot'(p_2)   &\text{otherwise}\\
        \end{cases}\\
        %%
        \pot'(\ceu{{p_1}\And{p_2}})&=\pot'(p_1)+\pot'(p_2)\\
        %%
        \pot'(\ceu{{p_1}\AtAnd{p_2}})&=\pot'(p_1)+\pot'(p_2)\\
        %%
        \pot'(\ceu{{p_1}\Or{p_2}})&=\pot'(p_1)+\pot'(p_2)\\
        %%
        \pot'(\ceu{{p_1}\AtOr{p_2}})&=\pot'(p_1)+\pot'(p_2)\,,
      \end{align*}
      where~(\dag) stands for: ``a~$\ceu{\Break}$ or~$\ceu{\AwaitExt}$
      occurs in all execution paths of~$p_1$'';
      \item Otherwise, if none of~(a)--(d) applies, $\pot(\ast)=0$.
    \end{enumerate}

  TODO: Define~$\pot'(\ceu{\Every{e}\,{p_1}})$ above.
\end{definition}


\begin{definition}\label{def.rank}
  The \emph{rank} of a description~$\delta=\<p,n,e>$,
  denoted~$\rank(\delta)$, is a pair of nonnegative integers~$\<i,j>$ such
  that
  \begin{alignat*}{2}
    i&=\pot(p,e) &\quad\text{and}\quad
    j&=
       \begin{cases}
         n  &\text{if~$e=\nil$}\\
         n+1&\text{otherwise\,.}
       \end{cases}
  \end{alignat*}
\end{definition}

The next two lemmas establish that a single application of an~$\out$
or~$\nst$ transition either preserves or decreases the rank of the input
description.  All rank comparisons assume lexicographical order.


\begin{lemma}\strut
  \begin{enumerate}[(a)]
  \item If~$\delta\outpush\delta'$ then~$\rank(\delta)=\rank(\delta')$.
  \item If~$\delta\outpop\delta'$ then~$\rank(\delta)>\rank(\delta')$.
  \end{enumerate}
\end{lemma}
\begin{proof}\strut
  Let~$\delta=\<p,n,e>$, $\delta'=\<p',n',e'>$,
  $\rank(\delta)=\<i,j>$, and~$\rank(\delta')=\<i',j'>$.
  \begin{enumerate}[(a)]
  \item Suppose~$\<p,n,e>\outpush\<p',n',e'>$.  Then, by~\R{push},
    $e\ne\nil$, $p'=\bcast(p,e)$, $n'=n+1$, and~$e'=\nil$.  By
    Definition~\ref{def.rank}, $j=n+1$, as~$e\ne\nil$, and~$j'=n+1$,
    as~$e'=\nil$ and~$n'=n+1$; hence~$j=j'$.
    %%
    It remains to be shown that~$i=i'$:
    \begin{align*}
      i&=\pot(p,e)
         \tag*{by Definition~\ref{def.rank}}\\
       &=\pot'(\bcast(p,e))
         \tag*{by Definition~\ref{def.pot}}\\
       &=\pot'(p')
         \tag*{since~$p'=\bcast(p,e)$}\\
       &=\pot'(\bcast(p',\nil))
         \tag*{by definition of~$\bcast$}\\
       &=\pot'(\bcast(p',e'))
         \tag*{since~$e'=\nil$}\\
       &=\pot(p',e')
         \tag*{by Definition~\ref{def.pot}}\\
       &=i'
         \tag*{by Definition~\ref{def.rank}}
    \end{align*}
    Therefore, $\<i,j>=\<i',j'>$.

  \item Suppose~$\<p,n,e>\outpop\<p',n',e'>$.  Then, by~\R{pop}, $p=p'$,
    $n>0$, $n'=n-1$, and~$e=e'=\nil$.
    %%
    By Definition~\ref{def.pot}, $\pot(\bcast(p,e))=\pot(\bcast(p',e'))$;
    hence~$i=i'$.  And by Definition~\ref{def.rank}, $j=n$, as~$e=\nil$,
    and~$j'=n-1$, as~$e'=\nil$ and~$n'=n-1$; hence~$j>j'$.
    Therefore,~${\<i,j>}>{\<i',j'>}$.\qedhere
  \end{enumerate}
\end{proof}


\begin{lemma}
  \label{lem.rank-nst}
  %%
  If~$\delta\nst\delta'$ then~$\rank(\delta)\ge\rank(\delta')$.
\end{lemma}
\begin{proof}
  We proceed by induction on the structure of~$\nst$ derivations.
  Let~$\delta=\<p,n,e>$, $\delta'=\<p',n',e'>$, $\rank(\delta)=\<i,j>$,
  and~$\rank(\delta')=\<i',j'>$.  By the theorem hypothesis, there is a
  derivation~$\pi$ such that
  \[
    \pi\Vdash\<p,n,e>\nst\<p',n',e'>\,.
  \]
  By definition of~$\nst$, $e=\nil$ and $n=n'$.  Depending on the structure
  of program~$p$, there are~11 possibilities.  In each one of them, we show
  that~$\rank(\delta)\ge\rank(\delta')$.

  \begin{case}
    $p=\ceu{\Mem(id)}$.
    %%
    Then~$\pi$ is an instance of~\R{mem}.  Hence~$p'=\ceu{\Nop}$
    and~$e'=\nil$.  Thus
    \[
      \rank(\delta)=\rank(\delta')=\<0,n>\,.
    \]
  \end{case}

  \begin{case}
    $p=\ceu{\EmitInt(e_1)}$.
    %%
    Then~$\pi$ is an instance of~\R{emit-int}.  Hence~$p'=\ceu{\Nop}$
    and~$e'=e_1\ne\nil$.
    Thus
    \[
      {\rank(\delta)={\<1,n>}}>{\<0,n+1>=\rank(\delta')}\,.
    \]
  \end{case}

  \begin{case}
    $p=\ceu{\CanRun(n)}$.
    %%
    Then~$\pi$ is an instance of~\R{can-run}.  Hence~$p'=\ceu{\Nop}$
    and~$e'=\nil$.  Thus
    \[
      \rank(\delta)=\rank(\delta')=\<0,n>\,.
    \]
  \end{case}

  \begin{case}
    $p=\ceu{\IfElse{p}{p_1}{p_2}}$.
    %%
    There are two subcases.
    \begin{subcase}
      \label{lem.rank-nst.if-true}
      $\eval(\ceu{\Mem(\Id)})$~is true.
      %%
      Then~$\pi$ is an instance of~\R{if-true}.  Hence~$p'=\ceu{p_1}$
      and~$e'=\nil$.  Thus
      \begin{align*}
        \rank(\delta)&=\<\max\{pot'(p_1),pot'(p_2)\},n>\\
                     &\ge\<\pot'(p_1),n>=\rank(\delta')\,.
      \end{align*}
    \end{subcase}
    \begin{subcase}
      $\eval(\ceu{\Mem(\Id)})$~is false.
      %%
      Similar to Case~\ref{lem.rank-nst.if-true}.
    \end{subcase}
  \end{case}

  \begin{case}
    $p=\ceu{p_1;\,p_2}$.
    %%
    There are three subcases.
    \begin{subcase}
      \label{lem.rank-nst.seq-nop}
      $p_1=\ceu{\Nop}$.
      %%
      Then~$\pi$ is an instance of~\R{seq-nop}.
      Hence~$p'=p_2$ and~$e'=\nil$.  Thus
      \begin{align*}
        \rank(\delta)&=\<\pot'(p_1)+\pot'(p_2),n>\\
                     &\ge\<pot'(p_2),n>=\rank(\delta')\,.
      \end{align*}
    \end{subcase}
    \begin{subcase}
      \label{lem.rank-nst.seq-brk}
      $p_1=\ceu{\Break}$.
      %%
      Then~$\pi$ is an instance of~\R{seq-brk}.
      Hence~$p'=p_1$ and~$e'=\nil$.  Thus
      \[
        \rank(\delta)=\<0,n>=\rank(\delta')\,.
      \]
    \end{subcase}
    \begin{subcase}
      \label{lem.rank-nst.seq-adv}
      $p_1\ne\ceu{\Nop},\ceu{\Break}$.
      %%
      Then~$\pi$ is an instance of~\R{seq-adv}.  Hence there is a
      derivation~$\pi'$ such that
      \[
        \pi'\Vdash\<p_1,n,\nil>\nst\<p_1',n,e_1'>\,,
      \]
      for some~$p_1'$ and~$e_1'$.  Thus~$p'=p_1';p_2$ and~$e'=e_1'$.  By the
      induction hypothesis,
      \begin{equation}
        \label{lem.rank-nst.seq-adv.eq1}
        \rank(\<p_1,n,\nil>)\ge\rank(\<p_1',n,e_1'>)\,.
      \end{equation}
      There are two subcases.
      \begin{subsubcase}
        $e'=\nil$.
        %%
        Then
        \begin{align*}
          \rank(\delta)&=\<\pot'(p_1)+\pot'(p_2),n>\enspace\text{and}\\
          \rank(\delta')&=\<\pot'(p_1')+\pot'(p_2),n>\,.
        \end{align*}
        By~\eqref{lem.rank-nst.seq-adv.eq1}, $\pot'(p_1)\ge\pot'(p_1')$.
        Thus
        \[
          \rank(\delta)\ge\rank(\delta')\,.
        \]
      \end{subsubcase}
      \begin{subsubcase}
        $e'\ne\nil$.
        %%
        Then~$\pi'$ contain one application of~\R{emit-int}, which consumes
        one~$\ceu{\EmitInt(e')}$ expression from~$p_1$ and implies
        $\pot'(p_1)>\pot'(p_1')$.  Thus
        \begin{align*}
          \rank(\delta)&=\<\pot'(p_1)+\pot'(p_2),n>\\
                       &>\<\pot'(p_1')+\pot'(p_2),n+1>=\rank(\delta')\,.
        \end{align*}
      \end{subsubcase}
    \end{subcase}
  \end{case}

  \begin{case}
    \label{lem.rank-nst.loop-expd}
    $p=\ceu{\Loop{p_1}}$.
    %%
    Then~$\pi$ is an instance of~\R{loop-expd}.
    Hence~$p'=\ceu{p_1\AtLoop{p_1}}$ and~$e'=\nil$.
    %%
    TODO: Use condition~(\dag) in definition of~$\pot'$.
  \end{case}

  \begin{case}
    $p=\ceu{{p_1}\AtLoop{p_2}}$.
    %%
    There are three cases.
    \begin{subcase}
      $p_1=\ceu{\Nop}$.
      %%
      Similar to Case~\ref{lem.rank-nst.seq-nop}.
    \end{subcase}
    \begin{subcase}
      $p_1=\ceu{\Break}$.
      %%
      Similar to Case~\ref{lem.rank-nst.seq-brk}.
    \end{subcase}
    \begin{subcase}
      \label{lem.rank-nst.loop-adv}
      $p_1\ne\ceu{\Nop},\ceu{\Break}$.
      %%
      Similar to Case~\ref{lem.rank-nst.seq-adv}.
    \end{subcase}
  \end{case}

  \begin{case}
    \label{lem.rank-nst.and.expd}
    $p=\ceu{{p_1}\And{p_2}}$.
    %%
    Then~$\pi$ is an instance of~\R{and-expd}.
    Hence~$p'=\ceu{{p_1}\AtAnd(\CanRun(n);\,p_2)}$ and~$e'=\nil$.
  \end{case}

  \begin{case}
    $p=\ceu{{p_1}\AtAnd{p_2}}$.
  \end{case}

  \begin{case}
    $p=\ceu{{p_1}\Or{p_2}}$.
  \end{case}

  \begin{case}
    $p=\ceu{{p_1}\AtOr{p_2}}$.
  \end{case}
\end{proof}

\begin{theorem}[Reaction termination]
  For any~$\delta$ there is a~$\delta'_\Hnst$ such
  that~$\delta\trans[*]\delta'$ and~$\rank(\delta')=\<0,0>$.
\end{theorem}
\begin{proof}
  By lexicographic induction on~$rank(\delta)$.
\end{proof}
%   \begin{basis}
%     If~$\<i,j>=\<0,0>$ then, by~\ref{?}, $\delta$ cannot be advanced
%     by~$\out$ transitions.  There are two possibilities\dots\ (it can or
%     cannot be advanced by nst transitions).
%   \end{basis}
%   \begin{induction}
%     Let~$\<i,j>\ne\<0,0>$ and suppose that the theorem holds for
%     all~$\delta''$ such that~$\rank(\delta'')<\rank(\delta)$.
%     \begin{case}
%       $\delta$~is nested-irreducible.
%       %%
%       \begin{subcase}
%         $\delta\outpush\delta_1$, for some~$\delta_1$.
%       \end{subcase}
%       \begin{subcase}
%         $\delta\outpop\delta_1$, for some~$\delta_1$.
%         %%
%         Then, by definition of~\R{pop}, $\rank(\delta_1)=\<i,j-1>$, and by
%         the induction hypothesis\dots
%       \end{subcase}
%     \end{case}
%     \begin{case}
%       $\delta$~is not nested-irreducible.
%       %%
%     \end{case}
%   \end{induction}

\nobalance
\clearpage
\nobalance
\section{Artifact Appendix}

%Submission and reviewing guidelines and methodology:
%{\em http://cTuning.org/ae/submission.html}

%%%%%%%%%%%%%%%%%%%%%%%%%%%%%%%%%%%%%%%%%%%%%%%%%%%%%%%%%%%%%%%%%%%%%
\subsection{Abstract}

Our artifact includes an open-source implementation of the programming language
\CEU.
The implementation is based on and should conform with the formal semantics
presented in this paper.
The artifact also includes an executable script with over 3500 test cases of
valid and invalid programs in \CEU.
The script is customizable and allows to create new tests providing inputs and
expected outputs.
The evaluation platform is a Linux/Intel with Lua-5.3 and GCC-7.2 installed.

\subsection{Artifact check-list (meta-information)}

{\small
\begin{itemize}
  %\item {\bf Algorithm: }
  %\item {\bf Program: }
  \item {\bf Compilation: } The output of the \CEU compiler is a C program that requires a C compiler (e.g., GCC-7.2).
  \item {\bf Transformations: } The compiler of \CEU is a Lua program (Lua-5.3) that generates a C program.
  %\item {\bf Binary: }
  \item {\bf Data set: } The data set is a set of program test cases included in the language distribution.
  \item {\bf Run-time environment: } Linux with Lua-5.3 and GCC-7.2. Root access is required to install a single executable file.
  \item {\bf Hardware: } An off-the-shelf Intel machine.
  %\item {\bf Run-time state: }
  \item {\bf Execution: } 5-10 minutes for the full test.
  \item {\bf Output: } Console output: Success (termination) or Fail (abortion).
  \item {\bf Experiments: } Manual steps performed by the user in the command line.
  %\item {\bf Workflow frameworks used?: }
  %\item {\bf Publicly available?: } Yes: \url{https://github.com/fsantanna/ceu}
\end{itemize}

\begin{itemize}
  \item {\bf Artifacts publicly available?:} Yes.
  \item {\bf Artifacts functional?:} Yes.
  \item {\bf Artifacts reusable?:} No.
  \item {\bf Results validated?:} No.
\end{itemize}

%%%%%%%%%%%%%%%%%%%%%%%%%%%%%%%%%%%%%%%%%%%%%%%%%%%%%%%%%%%%%%%%%%%%%
\subsection{Description}

\subsubsection{How delivered}

The compiler/distribution of \CEU is available on GitHub:
\url{https://github.com/fsantanna/ceu}

\subsubsection{Hardware dependencies}

An off-the-shelf Intel machine.

\subsubsection{Software dependencies}

Linux, Lua-5.3 (with lpeg-1.0.0), and GCC-7.2.

\subsubsection{Data sets}

A set of more than 3500 programs packed in a single script file which is
included with the compiler distribution.

%%%%%%%%%%%%%%%%%%%%%%%%%%%%%%%%%%%%%%%%%%%%%%%%%%%%%%%%%%%%%%%%%%%%%
\subsection{Installation}

Install all required software (assuming an Ubuntu-based distribution):

\begin{verbatim}
$ sudo apt-get install git lua5.3 lua-lpeg liblua5.3-0 \
                       liblua5.3-dev
\end{verbatim}

Clone the repository of \CEU:

\begin{verbatim}
$ git clone https://github.com/fsantanna/ceu
$ cd ceu/
$ git checkout v0.30
\end{verbatim}

Install \CEU:

\begin{verbatim}
$ make
$ sudo make install     # install as "/usr/local/bin/ceu"
\end{verbatim}

%%%%%%%%%%%%%%%%%%%%%%%%%%%%%%%%%%%%%%%%%%%%%%%%%%%%%%%%%%%%%%%%%%%%%
\subsection{Experiment workflow}

Run the experiment:

\begin{verbatim}
$ cd tst/
$ ./run.lua
\end{verbatim}

%%%%%%%%%%%%%%%%%%%%%%%%%%%%%%%%%%%%%%%%%%%%%%%%%%%%%%%%%%%%%%%%%%%%%
\subsection{Evaluation and expected result}

The experiment will execute all test cases.
In about 5-10 minutes, a summary will be printed on screen:

\begin{verbatim}
$ cd tst/
$ ./run.lua
<...> # output with the test programs
stats = {
    count  = 3483,
    trails = 9094,
    bytes  = 52555128,
    bcasts = 0,
    visits = 4065937,
}
\end{verbatim}

%%%%%%%%%%%%%%%%%%%%%%%%%%%%%%%%%%%%%%%%%%%%%%%%%%%%%%%%%%%%%%%%%%%%%
\subsection{Experiment customization}

The file \code{tst/tests.lua} includes all test cases.
Each test case contains a program in \CEU as well as the expected result, e.g.:

\begin{verbatim}
Test { [[
    var int ret = 0;
    var int i;
    loop i in [0 -> 10[ do
        ret = ret + 1;
    end
    escape ret;
]],
    run = 10,
}
\end{verbatim}

This test case should compile and run successfully yielding \code{10}.

To customize the experiment, include a new test case at line \code{440}, after
the string ``\code{--~OK:~well~tested}''.

%%%%%%%%%%%%%%%%%%%%%%%%%%%%%%%%%%%%%%%%%%%%%%%%%%%%%%%%%%%%%%%%%%%%%
%\subsection{Notes}

\nobalance

\end{document}
