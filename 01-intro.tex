\section{Introduction}
\label{sec.intro}

\CEU~\cite{ceu.sensys13,ceu.tecs17} is a Esterel-based~\cite{esterel.ieee91}
programming language for embedded soft real-time systems that aims to offer a
concurrent, safe, and expressive alternative to C with the characteristics that
follow:
%
\begin{description}
\item [Reactive:] code only executes in reactions to events.
\item [Structured:] programs use structured control mechanisms, such as
    \code{await} (to suspend a line of execution), and \code{par} (to combine
    multiple lines of execution).
\item [Synchronous:] reactions run atomically and to completion on each line of
    execution, i.e., there's no implicit preemption or real parallelism.
\end{description}
%
Structured reactive programming let developers write code in direct/sequential
style, recovering from the inversion of control imposed by event-driven
execution~\cite{rp.deprecating,rp.rescala,sync_async.cooperative}.
%
Synchronous languages offer a simple run-to-completion model of execution that
enable deterministic execution and make formal reasoning tractable.
For this reason, it has been successfully adopted in safety-critical real-time
embedded systems.~\cite{rp.twelve}

Previous work in the context of embedded sensor networks evaluates the
expressiveness of \CEU in comparison to event-driven code in C and attests a
reduction in source code size (around 25\%) with a small increase in memory
usage (around 5--10\% for \emph{text} and \emph{data})~\cite{ceu.sensys13}.
%
\CEU has also been used in the context of multimedia
systems~\cite{ceumedia.webmedia16} and games~\cite{ceu.mod15}, and as an
alternative language in an undergraduate-level course on embedded systems for
the past 6 years.

\CEU inherits the synchronous and imperative mindset of Esterel but adopts a
simpler semantics with fine-grained execution control.~\cite{ceu.tecs17}
%
The list that follows summarizes the semantic peculiarities of \CEU:
%
\begin{itemize}
    \item Stack-based execution for internal events, which provides a limited
          form of coroutines.
    \item Fine-grained, intra-reaction deterministic execution, which allows
          programs to safely share memory.
    \item Finalization mechanism for abortion of lines of execution, which
          safely release external resources.
    \item First-class synchronized timers.
\end{itemize}

In this work, we present a formal semantics for a subset of \CEU that focus on
its peculiarities in comparison to other synchronous languages.
\begin{itemize}
    \item qual a abordagem / operational semantics / dois passos
    \item quais os resultados / provas
    \item quais os desafios e limitações
    \item \gl{TODO}
\end{itemize}

\fs{Descrever seções.}
