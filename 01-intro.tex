\section{Introduction}
\label{sec.intro}

\CEU~\cite{ceu.sensys13} is a programming language for embedded soft real-time
systems that aims to offer a concurrent and expressive alternative to C and C++
with the characteristics that follow:

\begin{description}
\item [Reactive:] code only executes in reactions to events.
\item [Structured:] programs use structured control mechanisms, such as
    \code{await} (to suspend a line of execution), and \code{par} (to combine
    multiple lines of execution).
\item [Synchronous:] reactions run atomically and to completion on each line of
    execution, i.e., there's no implicit preemption or real parallelism.
\end{description}

Structured reactive programming eliminates the
\emph{callback hell}~\cite{TODO}, letting programmers write code in
direct/sequential style \cite{TODO}.
%
Previous work evaluates the expressiveness of \CEU in comparison to
event-driven code in C and attests a reduction in source code size (around
25\%) with a small increase in memory usage (around 5--10\% for \emph{text} and 
\emph{data})~\cite{ceu.sensys13}.
%
\CEU has been used in the context of wireless sensor
networks~\cite{ceu.sensys13,ceu.terra},
multimedia systems~\cite{ceu.media.webmedia16}, and
games \footnote{\fs{TODO}}.
It has also been used as an alternative language in an undergradutate-level
course on embedded systems for the past 5 years.

\fs{Embedded systems requirements, determinism, termination}

\begin{itemize}
\item \CEU
    \begin{itemize}
    \item variaveis compartilhadas
    \item concorrencia segura
    \item terminacao e determinismo
    \end{itemize}
\item Modelo Sincrono
    \begin{itemize}
    \item conhecido como ammenable to verification
    \item valido para \CEU?
    \end{itemize}
\end{itemize}

%only the static semantics for embedded systems

